\documentclass{standalone}
% font set
\usepackage{ctex}
\usepackage{fontspec}
\usepackage[T1]{fontenc}
\usepackage[sc]{mathpazo}
\usepackage{anyfontsize}
\setmainfont{Source Serif 4}
\setsansfont{Source Sans 3}
\setmonofont{Menlo}
\setCJKmainfont[BoldFont=黑体-简 中等,ItalicFont=楷体-简 常规体]{宋体-简 常规体}

% colors
\usepackage[dvipsnames]{xcolor}
\definecolor{pku-red}{RGB}{139,0,18}
\usepackage{colortbl}
\newcommand{\light}[1]{\textcolor{Orchid}{#1}}
\newcommand{\contrastlight}[1]{\textcolor{TealBlue}{#1}}

% plots
\usepackage{tikz}
\usepackage{tikz-cd}
\usetikzlibrary{arrows}
\usetikzlibrary{arrows.meta,positioning,calc,3d}
\usepackage{tikz-3dplot}
\usepackage{pgfplots}
\pgfplotsset{compat=newest}
\tikzset{
    punkt/.style={
        rectangle,
        rounded corners,
        draw=black, very thick,
        minimum height=2em,
        inner sep=6pt,
        text centered,
        fill=gray!30
    }
}

% math package
\let\Bbbk\relax
\usepackage{amsmath}
\usepackage{mathrsfs}
\usepackage{amssymb}
\usepackage{amsfonts}
\usepackage{stmaryrd}
\usepackage{latexsym}
\usepackage{extarrows}
\SetSymbolFont{stmry}{bold}{U}{stmry}{m}{n}


% math notations
\newcommand{\LHS}{\mathrm{LHS}}
\newcommand{\RHS}{\mathrm{RHS}}
\newcommand{\Z}{\mathbb{Z}}
\newcommand{\N}{\mathbb{N}}
\newcommand{\R}{\mathbb{R}}
\newcommand{\Q}{\mathbb{Q}}
\newcommand{\C}{\mathbb{C}}
\newcommand{\E}{\mathbb{E}}
\renewcommand{\O}{\mathcal{O}}
\newcommand{\id}{\mathrm{id}}
\DeclareMathOperator*{\Span}{Span}
\DeclareMathOperator*{\im}{Im}
\DeclareMathOperator*{\rank}{rank}
\DeclareMathOperator*{\card}{card}
\DeclareMathOperator*{\grad}{grad}
\DeclareMathOperator*{\argmax}{argmax}
\DeclareMathOperator*{\epi}{epi}
\DeclareMathOperator*{\maximize}{maximize}
\DeclareMathOperator*{\minimize}{minimize}
\renewcommand{\d}{\mathrm{d}}
\newcommand{\Pow}{\mathcal{P}}
\newcommand{\cov}{\mathsf{Cov}}
\newcommand{\var}{\mathsf{Var}}
\newcommand{\Nor}{\mathcal{N}}
\newcommand{\U}{\mathcal{U}}
\renewcommand{\t}{\mathsf{T}}
\newcommand{\T}{\top}
\newcommand{\F}{\bot}
\newcommand{\norm}[1]{\left\|#1\right\|}
\newcommand{\inner}[2]{\left\langle{#1},{#2}\right\rangle}
\newcommand{\e}{\mathrm{e}}
\newcommand{\const}{\mathrm{const}}
\newcommand{\scB}{\mathscr{B}}
\newcommand{\scF}{\mathscr{F}}
\newcommand{\G}{\mathscr{G}}
\newcommand{\Exp}{\mathsf{Exp}}
\newcommand{\DExp}{\mathsf{DExp}}
\newcommand{\Lap}{\mathsf{Lap}}
\newcommand{\calP}{\mathcal P}
\newcommand{\calS}{\mathcal S}
\newcommand{\calF}{\mathcal F}
\newcommand{\calM}{\mathcal M}
\newcommand{\KL}{\mathrm{KL}}
\newcommand{\ReLU}{\mathsf{ReLU}}
\newcommand{\val}{\mathsf{val}}

\DeclareSymbolFont{symbolsC}{U}{txsyc}{m}{n}
\DeclareMathSymbol{\strictif}{\mathrel}{symbolsC}{74}

\begin{document}

% Define the three-dimensional coordinate system and vector
\tdplotsetmaincoords{60}{100}

\begin{tikzpicture}[tdplot_main_coords]


    % Define the plane (xy-plane)
    \draw[Orchid!70,fill=Orchid!70,opacity=0.5] (0,0,0) -- (4,0,0) -- (4,4,0) -- (0,4,0) -- cycle;
    \node[Orchid] at (2,0.5,0) {$V$} ;

    % Define the vector and its projection on the plane
    \draw[very thick,-stealth,TealBlue] (1,1,0) -- (3,3,4)  node[anchor=west] {$b$};
    \draw[very thick,dashed,gray] (3,3,4) -- node[anchor=west] {$\norm{Ax^*-b}_2$} (3,3,0);
    \draw[very thick,-stealth,Salmon] (1,1,0) -- (3,3,0) node[anchor=south west] {$Ax^*$};

    \draw[thick,gray] (3,3,0.2) -- (2.8,2.8,0.2) -- (2.8,2.8,0);


    \draw[very thick,dashed,gray,opacity=0.4] (3,3,4) -- (3,2,0);
    \draw[very thick,-stealth,Salmon,opacity=0.4] (1,1,0) -- (3,2,0) ;

    \draw[very thick,dashed,gray,opacity=0.4] (3,3,4) -- (1,3,0);
    \draw[very thick,-stealth,Salmon,opacity=0.4] (1,1,0) -- (1,3,0) ;

    \draw[very thick,dashed,gray,opacity=0.4] (3,3,4) -- (3.5,1,0);
    \draw[very thick,-stealth,Salmon,opacity=0.4] (1,1,0) -- (3.5,1,0) ;
\end{tikzpicture}

\end{document}
