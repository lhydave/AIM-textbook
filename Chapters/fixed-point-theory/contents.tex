\chapter{不动点理论}\label{chap:fixed-point-theory}

考虑优化算法$A$,它在函数$f$上的的收敛性如何?算法运行所产生的点列记为$\{x_n\}$,它满足$x_{n+1}=A(x_n)$. 如果关注序列$x_n$本身,那么这是一个数学分析的思路,通过寻找不同量之间的联系,来分析收敛性. 如果从算法$A$本身来看,这是一个算子法与泛函分析的思路,研究算法本身的性质,收敛性往往归结为吸收点的存在性. 后者是更加抽象且现代的思路。在本章中,我们将看到,从算子的角度来理解收敛性,最终问题就归结到了\emph{不动点理论}\index{不动点理论}。

不动点的定义是非常直接的,考虑一个集合$X$以及它到自身的映射$f:X\to X$,元素$a\in X$称为映射$f:X\to X$的\textbf{不动点}\index{不动点},如果$f(a)=a$. 本章将介绍三种不动点存在性定理。

\section{Banach不动点定理}
为了陈述不动点定理,我们需要引入一些数学概念。

\begin{definition}[度量与度量空间]\index{度量}\index{度量空间}\index{距离}
集合$X$上的\textbf{度量}(或\textbf{距离})$d$是映射
\[d:X\times X\to \R\]
满足条件
\begin{itemize}
\item 非负性:$d(x_1,x_2)\geq 0$,并且$d(x_1,x_2)=0\iff x_1=x_2$.
\item 对称性:$d(x_1,x_2)=d(x_2,x_1)$.
\item 三角不等式:$d(x_1,x_3)\le d(x_1,x_2)+d(x_2,x_3)$.
\end{itemize}
其中$x_1,x_2,x_3$是$X$的任意元素. 

此时,$(X,d)$或$X$被称为\textbf{度量空间}.
\end{definition}

下面给出一些度量的例子。
\begin{example}
    考虑实数集$\R$,要成为度量空间,可以装备以下度量:
    \begin{itemize}
            \item 平凡的离散度量:$\forall x_1\neq x_2\ d(x_1,x_2)\equiv 1, d(x,x)=0$. \index{度量!离散~}
            \item $d(x_1,x_2)=|x_1-x_2|$. \index{度量!绝对值~}
        \end{itemize}
        考虑向量空间$\R^n$,要成为度量空间,可以装备以下度量:
        \begin{itemize}
            \item Minkowski度量($L^p$度量):$d(x_1,x_2)=(\sum_{i=1}^n|x_1^i-x_2^i|^p)^{1/p}\ (p\ge 1)$. \index{度量!Minkowski~}\index{度量!$L^p$~}
            \item Manhattan度量($L^1$度量): $d(x_1,x_2)=\sum_{i=1}^n|x_1^i-x_2^i|$. \index{度量!Manhattan~}\index{度量!$L^1$~}
            \item Euclid度量($L^2$度量):$d(x_1,x_2)=\sqrt{\sum_{i=1}^n|x_1^i-x_2^i|^2}$. \index{度量!Euclid~}\index{度量!$L^2$~}
            \item Chebyshev度量($L^\infty$度量): $d(x_1,x_2)=\max_i|x_1^i-x_2^i|=\lim_{p\to\infty}(\sum_{i=1}^n|x_1^i-x_2^i|^p)^{1/p}$. \index{度量!Chebyshev~}\index{度量!$L^\infty$~}
        \end{itemize}
\end{example}

我们的目标是找到一类和实数集非常像的度量空间。实数集一个非常重要的性质是实数列收敛当且仅当它是Cauchy列。我们把这一性质抽象出来,就得到了如下定义:

\begin{example}
    度量空间$(X,d)$的点列$\{x_n:n\in \N\}$称为\textbf{Cauchy列}\index{Cauchy列},如果对于任何$\epsilon>0$,都可以找到序号$N\in\N$,使得对于任何大于$N$的序号$m,n\in\N$,$d(x_m,x_n)<\epsilon$成立.

    度量空间$(X,d)$称为\textbf{完备的}\index{完备空间},如果任意Cauchy列$\{x_n:n\in \N\}$都收敛:$\exists a\in X, \lim_{n\to\infty}d(a,x_n)=0.$
\end{example}
度量空间的任何收敛点列显然是Cauchy列,完备性本质上只是假设Cauchy收敛准则在该空间成立.

下面是一些完备度量空间的例子
\begin{example}
\begin{itemize}
\item $L^p$度量下下$\R^n$是完备的.
\item 使用度量$d(x_1,x_2)=|x_1-x_2|$,则$X=\R\setminus\{0\}$不是完备度量空间. 考虑$\{x_n=\frac1n:n\in\N\}$,它是Cauchy列,但该点列在$X$中没有极限(极限是$0$).

\item $[0,1]$到自身的连续函数空间$C([0,1])$在$L^\infty$度量下是完备的.
此时
\[d(f,g)=\sup_{x\in[0,1]}|f(x)-g(x)|,\]
完备性由一致收敛得到,函数空间是泛函分析中一个典型的研究对象.
\end{itemize}
\end{example}

下面我们给出与Banach不动点定理相关的概念:

\begin{definition}[压缩映射]
度量空间$(X,d)$到自身的映射$f:X\to X$称为\textbf{压缩映射}\index{压缩映射},如果存在$0<q<1$,使得不等式
    \[d(f(x_1),f(x_2))\le q\cdot d(x_1,x_2)\]
对于$X$中的任何两个点$x_1,x_2$都成立.
\end{definition}

用$\delta-\epsilon$语言容易证明压缩映射一定是连续映射:
\begin{lemma}\label{lemma:contraction-continuous}
    压缩映射$f:X\to X$是连续映射.
\end{lemma}

压缩映射一定有不动点,这就是Banach不动点定理:
\begin{theorem}[Banach不动点定理,压缩映像原理]\label{thm:banach-fixed-point}\index{不动点定理!Banach~}\index{压缩映射原理}

完备度量空间$(X,d)$到自身的压缩映射$f:X\to X$具有唯一的不动点$a$. 

此外,对于任何点$x_0\in X$,迭代序列$x_0,x_1=f(x_0),\cdots,x_{n+1}=f(x_n),\cdots$收敛到$a$. 收敛速度由以下估计给出:
\[d(a,x_n)\le \frac{q^n}{1-q}d(x_1,x_0).\]
\end{theorem}

\begin{proof}
首先证明存在性。$d(x_{n+1},x_n)\le qd(x_n,x_{n-1})\le \cdots\le q^nd(x_1,x_0).$
从而
\begin{align*}
d(x_{n+k},x_n)&\le d(x_n,x_{n+1})+\cdots+d(x_{n+k-1},x_{n+k})\\
&\leq(q^n+\dots+q^{n+k-1})d(x_1,x_0)\le \frac{q^{n}}{1-q}d(x_1,x_0).
\end{align*}
因此$\{x_n\}$是Cauchy列,存在极限$\lim_{n\to\infty} x_n=a\in X$.
结合压缩映射的连续性,有
$a=\lim_{n\to\infty}x_{n+1} = \lim_{n\to\infty}f(x_n)=f(\lim_{n\to\infty}x_n)=f(a).$

然后证明唯一性。若$f$还有其他不动点$a_1,a_2$,则
$0\le d(a_1,a_2)=d(f(a_1),f(a_2))\le qd(a_1,a_2).$
而这当且仅当$d(a_1,a_2)=0$,即$a_1=a_2$时才可能成立.
最后证明收敛速度. 对
$d(x_{n+k},x_n)\le \frac{q^n}{1-q}d(x_1,x_0).$
取$k\to\infty$,得到
$d(a,x_n)\le \frac{q^n}{1-q}d(x_1,x_0).$
\end{proof}

\begin{example}[落在地面上的地图]
将一座公园的地图铺开在公园地面上,则地面上恰有唯一一点与地图上对应的点重合. 设公园可以用有界的面闭区域$\Omega$表示. 设地图的压缩比是$\lambda\in(0,1)$. 现在固定一个平面直角坐标系,把地图铺在区域$\Omega$内,则从$\Omega$内的点$x$(公园中的地点)到地图上对应点$x'$的变换由下面的公式给出:
\[x' = f(x) := \lambda Rx + b.\]
其中$R$和$b$分别为旋转和平移变换. 

考虑$\norm{\lambda R}=\sup_{\norm{x}=1}\norm{\lambda Rx}=\lambda<1$,由Banach不动点定理可知,压缩映射$f(x)$有唯一不动点$a=f(a)$.
\end{example}

\begin{example}[梯度下降的收敛性]\index{梯度下降}
我们优化目标是寻找二阶可微凸函数$f(x),x\in \R^n$的最小值. 使用梯度下降方法:每次往最小梯度方向移动.
假设对任意$x\in \R^n$,
    \[L \le \lambda_{\min}(\nabla^2 f(x))\le \lambda_{\max}(\nabla^2 f(x))\le U.\]
    其中$\nabla^2f(x)$是$f$的Hessian矩阵(二次导数),$U\ge L>0$为给定的常数,$\lambda_{\min}(A),\lambda_{\max}(A)$表示矩阵$A$的最小、最大特征值.

我们要证明:梯度下降能收敛到最小值点,且具有指数收敛速度.

先看一下证明的思路,我们要设法证明梯度下降算法是完备度量空间中的一个压缩映射。首先,二阶可微凸函数的最小值点充分必要地满足$\nabla f(x)=0$. 其次,$\nabla f(x)=0 \iff x\in\R^n$是梯度下降算子$\mathcal T^{(\alpha)}$的不动点,其中$\mathcal T^{(\alpha)}: x\mapsto x - \alpha\nabla f(x)$, 这里$\alpha\in \R_+$为步长. 最后,$\mathcal T^{(\alpha)}$是一个完备度量空间的压缩映射,其压缩系数为$q(\alpha)=1-L\alpha$. 因此梯度下降可以收敛至唯一的最小值点,收敛速度可以由压缩系数估计.

为了使$q(\alpha)$确实一个压缩系数,我们需要$\alpha<\min L^{-1}$. $\mathcal T^{(\alpha)}$的不动点恰好满足$\nabla f(x)=0$,因此是最小值点. 我们只需要证明$\mathcal T^{(\alpha)}$是压缩映射,并给出压缩系数

由有限增量原理:
    \[\norm{\mathcal T^{(\alpha)}x - \mathcal T^{(\alpha)}y} \le \sup_{z\in[x,y]}\norm{I - \alpha \nabla^2 f(z)}_2\cdot\norm{x-y}_2.\]
    \item 最后,注意到$\norm{I-\alpha\nabla^2f(z)}_2$等于$I-\alpha\nabla^2f(z)$特征值的最大模,根据条件可知特征值的最大模 $\leq 1-L\alpha$.
\end{example}

\section{Brouwer不动点定理}
下面我们考虑更一般的度量空间中的不动点定理,为此我们需要引入\emph{连续映射}的概念。

\begin{definition}
设$X$和$Y$是度量空间$(X,d_X),(Y,d_Y)$,映射$f:X\to Y$在点$a\in X$\textbf{连续}\index{连续},指的是
    \[\forall\epsilon>0\exists\delta>0\forall x\in X(d_X(a,x) < \delta\Rightarrow d_Y(f(a),f(x))<\epsilon).\]
如果它在每个点$x\in X$连续,称$f$为\textbf{连续映射}\index{连续映射}. $X$到$Y$的连续映射的集合记为$C(X,Y)$. 
\end{definition}

当度量空间为欧氏空间时,连续映射的定义与欧氏空间中连续映射的定义相同. 接下来,我们还需要几个集合的概念。

\begin{definition}[开集、闭集和紧集]
    考虑度量空间$(X,d)$,$a\in X$的邻域$B(a,\delta):=\{x\in X|d(a,x)<\delta\}$.
\begin{itemize}
    \item 集合$G$是\textbf{开集}\index{开集}$G$指的是对于任何点$x\in G$,满足$B(x,\delta)\subset G$的邻域$B(x,\delta)$存在.
    \item 集合$F$是\textbf{闭集}\index{闭集},如果它的补集$X\setminus F$是$(X,d)$中的开集.
    \item 集合$K$是\textbf{紧集}\index{紧集},如果从$X$中任何覆盖$K$的开集族中可以选出有限个开集来覆盖$K$.
\end{itemize}
\end{definition}
当度量空间为欧氏空间时,开集和紧集的定义与欧氏空间中的定义相同,紧集等价于有界闭集. 后一条在一般的度量空间不一定成立!

有了上面的准备,我们就可以叙述Brouwer不动点定理了:

\begin{theorem}[Brouwer不动点定理]\index{不动点定理!Brouwer~}
    设$M\subset\R^n$是一个非空紧凸集,而$F:M\to M$是一个连续函数. 则存在$x\in M$使得$F(x)=x$成立.
\end{theorem}

Brouwer不动点定理可以通过该实际的例子来理解:将一张白纸平铺在桌面上,再将它揉成一团(不撕裂),放在原来白纸所在的地方,那么只要它不超出原来白纸平铺时的边界,那么白纸上一定有一点在水平方向上没有移动过. 这个断言依据Brouwer不动点定理在$\R^2$的情况,因为把纸揉皱是一个连续的变换过程.

另一个例子:大商场等地方可以看到的平面地图,上面标有“您在此处”的红点. 如果标注足够精确,那么这个点就是把实际地形映射到地图的连续函数的不动点.

下面我们看一个Brouwer不动点定理的应用例子。首先引入矩阵\textbf{不可约}的概念:对于$n$阶方阵$A$而言,如果存在一个置换矩阵(通过交换单位阵的列获得)$P$使得$P^\t AP$为一个分块上三角阵,我们就称矩阵$A$是\emph{可约的},否则就称该矩阵是\emph{不可约的}.\index{不可约矩阵}

\begin{theorem}[Perron-Frobenius定理]\index{Perron-Frobenius定理}
    设$A=(a_{ij})$为$n\times n$不可约实矩阵,所有元素均非负,$a_{ij}\ge 0$,则下列结论成立.
    \begin{itemize}
        \item 存在一个实特征值$r$,其他特征值$\lambda$的模均不超过$r$,即$|\lambda|\le r$.
        \item 存在一个与$r$对应的特征向量,其所有元素恒正.
        \item $\min_i\sum_{j}a_{ij}\le r\le \max_i\sum_j a_{ij}$.
    \end{itemize}
\end{theorem}


\begin{proof}
首先证明$A$存在一个正的特征值$r>0$. 考虑单纯形$S:=\{x\in\R^n|x\ge 0,\sum_i x_i=1\}$. 则$\forall x\in S$,有$Ax\ge 0$. 

断言$Ax>0$,若不然,$A$存在某一列全$0$(由$x\ge 0$和$A$非负可得). 此时可通过置换阵将该$0$列交换到第一列,则得到的矩阵为分块上三角,与不可约性矛盾.

可以在$S$上定义映射
\[T(x) = \frac1{\rho(x)}Ax,\]
其中$\rho(x) > 0$使得$T(x)\in S$.

显然$T(x)$是$S\to S$的连续映射. $S$是一个有界凸闭集. 由Brouwer不动点定理,存在$x_0\in S, x_0 = T(x_0)=\frac1{\rho(x_0)}Ax_0$.

令$r=\rho(x_0)$,则可得$r$为$A$的一个正的特征值.

我们接下来证明,与$r$对应的特征向量所有元素恒正. 由之前的证明,与$r$对应的特征向量$x_0\in S$,则$x_0\ge 0$. 我们证明$x_0>0$.

考虑$A = PBP^{-1}$,其中$P$是置换矩阵,则
\[PBP^{-1}x_0=rx_0\implies B(P^{-1}x_0)=r(P^{-1}x_0).\]

记$\tilde{x}_0 = P^{-1}x_0$.取$B$使得$\tilde{x}_0 = (\xi,0)^\t, \xi > 0$. 则
\begin{align*}
    \left(                 %左括号
        \begin{array}{cc}   
        B_{11} & B_{12}\\  
        B_{21} & B_{22}\\  
        \end{array}
    \right)
    \left(                 
        \begin{array}{c}   
        \xi\\  %第一行元素
        0\\
        \end{array}
    \right)=
    \left(                 
        \begin{array}{c}   
        r\xi\\  %第一行元素
        0\\
        \end{array}
    \right).
\end{align*}

此时$B_{21}\xi=0$,由$\xi>0$可得$B_{21}=0$. 这与不可约矛盾,因此$x_0 > 0$.

然后我们证明:若$\alpha$是$A$的任意特征值,有$|\alpha|\le r$. 设$0\le B\le A, By=\beta y$. 记$y^\star = |y|=(|y_i|)_i$. 于是有
\[|\beta|y^\star = |\beta y| = |By| \le By^\star\le Ay^\star.\]

由$A^\t$不可约,存在特征值$r_1>0$和特征向量$x_1>0, A^\t x_1 = r_1x_1$. 因此有
\[|\beta|x_1^\t y^\star \le x_1^\t Ay^\star = r_1x_1^\t y^\star.\]

由$x_1^\t y^\star>0$,则$|\beta|\le r_1$. 令$B=A$可得$|\alpha|\le r_1$,特别地$r\le r_1$. 同样有$r_1\le r$,故$r=r_1$. 

最后证明:
\[\min_i\sum_j a_{ij}\le r\le \max_i\sum_j a_{ij}.\]
以这样的方式获得$\tilde A$: 将$A$的每一行都扩增(不减小某个元素),使得每一行都达到$\max_i\sum_j a_{ij}$. 此时$\max_i\sum_j a_{ij}$成为$A$的一个正特征值,且特征向量$\tilde{x}_0=\frac 1n\cdot\mathbf{1}\in S$. 由之前的结论,假设$0\le A\le \tilde A$,可以得到$\tilde A$的正特征值$\tilde r\ge r$. 因此$r \le \max_i\sum_j a_{ij}$. 同理缩小$A$可得$\min_i\sum_j a_{ij}\le r$.
\end{proof}

Perron-Frobenius定理在Markov链中的有非常重要应用。回忆Markov链的平稳分布:满足矩阵方程$\pi = \pi P$和$\sum_i\pi_i=1$.  设该Markov链状态有限且对应的转移矩阵$P$是(非负实)不可约方阵. 由Perron-Frobenius定理,$P$存在一个特征值$1=\min_i\sum_ja_{ij}\le r\le\max_i\sum_j a_{ij}=1$,对应一个正特征向量$x_0\in S=\{x\in\R^n|x\ge 0,\sum_ix_i=1\}$. 因此不可约有限状态Markov链必然存在平稳遍历分布.


\section{不动点的一般视角}
    