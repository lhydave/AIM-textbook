\chapter{模态逻辑基础}\label{chap:modal-logic}


\begin{frame}{动机}
    \begin{itemize}
        \item 人工智能的讨论不可避免要接触到很多人独有的哲学概念:认知、信念、知识、理解、情感、意识……
        \item 我们需要有一套恰当的数学工具来表述这些哲学概念,从而算法化、自动化地模拟人.
        \item 过去,现在,乃至未来,最为成功的数学模型就是模态逻辑.
        \item 很多不精确的哲学讨论可以通过逻辑的方式形式化、数学化,最后算法化.
        \item 模态逻辑已经在计算机科学中起到了重要的作用(模型验证、形式化方法),它势必会在人工智能中也起到根基性的作用.
    \end{itemize}
    \end{frame}
    
    \section{模态逻辑的起源}
    
    \begin{frame}{三段论}
    \begin{itemize}
        \item 早在亚里士多德的时期,模态逻辑的概念就被提了出来.
        \item 回忆:亚里士多德的强三段论(Barbara XXX)是有效的:
        \begin{itemize}
            \item 大前提:所有$A$都是$B$.
            \item 小前提:所有$B$都是$C$.
            \item 结论:因此,所有$A$都是$C$.
        \end{itemize}
    
        \item 人都会死,苏格拉底是人,所以苏格拉底会死.
        \item 三段论可以进行各种形式的扩展.
    \end{itemize}
    \end{frame}
    
    \begin{frame}{三段论}
    \begin{itemize}
        \item 三段论可以进行各种形式的扩展.
        \item 加入量词:\light{任何对象},如果这个对象是人,那么它会死,苏格拉底这个对象是人,所以苏格拉底会死.
        \item 加入性质词:肯定的、否定的
        \item 加入\emph{模态词}(mode):无效、可能、必然、根据情况……
    \end{itemize}
    \end{frame}
    
    \begin{frame}{三段论}
    \begin{itemize}
        \item 亚里士多德也考虑过模态三段论.
        \item 亚里士多德认为如下模态三段论(Barbara LXL)是有效的:% HW: 写一个具体的LLL的例子
        \begin{itemize}
            \item 大前提:所有$A$都必然是$B$.
            \item 小前提:所有$B$都是$C$.
            \item 结论:因此,所有$A$都必然是$C$.
        \end{itemize}
        \item 然而,他认为如下的模态三段论(Barbara XLL)不是有效的:
            \begin{itemize}
            \item 所有$A$都是$B$.
            \item 所有$B$都必然是$C$.
            \item 因此,所有$A$都必然是$C$.
        \end{itemize}
        \item 所有通班同学都是单身汉,所有单身汉都必然是男性. 那么是否有:所有通班同学都必然是男性?
    \end{itemize}
    \end{frame}
    
    \begin{frame}{三段论}
    \begin{itemize}
        \item 类似的例子是,从物(De re)和从言(De dicto):
            \begin{center}
            \it 我觉得有人作弊.
        \end{center}
        \begin{itemize}
            \item 这句话有两种解读方式.
            \item 从物:$\exists x$(我觉得$:x$作弊).
            \item 从言:我觉得($\exists x:x$作弊).
        \end{itemize}
        \item 模态三段论的讨论并没有流行起来,因为它并没有非常干净漂亮的、符合直观的定义.
    \end{itemize}
    \end{frame}
    
    \begin{frame}{非经典逻辑}
    \begin{itemize}
        \item 回忆:经典逻辑中的语义等值
        \[p\to q\iff\neg p\vee q.\]
        \item 然而,在哲学上,这两者是不一样的含义.
        \item $p$:$1+1>2$,$q$:太阳从东边升起.
        \begin{itemize}
            \item “如果$1+1>2$,那么太阳从东边升起”是毫无道理的.
            \item 然而,“或者$1+1\not>2$,或者太阳从东边升起”是含义清晰的.
        \end{itemize}
        \item $p_n$:$\pi$的小数位包含连续的$n$个1.
        \begin{itemize}
            \item $p_{100}\to p_{99}$是显然的,然而$\neg p_{100}\vee p_{99}$并不直观!
        \end{itemize}
    \end{itemize}
    \end{frame}
    
    \begin{frame}{非经典逻辑}
    
    \begin{itemize}
        \item C. I. Lewis \emph{严格蕴含}(strict implication):
         $p\strictif q$.
        \begin{itemize}
            \item 
            %\light{(}
            \light{必然有}当$p$是真的时候,$q$是真的.
            \item 也可以说,\light{不可能}有$p$是真且$q$是假.
        \end{itemize}
            \item 实质蕴含(materially imply)
        $p\rightarrow q \iff
        \lnot {p}\vee q$
        \begin{itemize}
              \item 允许有$p$假但是$q$真(false negatives).
        \end{itemize}
        \item 换句话说:$p$严格蕴含$q$当且仅当\light{必然有} $p$实质蕴含$q$.
    \end{itemize}
    \end{frame}
    
    \begin{frame}{非经典逻辑}
    \begin{itemize}
        \item 还有很多重言式也是不合乎常理的.
        \begin{itemize}
            \item $p\to(q\to p)$.
            \item $(p\to q)\vee (q\to r)$.
        \end{itemize}
        \item Brouwer\emph{直觉逻辑}(intuitionistic logic),不承认反证法. 
        \begin{itemize}
            \item 因而否定和蕴含的含义发生了变化,例如$\neg\neg p$不再等价于$p$.
        \end{itemize}
    \end{itemize}
    \end{frame}
    
    \begin{frame}{非经典逻辑}
    \begin{itemize}
        \item 非经典逻辑本质上说,都是尝试将\emph{元语言}(meta language)中的概念拿到\emph{对象语言}(objective language)中.
        \begin{itemize}
        \item 例如,元语言:自然语言,对象语言:经典逻辑形式系统.
        \end{itemize}
        \item 在经典逻辑中,必然、可能、过去、未来、知识、信念、可证明等概念都没有办法表示,因此模态逻辑的解决方案是:将这些元语言的概念拿到对象语言中,并进行形式化.
    \end{itemize}
    \end{frame}
    
    \section{模态语言}
    
    \begin{frame}{基本命题模态语言}
    \begin{itemize}
        \item 我们只考虑最简单的情况,没有量词,只有一个\emph{模态算子}(modality operator).
        \item \emph{基本模态语言}(basic modal logic language)$L$ 可以按照如下定义递归生成:
        \begin{itemize}
            \item 命题字母$p\in \mathbf P$属于$L$,$\top$属于$L$.
            \item 如果$\phi$属于$L$,那么$\neg\phi$ 和$\Box\phi$也属于$L$.
            \item 如果$\phi_1,\phi_2$属于$L$,那么$(\phi_1\wedge\phi_2)$也属于$L$.
        \end{itemize}
        \item $\Box$:读作“Box”.
        \item 更便捷的记号是使用Backus-Naur范式(BNF):
        \[\phi\quad::=\quad p\mid \top\mid \neg\phi\mid (\phi\wedge\phi)\mid \Box\phi.\] 
    \end{itemize}
    \end{frame}
    \begin{frame}{基本命题模态语言}
    \begin{itemize}
        \item 类似命题逻辑,我们有如下缩写:
        \begin{itemize}
            \item $\phi\vee\psi\iff\neg(\neg \phi\wedge\neg\psi)$.
            \item $\phi\to\psi\iff\neg\phi\vee\psi$.
            \item $\bot\iff\neg\top$.
        \end{itemize}
        \item 这些缩写意味着,我们对Boole连接词,依然保持经典逻辑的含义.
        \item 非经典性只体现在模态算子$\Box$.
        \item 对偶模态算子$\Diamond$定义为$\neg\Box\neg$,读作“diamond”.
        \begin{itemize}
            \item 类比:$\exists=\neg\forall\neg$.
        \end{itemize}
    \end{itemize}
    \end{frame}
    \begin{frame}{}
        \begin{quotation}
        模态逻辑的哲学:多视角下看同一个数学概念.
        \end{quotation}
    \end{frame}
    
    \begin{frame}{模态逻辑的例子}
    \begin{itemize}
        \item 基本模态逻辑(basic modal logic):可能/必然是
        \item 时序逻辑(temporal logic):将会是
        \item 道义逻辑(deontic logic):被允许是
        \item \light{认知逻辑(epistemic logic):被知道是}
        \item 可证性逻辑(provability logic):可以被证明是
        \item 动态逻辑(dynamic logic):(在经过某些程序步骤之后)会是
        \item 联盟逻辑(coalition logic):被(她的父母)确保是
        \item 特征逻辑和描述逻辑(feature logic and description logic):具有的属性是
    \end{itemize}
    \end{frame}
    
    \begin{frame}{模态算子的解读:基础语义}
    \begin{itemize}
        \item 我们可以把模态算子$\Box$读成“必然”.
        \item $\Box\phi$:必然有$\phi$.
        \item $\Diamond\phi$:不是必然有非$\phi$,即可能有$\phi$.
        \item 因此,$\Diamond$读作“可能”.
        \item 反之,$\Box\phi$也可以读作:不可能有非$\phi$,即必然有$\phi$.
        \item 因此$\Diamond$和$\Box$确实是对偶的.
    \end{itemize}
    \end{frame}
    \begin{frame}{基础语义:例子}
    \begin{itemize}
        \item $\Box p\to\Diamond p$:必然的是可能的.
        \item $p\to\Box p$:真的是必然的.
        \item $\Diamond p\to\Box\Diamond p$:可能的是必然可能的.
        %HW: $p\to\Box\Diamond p$ 的解释,以及是否合理.
    \end{itemize}
    \end{frame}
    \begin{frame}{模态算子的解读:认知逻辑}
    \begin{itemize}
        \item 我们可以把模态算子$\Box$读成“知道”,并写成$K$(know).
        \item 于是,$K$表示某个特定的个体对世界的认知.
        \item $K\phi$(即$\Box\phi$):我知道$\phi$.
        \item $K\phi\to\phi$:如果我知道$\phi$,那么$\phi$是真的.
        \item $\phi\to K\phi$:如果$\phi$是真的,那么我知道$\phi$.
        \item $\neg K\phi$ vs. $K(\neg\phi)$.
        \begin{itemize}
            \item 我不知道上帝存在 vs. 我知道上帝不存在.
        \end{itemize}
    \end{itemize}
    \end{frame}
    \newcommand{\PA}{\mathbf{PA}}
    \begin{frame}{模态算子的解读:可证性逻辑}
    \begin{itemize}
        \item 我们可以把模态算子$\Box$读成“可证明”.
        \item $\Box\phi$:$\phi$是可证明的.
        \item 考虑Peano算术系统$\PA$,即一阶逻辑加上Peano公理.
        \item 符号$\PA\vdash \phi$表示$\phi$可以由$\PA$演绎出,即$\phi$可以被证明.
        \item Löb定理:如果$\PA\vdash\mathsf{Prov}(\ulcorner\phi\urcorner)\to\phi$,那么$\PA\vdash\phi$.
        \begin{itemize}
            \item 如果可以证明“如果$\phi$是可证明的,那么$\phi$是真的”,那么就可以证明$\phi$.
        \end{itemize}
        \item 在可证逻辑中,它对应Löb公式:$\Box(\Box\phi\to\phi)\to\Box\phi$.
    \end{itemize}
    \end{frame}
    
    \begin{frame}{命题模态逻辑:一般情形}
    \begin{itemize}
        \item 一般地,我们可以考虑多个模态算子、一个模态算子涉及多个公式的情形.
        \item \emph{模态语言类型}(modal similarity type)是一个元组$(O,\rho)$,其中$O$是一个模态算子$\nabla$的非空集合,$\rho:O\to\N$表示每一个模态算子的元数.
        \item 多元模态语言的BNF为:
        \[\phi\quad::=\quad p\mid \top\mid \neg\phi\mid (\phi\wedge\phi)\mid \nabla(\underbrace{\phi,\dots,\phi}_{\rho(\nabla)}),\]
        其中$p\in \mathbf P$,$\nabla\in O$.
        \item 类似地,定义对偶模态算子$\triangle(\phi_1,\dots,\phi_k)$为$\neg\nabla(\neg\phi_1,\dots,\neg\phi_k)$.
    \end{itemize}
    \end{frame}
    
    \begin{frame}{例子:时序逻辑}
    \begin{itemize}
        \item 基础时序逻辑有两个一元模态算子:$G$和$H$.
        \begin{itemize}
            \item $G\phi$:未来总会有$\phi$(always $G$oing to be).
        \item $H\phi$:过去总有$\phi$(always $H$as been).
        \end{itemize}
        \item 他们的对偶算子是:$F$和$P$.
        \begin{itemize}
             \item $F\phi$:在未来某个时刻会有$\phi$(be true at some $F$uture time).
        \item $P\phi$:在过去某个时刻有$\phi$(was true at some $P$ast time).
        \end{itemize}
        \item 还可以加入一个“直到”($U$ntil)算子$U(\phi,\psi)$:直到$\phi$发生都有$\psi$.
    \end{itemize}
    \end{frame}
    
    \begin{frame}{例子:时序逻辑}
    \begin{itemize}
        \item $P\phi\to GP\phi$:如果过去发生过$\phi$,那么$\phi$在未来总会发生过.
        \item $F\phi\to FF\phi$:如果未来某个时刻会有$\phi$,那么在未来的某个时刻会发生:未来的某个时刻会有$\phi$.
        \begin{itemize}
            \item 这一公式意味着时间是稠密的.
        \end{itemize}
        \item McKinsey公式$GF p\to FG p$:如果原子的信息总会在某个未来时刻为真,那么他会在未来某个时刻之后变得总为真.
    \end{itemize}
    \end{frame}
    
    \begin{frame}{例子:认知逻辑}
    \begin{itemize}
        \item 基本模态算子:$K_a$:个体$a$知道,$B_a$:个体$a$相信.
        \item 例:$K_aK_b\phi\leftrightarrow K_bK_a\phi$:我知道你知道$\phi$当且仅当你知道我知道$\phi$.
        \item 此外,我们也可以加入共同知识算子$C$,$C \phi$当且仅当$K_a(\phi\wedge C\phi)$对任意$a$成立.
        \begin{itemize}
            \item 注意,$C\phi$并不等价于$K_a\phi$,$\forall a$.
        \end{itemize}
        \item 我们也可以加入二元的相对算子.
        \begin{itemize}
            \item 相对共同知识$C^r(\phi,\psi)$:当所有人都知道$\psi$        时,所有人具有共同知识$\phi$.
            \item 条件信念$B_a(\phi,\psi)$:当$\psi$为真时,个体$a$相信$\phi$.
        \end{itemize} 
    \end{itemize}
    \end{frame}
    
    \begin{frame}{*例子:命题动态逻辑}
    \begin{itemize}
        \item 命题动态逻辑(PDL),有无穷多个模态算子.
        \item 模态算子被记为$[\pi]$,这里$\pi$按照程序来理解.
        \item $[\pi]\phi$解释为:从当前状态开始运行程序$\pi$,任何一种终止状态,$\phi$都成立.
        \item 它的对偶算子记为$\langle\pi\rangle$.
        \item $\langle\pi\rangle\phi$解释为:从当前状态开始运行程序$\pi$,存在一种终止状态,$\phi$成立.
    \end{itemize}
    \end{frame}
    
    \begin{frame}{*例子:命题动态逻辑}
    \begin{itemize}
        \item PDL的重要区别在于:我们可以用模态算子来构造新的模态算子.
        \item 基本程序:$a,b,\dots$.
        \item 我们可以用三种操作构造新的程序(模态算子):
        \begin{itemize}
            \item 选择:如果$\pi_1,\pi_2$是程序,那么$\pi_1\cup\pi_2$也是程序,它(非确定性地)执行$\pi_1$或$\pi_2$. 模态算子为$[\pi_1\cup\pi_2]$和$\langle\pi_1\cup\pi_2\rangle$.
            \item 复合:如果$\pi_1,\pi_2$是程序,那么$\pi_1;\pi_2$也是程序,它先执行$\pi_1$再执行$\pi_2$. 模态算子为$[\pi_1;\pi_2]$和$\langle\pi_1;\pi_2\rangle$.
            \item 迭代:如果$\pi$是程序,那么$\pi^*$也是程序,它执行$\pi$有限次(可能是零次). 模态算子为$[\pi^*]$和$\langle\pi^*\rangle$.
        \end{itemize}
        \item 以上构造得到的PDL被称为\emph{正则PDL}.
        \item 我们还可以引入交$\pi_1\cap\pi_2$,表示并行计算;也可以引入条件程序$\phi?$,其中$\phi$是公式.
    \end{itemize}
    \end{frame}
    
    \begin{frame}{*例子:命题动态逻辑}
    \begin{itemize}
        \item $\left\langle\pi^*\right\rangle \phi \leftrightarrow \phi \vee\left\langle\pi ; \pi^*\right\rangle \phi$.
        \begin{itemize}
            \item 在执行$\pi$有限次数后到达一个带有信息$\phi$的状态当且仅当要么我们已经在当前状态中拥有信息$\phi$,要么我们可以执行一次$\pi$,然后在有限次数的$\pi$迭代后找到一个带有信息$\phi$的状态.
        \end{itemize}
        \item Segerberg公理(归纳公理):$[\pi^*](\phi \to [\pi]\phi) \to (\phi \to [\pi^*]\phi).$
        \begin{itemize}
            \item 思考:这个公式的含义是什么?
        \end{itemize}
        \item 如何用模态算子表示
        \texttt{if $\phi$ then $a$ else $b$}?
        \begin{itemize}
            \item $(\phi?;a)\cup(\neg\phi?;b)$.
        \end{itemize}
    \end{itemize}
    \end{frame}
    
    \section{Kripke语义与框架语义}
    \begin{frame}{模态逻辑的模型论} 
    \begin{itemize} 
    \item 一个逻辑框架是一个三元组:(语言,模型,语义) $(\mathbf L,C,\vDash)$.
    \item 命题逻辑的例子:
    \begin{itemize}
        \item 语言:命题公式的集合$\{\top,\bot,p,p\vee q,\dots\}$.
        \item 模型:常值和命题字母的真假:$\top:\top$,$p:\top$,$q:\bot$,等等.
        \item 语义:Boole函数的真值表递归定义.
    \end{itemize}
    \item 对于基本模态逻辑,我们有如下要素:
    \begin{itemize}
        \item 语言:基本模态语言$\mathbf{ML}(\mathbf P, \Box)$.
        \item 模型:Kripke模型.
        \item 语义:Kripke语义.
    \end{itemize}
    \end{itemize} 
    \end{frame}
    
    \begin{frame}{Kripke模型和框架} 
    \begin{itemize} 
    \item 一个\emph{Kripke模型}(\emph{关系模型},Kripke model)可以看作是一个带有标记的有向边和节点的图: 
    \begin{itemize} 
    \item 节点表示可能的世界,状态或对象等,用命题字母标记; 
    \item 边表示节点之间的关系,用模态算子标记.
    \end{itemize} 
    \item 一个\emph{框架}(frame)是一个没有节点标记的模型.
    \end{itemize} 
    \end{frame}
    
    \begin{frame}{可能世界语义} 
    \begin{itemize} 
    \item 我们将节点解读为\emph{可能世界}(possible world).
    \item 此时$\Box$被理解为“必然”,$\Diamond$被理解为“可能”.
    \item $\Box\phi$在当前世界为真当且仅当$\phi$在当前世界的所有可能的替代世界上为真.
    \item 形式化:$\Box\phi$在世界$w$上成立当且仅当$\phi$在$w$的所有后继上为真.
    \item 这种语义通常被称为Kripke语义或可能世界语义. 一个世界的意义取决于它与其他世界的联系.
    \end{itemize} 
    \end{frame}
    
    \begin{frame}{状态语义}
    \begin{itemize}
    \item 我们将节点解读为\emph{状态}(state).
    \item 于是边就被解读为状态的转移.
    \item $\Box\phi$在当前状态为真当且仅当$\phi$在所有可能转移到的状态上为真.
    \item PDL 可以用以上语义来理解.
    \item 在哲学中,状态往往是不完全可观测的,此时模态逻辑可以被理解为不完全信息中可以确定的性质.
    \end{itemize}
    \end{frame}
    
    \begin{frame}{对象语义}
    \begin{itemize}
    \item 我们将节点解读为\emph{对象}(object).
    \item $w$有边指向$v$意味着$w$是$v$包含的一个整体,$v$是$w$的一个部分.
    \item 在哲学上,模态逻辑可以讨论整体论与还原论.
    \item $\Box\phi$对一个对象为真当且仅当$\phi$在它的所有部分都为真.
    \end{itemize}
    \end{frame}
    
    \begin{frame}{Kripke模型:基本情形}
    \begin{itemize}
        \item 考虑$\mathbf{ML}(\mathbf{P},\Box)$.
        \item 一个\emph{Kripke框架}(Kripke frame)指的是元组$\mathcal F=(W,R)$,其中
        \begin{itemize}
            \item $W$是非空集合(可能世界集);
            \item $R\subseteq W\times W$是一个$W$上的二元关系(边).
        \end{itemize}
        \item 一个\emph{Kripke模型}(Kripke model)$\mathcal{M}$指的是元组$(\mathcal F,V)$,其中$\mathcal F$是框架,$V:\mathbf P\to 2^W$是\emph{赋值函数}(valuation function).
        \item 一个\emph{Kripke点模型}(pointed Kripke model)$(\mathcal{M},w)$是Kripke模型$\mathcal M$加上一个指定的点$w\in W$.
    \end{itemize}
    \end{frame}
    
    \begin{frame}{Kripke模型:基本情形}
    \begin{figure}
        \centering
        \begin{tikzpicture}[node distance=2cm]
\node (w1) [circle,draw] {$w_1$};
\node (w2) [circle,draw,right=of w1] {$w_2$};
\node (w3) [circle,draw,below=of w1] {$w_3$};
\node (w4) [circle,draw,right=of w3] {$w_4$};
\draw[-Latex] (w1) -- (w2);
\draw[-Latex] (w1) -- (w3);
\draw[-Latex] (w2) -- (w4);
\draw[-Latex] (w3) -- (w4);
\node [above left=1pt of w1] {$p$};
\node [above right=1pt of w2] {$p,q$};
\node [below left=1pt of w3] {};
\node [below right=1pt of w4] {$q$};
\end{tikzpicture}
    \end{figure}
    \end{frame}
    
    \begin{frame}{Kripke模型:一般情形}
    \begin{itemize}
        \item 考虑$\mathbf{ML}(\mathbf{P},(O,\rho))$.
        \item 一个\emph{Kripke框架}指的是元组$\mathcal F=(W,\{R_\nabla:\nabla\in O\})$,其中
        \begin{itemize}
            \item $W$是非空集合(可能世界集);
            \item $R_\nabla$是一个$W$上的$\rho(\nabla)+1$元关系.
        \end{itemize}
        \item 一个\emph{Kripke模型}$\mathcal{M}$指的是元组$(\mathcal F,V)$,其中$\mathcal F$是框架,$V:\mathbf P\to 2^W$是\emph{赋值函数}(valuation function).
        \item 一个\emph{Kripke点模型}(pointed Kripke model)$(\mathcal{M},w)$是Kripke模型$\mathcal M$加上一个指定的点$w\in W$.
    \end{itemize}
    \end{frame}
    
    \begin{frame}{Kripke语义:基本情形}
    \begin{itemize}
        \item 考虑$\mathbf{ML}(\mathbf{P},\Box)$.
        \item 符号$\mathcal M,w\vDash\phi$表示$\phi$在点模型$\mathcal M,w$是\emph{可满足的}(satisfiable).
        \item 这一个概念可以递归定义如下
        \begin{itemize}
            \item $\mathcal M, w\vDash\top\iff$ 总是.
            \item $\mathcal M, w\vDash p\iff p\in V(w)$.
            \item $\mathcal M, w\vDash (\phi\wedge\psi)\iff\mathcal M,w\vDash\phi$且$\mathcal M,w\vDash\phi$.
            \item $\mathcal M, w\vDash \neg\phi\iff\mathcal M,w\not\vDash\phi$.
            \item $\mathcal M, w\vDash \Box\phi\iff$对所有$v$,如果$wRv$,那么$\mathcal M,v\vDash\phi$.
        \end{itemize}
        \item 因此,$\mathcal M, w\vDash \Diamond\phi\iff$存在$v$满足$wRv$使得$\mathcal M,v\vDash\phi$.
    \end{itemize}
    \end{frame}
    
    \begin{frame}{Kripke语义:基本情形}
    \begin{figure}
        \centering
        \begin{tikzpicture}[node distance=2cm]
\node (w1) [circle,draw] {$w_1$};
\node (w2) [circle,draw,right=of w1] {$w_2$};
\node (w3) [circle,draw,below=of w1] {$w_3$};
\node (w4) [circle,draw,right=of w3] {$w_4$};
\draw[-Latex] (w1) -- (w2);
\draw[-Latex] (w1) -- (w3);
\draw[-Latex] (w2) -- (w4);
\draw[-Latex] (w3) -- (w4);
\node [above left=1pt of w1] {$p$};
\node [above right=1pt of w2] {$p,q$};
\node [below left=1pt of w3] {};
\node [below right=1pt of w4] {$q$};
\end{tikzpicture}
    \end{figure}
    \begin{itemize}
        \item 对哪些$i$来说,$\mathcal M,w_i\vDash\Box (p\to\Diamond q)$?
    \end{itemize}
    \end{frame}
    
    \begin{frame}{Kripke语义:一般情形}
    \begin{itemize}
        \item 考虑$\mathbf{ML}(\mathbf{P},(O,\rho))$.
        \item 符号$\mathcal M,w\vDash\phi$表示$\phi$在点模型$\mathcal M,w$是\emph{可满足的}.
        \item 这一个概念可以递归定义如下
        \begin{itemize}
            \item $\mathcal M, w\vDash\top\iff$ 总是.
            \item $\mathcal M, w\vDash p\iff p\in V(w)$.
            \item $\mathcal M, w\vDash (\phi\wedge\psi)\iff\mathcal M,w\vDash\phi$且$\mathcal M,w\vDash\psi$.
            \item $\mathcal M, w\vDash \neg\phi\iff\mathcal M,w\not\vDash\phi$.
            \item $\mathcal M, w\vDash \nabla(\phi_1,\dots,\phi_{\rho(\nabla)})\iff$对\light{任意} $w_1,w_2,\dots w_{\rho(\nabla)}$,如果$R(w,w_1,\dots,w_{\rho(\nabla)})$,那么\light{存在} $w_i$使得$\mathcal M,w_i\vDash\phi_1$.
        \end{itemize}
        \item 思考:为什么要这么定义$\nabla$的语义?
    \end{itemize}
    \end{frame}
    
    \begin{frame}{Kripke语义:一般情形}
    \begin{itemize}
        \item 如果一个模态算子对应的关系是二元关系,我们就称这个模态算子是\emph{一元的}(unary).
        \item 此时,关系$wRv$可以记为$w\to_a v$.
        \item 模态算子一般写作$\Box_a$.
    \end{itemize}
    \end{frame}
    
    \begin{frame}{*模型验证}
    \begin{itemize}
        \item 我们考虑如下两个\emph{模型验证}(model checking)问题:
        \begin{itemize}
        \item 局部模型验证:测试 $\mathcal M,w \vDash \varphi$ 是否成立;
        \item 全局模型验证:计算集合 $\{w \in W_{\mathcal M} : \mathcal M,w \vDash \varphi\}$.
        \end{itemize}
        \item 设 $l_R(X) = \{w \in W_{\mathcal M} : \forall v : w \to_{\mathcal M} v \implies v \in X\}$,我们可以递归定义$\mathcal M$中公式的\emph{扩张}(extension):
    \[\begin{array}{rlrl}\llbracket \top \rrbracket^{\mathcal{M}} & =W_{\mathcal{M}}, & \llbracket p \rrbracket^{\mathcal{M}} & =\{w : p \in V(w)\}, \\ 
    \llbracket \neg \varphi \rrbracket^{\mathcal{M}} & =W_{\mathcal M}\setminus \llbracket \varphi \rrbracket^{\mathcal{M}}, & \llbracket(\varphi \wedge \psi) \rrbracket^{\mathcal{M}} & =\llbracket \varphi \rrbracket^{\mathcal{M}} \cap \llbracket \psi \rrbracket^{\mathcal{M}}, \\
    \llbracket \square \varphi \rrbracket^{\mathcal{M}} & =l_R\left(\llbracket \varphi \rrbracket^{\mathcal{M}}\right).\end{array}\]
        \item 全局模型验证的一个算法:按照公式的复杂程度,用 $\varphi$的子公式标记$\mathcal M$中每个状态的真值.
        \item 这个问题在实践中并不平凡,因为状态的数量可能是指数多的!
    \end{itemize}
    \end{frame}
    
    \begin{frame}{模态公式的真值} % HW:给个例子算一算
    \begin{itemize}
        \item 模态公式的(语义)真值可以从两个维度来讨论:全局还是局部,模型还是框架.
        \item $\phi$在点模型$\mathcal M,w$\emph{可满足}(satisfiable)指的是$\mathcal M,w\vDash \phi$.
        \item $\phi$在模型$\mathcal M$\emph{有效}(valid),记为$\mathcal M\vDash \phi$指的是$\mathcal M,w\vDash\phi$对所有$w$成立.
        \item $\phi$在点框架$\mathcal F,w$\emph{有效},记为$\mathcal F,w\vDash \phi$指的是$\mathcal M,w\vDash\phi$对所有基于点框架$\mathcal F,w$的点模型$\mathcal M, w$上可满足.
        \item $\phi$在框架$\mathcal F$\emph{有效},记为$\mathcal F\vDash \phi$指的是$\mathcal M\vDash\phi$对所有基于框架$\mathcal F$的模型$\mathcal M$上有效.
        \item $\phi$对框架类$K$\emph{有效},记为$\vDash_K\phi$指的是$\mathcal F\vDash\phi$对所有$\mathcal F\in K$成立.
    \end{itemize}
    \end{frame}
    
    \begin{frame}{模态公式的真值}
    \begin{itemize}
        \item 模态公式的(语义)真值可以从两个维度来讨论:全局还是局部,模型还是框架.
        \begin{table}[ht]
        \centering
        \begin{tabular}{c|cc}
             &模型&框架  \\\hline
             局部&\light{$\mathcal M,w\vDash \phi$} &$\mathcal F,w\vDash \phi$\\
             全局&$\mathcal M\vDash \phi$ &\light{$\mathcal F\vDash \phi$}\\
        \end{tabular}
    \end{table}
        \item 我们主要讨论高亮的两个部分.
    \end{itemize}
    \end{frame}
    
    \begin{frame}{框架语义:例子}
    \begin{figure}
        \centering
        \input{Slides/Figures/10/frame-validity}
    \end{figure}
    \begin{itemize}
        \item 是否成立$\mathcal F\vDash\Box(p\to\Diamond p)$?
    \end{itemize}
    \end{frame}
    
    \begin{frame}{例子:基本模态逻辑}
    \begin{itemize}
        \item 考虑基本模态逻辑,因此他只有模态算子$\Box$和$\Diamond$,分别表示必然和可能.
        \item Kripke模型的点应该被解读为可能世界.
        \item 我们可以将对于可能和必然的理解写成模态公式.
        \item 如果一个东西是真的,那么他也是可能的:$\phi:p\to\Diamond p$.
    \end{itemize}
    \end{frame}
    
    \begin{frame}{例子:基本模态逻辑}
    \begin{itemize}
        \item 如果一个东西是真的,那么他也是可能的:$\phi:p\to\Diamond p$.
        \item 我们可以将对于可能和必然的理解反映到Kripke模型中.
        \item 真实世界是一个可能的世界:$xRx$,这是一个自反关系.
        \item 对于自反点模型以及框架,$\mathcal M,v\vDash \phi$以及$\mathcal F\vDash \phi$.
    \end{itemize}
    \end{frame}
    
    \begin{frame}{例子:时序逻辑}
    \begin{itemize}
        \item 考虑基本的时序逻辑,因此他只有模态算子$G$和$H$,以及对偶$F$和$P$,他们分别对应未来和过去.
        \item Kripke模型的点应该被解读为时刻,时刻之间有两个关系:
        \begin{itemize}
            \item $t_1 R_F t_2$表示时刻$t_2$是时刻$t_1$的未来.
            \item $t_1 R_P t_2$表示时刻$t_2$是时刻$t_1$的过去.
        \end{itemize}
        \item 我们对时间的理解将会反映在Kripke模型上.
    \end{itemize}
    \end{frame}
    
    \begin{frame}{例子:时序逻辑}
    \begin{itemize}
        \item 过去是未来的倒转:对任意时刻$t_1,t_2$,$t_1 R_F t_2\iff t_2 R_P t_1$.
        \item 如果我们承认时间具有这样的性质,那么$R_F$和$R_P$实际上就是箭头倒转一下.
        \item 记$R_F=R$,我们有:
        \[\mathcal M,w\vDash F\phi\iff\exists v(wRv\wedge\mathcal M,v\vDash \phi).\]
        \[\mathcal M,w\vDash P\phi\iff\exists v(vRw\wedge\mathcal M,v\vDash \phi).\]
    \end{itemize}
    \end{frame}
    
    \begin{frame}{例子:时序逻辑}
    \begin{itemize}
        \item 我们再进一步假设时间是线性的,也就是关系$R$是一个严格全序:
        \begin{itemize}
            \item 反自反:$\forall x\neg xRx$.
            \item 传递:$\forall x,y,z(xRy\wedge yRz\to xRz)$.
            \item 完全:$\forall x,y(xRy\vee yRx\vee x=y)$.
        \end{itemize}
        \item 设$\mathcal F$是时间框架,是否有:$\mathcal F\vDash Pp\to GP p$以及$\mathcal F\vDash Fp\to HF p$?
    \end{itemize}
    \end{frame}
    
    \section{模态可定义性}
    \begin{frame}{模态可定义性}
    \begin{itemize}
        \item 逻辑的意义在于把对事物的抽象认知用形式化的语言表述出来.
        \item 我们已经看到,我们对事物的认知可以被两种方式描述出来:
        \begin{itemize}
            \item Kripke模型(框架)的特殊结构.
            \item 具体的模态公式.
        \end{itemize}
        \item 这两种方式之间有什么联系?
    \end{itemize}
    \end{frame}
    \begin{frame}{例子}
    \begin{itemize}
        \item $\mathcal M,w\vDash\Diamond\top$.
        \begin{itemize}
            \item 存在一个$v$,$w\to v$并且$\mathcal M,v\vDash\top$,也就是$w$有一个后继.
        \end{itemize}
        \item $\mathcal F\vDash\Diamond \top$.
        \begin{itemize}
            \item 每个基于$\mathcal F$的点模型$\mathcal M,v$都有$\Diamond \top$,即$\mathcal F$每个点都有后继.
        \end{itemize}
        \item $\mathcal F\vDash p\to\Diamond p$.
        \begin{itemize}
            \item 任意赋值$V$和任意点$w$,都有$\mathcal M,w\vDash p\to\Diamond p$.
            \item 因此,如果$w$上有$p$,那么$w$必须有一个后继上面也有$p$.
            \item  取$V$使得只有$w$上有$p$,因为对任意赋值都要成立,所以在这个赋值下,$w$必须要以自己为后继.
            \item 因此$\mathcal F$充分必要地是一个自反框架.
        \end{itemize}
    \end{itemize}
    \end{frame}
    
    \begin{frame}{模态可定义性}
    \begin{itemize}
        \item 从点模型的角度,我们可以讨论模态公式定义了什么样的点模型.
        \item 设$K$是一些点模型的集合,$\Sigma$是一些模态公式的集合.
        \item 我们说$K$可由公式集$\Sigma$定义,指的是对任意点模型$\mathcal M,w$,$\mathcal M,w\in K$当且仅当任何$\Sigma$中的公式在$\mathcal M,w$都是可满足的.
        \item 如果$\Sigma=\{\phi\}$,我们就说$K$可以由公式$\phi$定义.
        \item 对于框架可定义性,我们有类似的定义.
    \end{itemize}
    \end{frame}
    
    \begin{frame}{模态可定义性}
    \begin{itemize}
        \item 如果定义$K$的公式集$\Sigma$有限,那么$K$也可以由单个公式$\bigwedge_{\phi\in\Sigma}\phi$定义.
        \item 如果$K$由公式$\phi$定义,那么它的补$\overline{K}$就可以由$\neg\phi$定义.
        \item 然而,如果$K$由无穷个公式定义,它的补$\overline{K}$不一定可以用无穷个公式定义!
        \begin{itemize}
            \item 形式上来说,$\overline{K}$可以被$\bigvee_{\phi\in\Sigma}\neg\phi$定义,然而这是一个无穷析取,不一定能等价于某一个集合的公式.
        \end{itemize}
    \end{itemize}
    \end{frame}
    
    \begin{frame}{框架类可定义性:更多例子}
    \begin{itemize}
        \item $\Box p\to\Diamond p$.
        \begin{itemize}
            \item 定义了每一个点都有后继的框架,与$\Diamond\top$定义了一样的框架类.
        \end{itemize}
        \item $\Box p\to\Box\Box p$.
        \begin{itemize}
            \item 定义了传递的框架类.
        \end{itemize}
        \item $\Diamond p\to\Diamond \Diamond p$.
        \begin{itemize}
            \item 定义了稠密的框架类,即如果$x\to y$,那么存在$z$满足$x\to z$且$z\to y$.
        \end{itemize}
        \item 如果将这些模态算子放在时序逻辑中理解,我们实际上已经得到了关于时间的公理!% HW:Lob公式
    \end{itemize}
    \end{frame}
    
    \begin{frame}{*模态可定义与一阶可定义}
    \begin{itemize}
        \item 我们注意到,所有以上的例子,模型的结构都是可以用一阶公式描述的.
        \begin{itemize}
            \item 每个点都有后继:$\forall x\exists y(xRy)$.
            \item 传递:$\forall x,y,z(xRy\wedge yRz\to xRz)$.
            \item 稠密:$\forall x,y(xRy\to\exists z(xRz\wedge zRy))$.
        \end{itemize}
        \item 将一阶公式中的变元$x,y,\dots$看成模型的点,类似模态公式,我们可以讨论一阶公式可定义的模型类/框架类.
        \item 思考:一阶可定义和模态可定义是什么样的关系?
    \end{itemize}
    \end{frame}
    
    \begin{frame}{*模态可定义性:一般结果}
    \begin{itemize}
        \item 更一般地,给定一个点模型类$K$,是否存在模态公式(集)可以定义$K$?
        \item 类似地,给定一个框架类$K$,是否存在模态公式(集)可以定义$K$?
        \item 对于点模型来说,可以被模态公式集定义以及可以被模态公式定义,都有充分必要的刻画定理.
        \item 对于框架来说,如果限制框架是一阶可定义的框架,我们有GoldBlatt-Thomason定理,这是一个充分必要条件.
    \end{itemize}
    \end{frame}
    