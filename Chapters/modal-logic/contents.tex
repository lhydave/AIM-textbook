\chapter{模态逻辑基础}\label{chap:modal-logic}

\begingroup
\newcommand{\pref}{Chapters/modal-logic/figures}

 人工智能的讨论不可避免要接触到很多人独有的哲学概念:认知、信念、知识、理解、情感、意识……我们需要有一套恰当的数学工具来表述这些哲学概念,从而算法化、自动化地模拟人. 过去,现在,乃至未来,最为成功的数学模型就是\textbf{模态逻辑}\index{模态逻辑}. 很多不精确的哲学讨论可以通过逻辑的方式形式化、数学化,最后算法化. 模态逻辑已经在计算机科学中起到了重要的作用(模型验证、形式化方法),它势必会在人工智能中也起到根基性的作用.
    
\section{模态逻辑的起源}
\subsection{三段论}\index{三段论}

早在亚里士多德的时期,模态逻辑的概念就被提了出来. 回忆:亚里士多德的\emph{强三段论}\index{三段论!强~}是有效的:
\begin{itemize}
    \item 大前提:所有$A$都是$B$.
    \item 小前提:所有$B$都是$C$.
    \item 结论:因此,所有$A$都是$C$.
\end{itemize}
人都会死,苏格拉底是人,所以苏格拉底会死.

三段论可以进行各种形式的扩展. 例如:
\begin{itemize}
    \item 加入量词:\light{任何对象},如果这个对象是人,那么它会死,苏格拉底这个对象是人,所以苏格拉底会死.
    \item 加入性质词:肯定的、否定的
    \item 加入\emph{模态词}\index{模态词}:无效、可能、必然、根据情况……
\end{itemize}
    
实际上亚里士多德也考虑过\emph{模态三段论}\index{三段论!模态~}. 亚里士多德认为如下模态三段论是有效的:% HW: 写一个具体的LLL的例子
\begin{itemize}
    \item 大前提:所有$A$都必然是$B$.
    \item 小前提:所有$B$都是$C$.
    \item 结论:因此,所有$A$都必然是$C$.
\end{itemize}
然而,他认为如下的模态三段论不是有效的:
\begin{itemize}
\item 所有$A$都是$B$.
\item 所有$B$都必然是$C$.
\item 因此,所有$A$都必然是$C$.
\end{itemize}
所有通班同学都是单身汉,所有单身汉都必然是男性. 那么是否有:所有通班同学都必然是男性?实际上,通过现代模态逻辑的表述可以证明,XLL型三段论也是有效的. 这足以说明,亚里士多德对于模态的理解还有很多缺陷,或者反过来说,模态三段论比经典的三段论更加复杂.
    
 类似的例子是,从物和从言,考虑如下句子:
\begin{center}
\itshape 我觉得有人作弊.
\end{center}
这句话有两种解读方式. 从物的角度,应该读作$\exists x$(我觉得$:x$作弊). 从言的角度,应该读作我觉得($\exists x:x$作弊).

以上例子说明,关于模态的逻辑,并没有非常干净漂亮的、符合直观的定义,所以模态三段论的讨论并没有流行起来. 
    
\subsection{非经典逻辑}
经典逻辑并不能够非常准确刻画我们对于“逻辑”的认识,我们将给出以下几个例子. 

回忆经典逻辑中的语义等值
        \[p\to q\iff\neg p\vee q.\]
然而,在哲学上,这两者是不一样的含义. 考虑命题$p$:$1+1>2$,$q$:太阳从东边升起. “如果$1+1>2$,那么太阳从东边升起”是毫无道理的. 然而,“或者$1+1\not>2$,或者太阳从东边升起”是含义清晰的.

再看另一个例子,考虑命题$p_n$:$\pi$的小数位包含连续的$n$个1. $p_{100}\to p_{99}$是显然的,然而$\neg p_{100}\vee p_{99}$并不直观!
    
以上的缺陷都源自我们对蕴含的理解过于简单粗暴. 上面的那种蕴含被称为\textbf{实质蕴含}\index{实质蕴含},它最重要的特性是承认$p\rightarrow q \iff
\lnot {p}\vee q$,特别地允许有$p$假但是$q$真.

为了解决上面的问题,C. I. Lewis提出了\textbf{严格蕴含}\index{严格蕴含}的概念,其符号为$p\strictif q$. 从语义上说,这个符号的意思是\light{必然有}当$p$是真的时候,$q$是真的. 也可以说,\light{不可能}有$p$是真且$q$是假. 严格蕴含其实是对实质蕴含的一种改造,我们后面可以证明,$p$严格蕴含$q$当且仅当\light{必然有} $p$实质蕴含$q$.
    
实质蕴含还会导出还有很多不合乎常理的重言式. 比如$p\to(q\to p)$或者$(p\to q)\vee (q\to r)$.

另一种解决此问题的方式是从否定入手,这就是Brouwer的\emph{直觉主义逻辑}\index{逻辑!直觉主义~},它不承认反证法. 因而否定和蕴含的含义发生了变化,例如$\neg\neg p$不再等价于$p$. 
    
从本质上说,非经典逻辑都是尝试将\emph{元语言}中的概念拿到\emph{对象语言}中. 例如经典逻辑的元语言就是自然语言,而对象语言就是经典逻辑形式系统. 在经典逻辑中,必然、可能、过去、未来、知识、信念、可证明等概念都没有办法表示,因此模态逻辑的解决方案是:将这些元语言的概念拿到对象语言中,并进行形式化. \index{元语言}\index{对象语言}
    
\section{模态语言}\index{模态语言}
现在我们正式定义模态语言. 我们只考虑最简单的情况,没有量词,只有一个\emph{模态算子}\index{模态算子}. \emph{基本模态语言}\index{模态语言!基本~} $L$ 可以按照如下定义递归生成:
\begin{itemize}
    \item 命题字母$p\in \mathbf P$属于$L$,$\top$属于$L$.
    \item 如果$\phi$属于$L$,那么$\neg\phi$ 和$\Box\phi$也属于$L$.
    \item 如果$\phi_1,\phi_2$属于$L$,那么$(\phi_1\wedge\phi_2)$也属于$L$.
\end{itemize}
$\Box$:读作“Box”. 定义$L$更便捷的记号是使用Backus-Naur范式:
    \[\phi\quad::=\quad p\mid \top\mid \neg\phi\mid (\phi\wedge\phi)\mid \Box\phi.\] 

类似命题逻辑,我们有如下缩写:
\begin{itemize}
    \item $\phi\vee\psi\iff\neg(\neg \phi\wedge\neg\psi)$.
    \item $\phi\to\psi\iff\neg\phi\vee\psi$.
    \item $\bot\iff\neg\top$.
\end{itemize}
这些缩写意味着,我们对Boole连接词,依然保持经典逻辑的含义. 非经典性只体现在模态算子$\Box$. 对偶模态算子$\Diamond$定义为$\neg\Box\neg$,读作“diamond”. 类比:$\exists=\neg\forall\neg$.

模态逻辑的哲学是\emph{多视角下看同一个数学概念}. 下面是一些模态逻辑的例子,他们对于模态算子的解读都不一样:

\begin{itemize}
    \item \index{逻辑!基本模态~}基本模态逻辑:可能/必然是
    \item \index{逻辑!时序~}时序逻辑:将会是
    \item \index{逻辑!道义~}道义逻辑:被允许是
    \item \index{逻辑!认知~}\light{认知逻辑:被知道是}
    \item \index{逻辑!可证性~}可证性逻辑:可以被证明是
    \item \index{逻辑!动态~}动态逻辑:(在经过某些程序步骤之后)会是
    \item \index{逻辑!联盟~}联盟逻辑:被(她的父母)确保是
    \item \index{逻辑!特征~}\index{逻辑!描述~}特征逻辑和描述逻辑:具有的属性是
\end{itemize}

下面我们看一些不同的逻辑下模态算子的解读:

\begin{example}[基础语义]\index{逻辑!基本模态~}
如果我们把模态算子按照基础语义来解读的话,我们可以把模态算子$\Box$读成“必然”. $\Box\phi$读作“必然有$\phi$”. $\Diamond\phi$读作“不是必然有非$\phi$”,即可能有$\phi$. 因此,$\Diamond$读作“可能”. 反之,$\Box\phi$也可以读作“不可能有非$\phi$”,即必然有$\phi$. 因此$\Diamond$和$\Box$确实是对偶的. 在这样的语义下,我们还有这些例子:
\begin{itemize}
    \item $\Box p\to\Diamond p$:必然的是可能的.
    \item $p\to\Box p$:真的是必然的.
    \item $\Diamond p\to\Box\Diamond p$:可能的是必然可能的.
    %HW: $p\to\Box\Diamond p$ 的解释,以及是否合理.
\end{itemize}
\end{example}

\begin{example}[认知语义]\index{逻辑!认知~}
    如果把模态算子按照认知逻辑来解读解读,我们可以把模态算子$\Box$读成“知道”,并写成$K$. 于是,$K$表示某个特定的个体对世界的认知. 有如下例子:
    \begin{itemize}
        \item $K\phi$(即$\Box\phi$)表示我知道$\phi$. 
        \item $K\phi\to\phi$表示如果我知道$\phi$,那么$\phi$是真的. 
        \item $\phi\to K\phi$表示如果$\phi$是真的,那么我知道$\phi$. 
        \item 最后,$\neg K\phi$与$K(\neg\phi)$的含义是不同的,比如$\phi$表示“上帝存在”,那么前者是“我不知道上帝存在”,而后者是“我知道上帝不存在”,明显后者的判断要更强一些.
\end{itemize}
\end{example}

\newcommand{\PA}{\mathbf{PA}}

\begin{example}[可证性语义]\index{逻辑!可证性~}
如果把模态算子按照可证性逻辑解读,我们可以把模态算子$\Box$读成“可证明”. 例如,$\Box\phi$读作$\phi$是可证明的. 考虑Peano算术系统$\PA$,即一阶逻辑加上Peano公理. 符号$\PA\vdash \phi$表示$\phi$可以由$\PA$演绎出,即$\phi$可以被证明. 算术公理系统一个重要的定理叫\emph{Löb定理}\index{Löb定理}:如果$\PA\vdash\mathsf{Prov}(\ulcorner\phi\urcorner)\to\phi$,那么$\PA\vdash\phi$. 用自然语言来读,如果可以证明“如果$\phi$是可证明的,那么$\phi$是真的”,那么就可以证明$\phi$. 因此,在可证逻辑中,它对应\emph{Löb公式}\index{Löb公式}:$\Box(\Box\phi\to\phi)\to\Box\phi$. 
\end{example}

 一般地,我们可以考虑多个模态算子、一个模态算子涉及多个公式的情形. \emph{模态语言类型}\index{模态语言类型}是一个元组$(O,\rho)$,其中$O$是一个模态算子$\nabla$的非空集合,$\rho:O\to\N$表示每一个模态算子的元数. \emph{多元模态语言}\index{模态语言!多元~}的BNF为:
    \[\phi\quad::=\quad p\mid \top\mid \neg\phi\mid (\phi\wedge\phi)\mid \nabla(\underbrace{\phi,\dots,\phi}_{\rho(\nabla)}),\]
    其中$p\in \mathbf P$,$\nabla\in O$.

类似地,定义对偶模态算子$\triangle(\phi_1,\dots,\phi_k)$为$\neg\nabla(\neg\phi_1,\dots,\neg\phi_k)$.


\begin{example}[时序逻辑]\index{逻辑!时序~}
基础时序逻辑有两个一元模态算子:$G$和$H$. $G\phi$表示未来总会有$\phi$(always $G$oing to be). $H\phi$表示过去总有$\phi$(always $H$as been).

他们的对偶算子是:$F$和$P$. $F\phi$表示在未来某个时刻会有$\phi$(be true at some $F$uture time). $P\phi$表示在过去某个时刻有$\phi$(was true at some $P$ast time).

还可以加入一个“直到”($U$ntil)算子$U(\phi,\psi)$,它表示直到$\phi$发生都有$\psi$.

下面是一些时序逻辑的例子:
\begin{itemize}
    \item $P\phi\to GP\phi$:如果过去发生过$\phi$,那么$\phi$在未来总会发生过.
    \item $F\phi\to FF\phi$:如果未来某个时刻会有$\phi$,那么在未来的某个时刻会发生:未来的某个时刻会有$\phi$.
    \item \emph{McKinsey公式}\index{McKinsey公式}$GF p\to FG p$:如果原子的信息总会在某个未来时刻为真,那么他会在未来某个时刻之后变得总为真.
\end{itemize}
\end{example}


\begin{example}[认知逻辑]\index{逻辑!认知~}
基本模态算子为$K_a$和$B_a$,$K_a$表示个体$a$知道,$B_a$表示个体$a$相信. 例如$K_aK_b\phi\leftrightarrow K_bK_a\phi$表示我知道你知道$\phi$当且仅当你知道我知道$\phi$.

此外,我们也可以加入共同知识算子$C$,$C \phi$当且仅当$K_a(\phi\wedge C\phi)$对任意$a$成立.

我们也可以加入二元的相对算子. 如相对共同知识$C^r(\phi,\psi)$,表示当所有人都知道$\psi$时,所有人具有共同知识$\phi$. 以及条件信念$B_a(\phi,\psi)$,表示当$\psi$为真时,个体$a$相信$\phi$.
\end{example}

\begin{example}[命题动态逻辑]\index{逻辑!命题动态~}\index{PDL}
命题动态逻(PDL)辑有无穷多个模态算子,记为$[\pi]$,其中$\pi$按照程序来理解. $[\pi]\phi$解释为:从当前状态开始运行程序$\pi$,任何一种终止状态,$\phi$都成立. 它的对偶算子记为$\langle\pi\rangle$,$\langle\pi\rangle\phi$解释为:从当前状态开始运行程序$\pi$,存在一种终止状态,$\phi$成立. 

PDL的重要区别在于我们可以用模态算子来构造新的模态算子. 假设基本程序为$a,b,\dots$,我们可以用三种操作构造新的程序(模态算子):
\begin{itemize}
    \item 选择:如果$\pi_1,\pi_2$是程序,那么$\pi_1\cup\pi_2$也是程序,它(非确定性地)执行$\pi_1$或$\pi_2$. 模态算子为$[\pi_1\cup\pi_2]$和$\langle\pi_1\cup\pi_2\rangle$.
    \item 复合:如果$\pi_1,\pi_2$是程序,那么$\pi_1;\pi_2$也是程序,它先执行$\pi_1$再执行$\pi_2$. 模态算子为$[\pi_1;\pi_2]$和$\langle\pi_1;\pi_2\rangle$.
    \item 迭代:如果$\pi$是程序,那么$\pi^*$也是程序,它执行$\pi$有限次(可能是零次). 模态算子为$[\pi^*]$和$\langle\pi^*\rangle$.
\end{itemize}
以上构造得到的PDL被称为\textbf{正则PDL}\index{PDL!正则~}. 我们还可以引入交$\pi_1\cap\pi_2$,表示并行计算;也可以引入条件程序$\phi?$,其中$\phi$是公式. 

下面是一些命题动态逻辑的例子:
\begin{itemize}
    \item $\left\langle\pi^*\right\rangle \phi \leftrightarrow \phi \vee\left\langle\pi ; \pi^*\right\rangle \phi$. 在执行$\pi$有限次数后到达一个带有信息$\phi$的状态当且仅当要么我们已经在当前状态中拥有信息$\phi$,要么我们可以执行一次$\pi$,然后在有限次数的$\pi$迭代后找到一个带有信息$\phi$的状态. 
    \item Segerberg公理(归纳公理):$[\pi^*](\phi \to [\pi]\phi) \to (\phi \to [\pi^*]\phi)$. 这个公式的含义是什么?
    \item 用模态算子表示\texttt{if $\phi$ then $a$ else $b$}:$(\phi?;a)\cup(\neg\phi?;b)$. 
\end{itemize}
\end{example}

\section{Kripke语义与框架语义}

从模型论的角度来说,一个逻辑框架是一个三元组:(语言,模型,语义) $(\mathbf L,C,\vDash)$. 例如在命题逻辑中,这三元组分别是:
\begin{itemize}
    \item 语言:命题公式的集合$\{\top,\bot,p,p\vee q,\dots\}$.
    \item 模型:常值和命题字母的真假:$\top:\top$,$p:\top$,$q:\bot$,等等.
    \item 语义:Boole函数的真值表递归定义.
\end{itemize}

对于基本模态逻辑,我们有如下要素:
\begin{itemize}
    \item 语言:基本模态语言$\mathbf{ML}(\mathbf P, \Box)$.
    \item 模型:Kripke模型.
    \item 语义:Kripke语义.
\end{itemize}

一个\emph{Kripke模型}(关系模型)\index{Kripke模型}\index{关系模型}可以看作是一个带有标记的有向边和节点的图,节点表示可能的世界,状态或对象等,用命题字母标记;边表示节点之间的关系,用模态算子标记. 一个\emph{框架}\index{框架}是一个没有节点标记的模型.

在数学上,Kripke模型是唯一的定义,然而在哲学上,我们可以对这一数学结构做不同的解读. 

在\emph{可能世界语义}(或\emph{Kripke语义})\index{语义!可能世界~}\index{语义!Kripke~}中,我们将节点解读为\emph{可能世界}. 此时$\Box$被理解为“必然”,$\Diamond$被理解为“可能”. $\Box\phi$在当前世界为真当且仅当$\phi$在当前世界的所有可能的替代世界上为真. 用形式化的表述就是,$\Box\phi$在世界$w$上成立当且仅当$\phi$在$w$的所有后继上为真. 在这种语义中,一个世界的意义取决于它与其他世界的联系.

在\emph{状态语义}\index{语义!状态~}中,我们将节点解读为\emph{状态}. 于是边就被解读为状态的转移. $\Box\phi$在当前状态为真当且仅当$\phi$在所有可能转移到的状态上为真. PDL 可以用以上语义来理解. 在哲学中,状态往往是不完全可观测的,此时模态逻辑可以被理解为不完全信息中可以确定的性质.

在\emph{对象语义}\index{语义!对象~}中,我们将节点解读为\emph{对象}. $w$有边指向$v$意味着$w$是$v$包含的一个整体,$v$是$w$的一个部分. 在哲学上,模态逻辑可以讨论整体论与还原论. $\Box\phi$对一个对象为真当且仅当$\phi$在它的所有部分都为真.

在基本情形下,Kripke模型的正式定义如下:

\begin{definition}[Kripke框架,模型,点模型,二元情形]\index{Kripke框架}\index{Kripke点模型}\index{Kripke模型}
    一个\textbf{Kripke框架}是元组$\mathcal F=(W,R)$,其中
    \begin{itemize}
        \item $W$是非空集合(可能世界集);
        \item $R\subseteq W\times W$是一个$W$上的二元关系(边).
    \end{itemize}

    一个\textbf{Kripke模型}$\mathcal{M}$是一个元组$(\mathcal F,V)$,其中$\mathcal F$是Kripke框架,$V:\mathbf P\to 2^W$是赋值函数.

    一个\textbf{Kripke点模型}$(\mathcal{M},w)$是一个Kripke模型$\mathcal M$加上一个指定的点$w\in W$.
\end{definition}

一个典型的Kripke模型如\Cref{fig:Kripke-model-basic}所示.

\begin{figure}[ht]
    \centering
    \begin{tikzpicture}[node distance=2cm]
\node (w1) [circle,draw] {$w_1$};
\node (w2) [circle,draw,right=of w1] {$w_2$};
\node (w3) [circle,draw,below=of w1] {$w_3$};
\node (w4) [circle,draw,right=of w3] {$w_4$};
\draw[-Latex] (w1) -- (w2);
\draw[-Latex] (w1) -- (w3);
\draw[-Latex] (w2) -- (w4);
\draw[-Latex] (w3) -- (w4);
\node [above left=1pt of w1] {$p$};
\node [above right=1pt of w2] {$p,q$};
\node [below left=1pt of w3] {};
\node [below right=1pt of w4] {$q$};
\end{tikzpicture}
    \caption{Kripke模型的例子}
    \label{fig:Kripke-model-basic}
\end{figure}

对于多元情形,我们有如下定义:

\begin{definition}[Kripke框架,模型,点模型:一般情形]\index{Kripke框架}\index{Kripke点模型}\index{Kripke模型}
    考虑$\mathbf{ML}(\mathbf{P},(O,\rho))$,一个\textbf{Kripke框架}指的是元组$\mathcal F=(W,\{R_\nabla:\nabla\in O\})$,其中$W$是非空集合(可能世界集),$R_\nabla$是一个$W$上的$\rho(\nabla)+1$元关系.

    一个\textbf{Kripke模型}$\mathcal{M}$指的是元组$(\mathcal F,V)$,其中$\mathcal F$是框架,$V:\mathbf P\to 2^W$是\emph{赋值函数}.

    一个\textbf{Kripke点模型}$(\mathcal{M},w)$是Kripke模型$\mathcal M$加上一个指定的点$w\in W$.
\end{definition}


下面我们再给出Kripke语义的形式化定义.

\begin{definition}[Kripke语义:基本情形]\index{Kripke语义}
    考虑$\mathbf{ML}(\mathbf{P},\Box)$. 符号$\mathcal M,w\vDash\phi$表示$\phi$在点模型$\mathcal M,w$是\emph{可满足的}\index{可满足}. 这一个概念可以递归定义如下
    \begin{itemize}
        \item $\mathcal M, w\vDash\top\iff$ 总是.
        \item $\mathcal M, w\vDash p\iff p\in V(w)$.
        \item $\mathcal M, w\vDash (\phi\wedge\psi)\iff\mathcal M,w\vDash\phi$且$\mathcal M,w\vDash\phi$.
        \item $\mathcal M, w\vDash \neg\phi\iff\mathcal M,w\not\vDash\phi$.
        \item $\mathcal M, w\vDash \Box\phi\iff$对所有$v$,如果$wRv$,那么$\mathcal M,v\vDash\phi$.
    \end{itemize}
    因此,$\mathcal M, w\vDash \Diamond\phi\iff$存在$v$满足$wRv$使得$\mathcal M,v\vDash\phi$.
\end{definition}

思考:在\Cref{fig:Kripke-model-basic}中,有对哪些$i$来说,成立$\mathcal M,w_i\vDash\Box (p\to\Diamond q)$?


\begin{definition}[Kripke语义:一般情形]\index{Kripke语义}
    考虑$\mathbf{ML}(\mathbf{P},(O,\rho))$. 符号$\mathcal M,w\vDash\phi$表示$\phi$在点模型$\mathcal M,w$是\emph{可满足的}\index{可满足}. 这一个概念可以递归定义如下
    \begin{itemize}
        \item $\mathcal M, w\vDash\top\iff$ 总是.
        \item $\mathcal M, w\vDash p\iff p\in V(w)$.
        \item $\mathcal M, w\vDash (\phi\wedge\psi)\iff\mathcal M,w\vDash\phi$且$\mathcal M,w\vDash\psi$.
        \item $\mathcal M, w\vDash \neg\phi\iff\mathcal M,w\not\vDash\phi$.
        \item $\mathcal M, w\vDash \nabla(\phi_1,\dots,\phi_{\rho(\nabla)})\iff$对\light{任意} $w_1,w_2,\dots w_{\rho(\nabla)}$,如果$R(w,w_1,\dots,w_{\rho(\nabla)})$,那么\light{存在} $w_i$使得$\mathcal M,w_i\vDash\phi_1$.
    \end{itemize}
\end{definition}

思考:为什么要这么定义$\nabla$的语义?

如果一个模态算子对应的关系是二元关系,我们就称这个模态算子是\emph{一元的}. 此时,关系$wRv$可以记为$w\to_a v$. 模态算子一般写作$\Box_a$.


\begin{remark}[模型验证]\index{模型验证}
我们考虑如下两个\textbf{模型验证}问题:局部模型验证,即测试 $\mathcal M,w \vDash \varphi$ 是否成立;全局模型验证,即计算集合 $\{w \in W_{\mathcal M} : \mathcal M,w \vDash \varphi\}$.

设 $l_R(X) = \{w \in W_{\mathcal M} : \forall v : w \to_{\mathcal M} v \implies v \in X\}$,我们可以递归定义$\mathcal M$中公式的\emph{扩张}:
\[\begin{array}{rlrl}\llbracket \top \rrbracket^{\mathcal{M}} & =W_{\mathcal{M}}, & \llbracket p \rrbracket^{\mathcal{M}} & =\{w : p \in V(w)\}, \\ 
\llbracket \neg \varphi \rrbracket^{\mathcal{M}} & =W_{\mathcal M}\setminus \llbracket \varphi \rrbracket^{\mathcal{M}}, & \llbracket(\varphi \wedge \psi) \rrbracket^{\mathcal{M}} & =\llbracket \varphi \rrbracket^{\mathcal{M}} \cap \llbracket \psi \rrbracket^{\mathcal{M}}, \\
\llbracket \square \varphi \rrbracket^{\mathcal{M}} & =l_R\left(\llbracket \varphi \rrbracket^{\mathcal{M}}\right).\end{array}\]

全局模型验证的有一个很容易想到的算法:按照公式的复杂程度,用 $\varphi$的子公式标记$\mathcal M$中每个状态的真值. 然而,这个问题在实践中并不平凡,因为状态的数量可能是指数多的!
\end{remark}

模态公式的(语义)真值可以从两个维度来讨论:全局还是局部,模型还是框架. 我们有如下四种语义真值的定义:

\begin{itemize}
    \item $\phi$在点模型$\mathcal M,w$\emph{可满足}\index{可满足}指的是$\mathcal M,w\vDash \phi$.
    \item $\phi$在模型$\mathcal M$\emph{有效}\index{有效},记为$\mathcal M\vDash \phi$指的是$\mathcal M,w\vDash\phi$对所有$w$成立.
    \item $\phi$在点框架$\mathcal F,w$\emph{有效}\index{有效},记为$\mathcal F,w\vDash \phi$指的是$\mathcal M,w\vDash\phi$对所有基于点框架$\mathcal F,w$的点模型$\mathcal M, w$上可满足.
    \item $\phi$在框架$\mathcal F$\emph{有效}\index{有效},记为$\mathcal F\vDash \phi$指的是$\mathcal M\vDash\phi$对所有基于框架$\mathcal F$的模型$\mathcal M$上有效.
    \item $\phi$对框架类$K$\emph{有效}\index{有效},记为$\vDash_K\phi$指的是$\mathcal F\vDash\phi$对所有$\mathcal F\in K$成立.
\end{itemize}

模态逻辑的真值总结为\Cref{tab:modal-logic-validity},我们主要关心高亮的两个部分.

\begin{table}[ht]
    \centering
    \begin{tabular}{c|cc}
            &模型&框架  \\\hline
            局部&\light{$\mathcal M,w\vDash \phi$} &$\mathcal F,w\vDash \phi$\\
            全局&$\mathcal M\vDash \phi$ &\light{$\mathcal F\vDash \phi$}\\
    \end{tabular}
    \caption{模态逻辑的真值}
    \label{tab:modal-logic-validity}
\end{table}


我们来看一个框架语义的例子. 如\Cref{fig:modal-logic-frame} 所示,是否成立$\mathcal F\vDash\Box(p\to\Diamond p)$?
\begin{figure}[ht]
    \centering
    \input{\pref/frame-validity}
    \caption{框架语义的例子}
    \label{fig:modal-logic-frame}
\end{figure}

\begin{example}[基本模态逻辑]\index{逻辑!基本模态~}
考虑基本模态逻辑,因此他只有模态算子$\Box$和$\Diamond$,分别表示必然和可能. Kripke模型的点应该被解读为可能世界. 我们可以将对于可能和必然的理解写成模态公式. 如果一个东西是真的,那么他也是可能的:$\phi:p\to\Diamond p$.

如果一个东西是真的,那么他也是可能的:$\phi:p\to\Diamond p$. 我们可以将对于可能和必然的理解反映到Kripke模型中. 真实世界是一个可能的世界:$xRx$,这是一个自反关系. 对于自反点模型以及框架,$\mathcal M,v\vDash \phi$以及$\mathcal F\vDash \phi$.
\end{example}


\begin{example}[时序逻辑]\index{逻辑!时序~}
考虑基本的时序逻辑,因此他只有模态算子$G$和$H$,以及对偶$F$和$P$,他们分别对应未来和过去. Kripke模型的点应该被解读为时刻,时刻之间有两个关系:$t_1 R_F t_2$表示时刻$t_2$是时刻$t_1$的未来,$t_1 R_P t_2$表示时刻$t_2$是时刻$t_1$的过去. 我们对时间的理解将会反映在Kripke模型上.

关于时间的的一个自然的假设是过去是未来的倒转:对任意时刻$t_1,t_2$,$t_1 R_F t_2\iff t_2 R_P t_1$. 如果我们承认时间具有这样的性质,那么$R_F$和$R_P$实际上就是箭头倒转一下. 记$R_F=R$,我们有:
    \[\mathcal M,w\vDash F\phi\iff\exists v(wRv\wedge\mathcal M,v\vDash \phi).\]
    \[\mathcal M,w\vDash P\phi\iff\exists v(vRw\wedge\mathcal M,v\vDash \phi).\]

进一步假设时间是线性的,也就是关系$R$是一个严格全序:
\begin{itemize}
    \item 反自反:$\forall x\neg xRx$.
    \item 传递:$\forall x,y,z(xRy\wedge yRz\to xRz)$.
    \item 完全:$\forall x,y(xRy\vee yRx\vee x=y)$.
\end{itemize}
设$\mathcal F$是时间框架,是否有:$\mathcal F\vDash Pp\to GP p$以及$\mathcal F\vDash Fp\to HF p$?
\end{example}


\section{模态可定义性}

逻辑的意义在于把对事物的抽象认知用形式化的语言表述出来. 我们已经看到,我们对事物的认知可以被两种方式描述出来: Kripke模型(框架)的特殊结构,或者具体的模态公式. 本节探讨这两种方式之间的联系. 

我们先看几个简单的例子:
\begin{example}
\begin{itemize}
    \item $\mathcal M,w\vDash\Diamond\top$.

    存在一个$v$,$w\to v$并且$\mathcal M,v\vDash\top$,也就是$w$有一个后继.

    \item $\mathcal F\vDash\Diamond \top$.
    
    每个基于$\mathcal F$的点模型$\mathcal M,v$都有$\Diamond \top$,即$\mathcal F$每个点都有后继.

    \item $\mathcal F\vDash p\to\Diamond p$.
    
    任意赋值$V$和任意点$w$,都有$\mathcal M,w\vDash p\to\Diamond p$. 因此,如果$w$上有$p$,那么$w$必须有一个后继上面也有$p$. 取$V$使得只有$w$上有$p$,因为对任意赋值都要成立,所以在这个赋值下,$w$必须要以自己为后继. 因此$\mathcal F$充分必要地是一个自反框架.
    \end{itemize}
\end{example}

以上例子启发我们,模态公式可以定义Kripke模型的特殊结构,特殊的Kripke模型划定了具体的模态公式,这种联系可以被形式化为\textbf{模态可定义性}\index{模态可定义性}.

从点模型的角度,我们可以讨论模态公式定义了什么样的点模型. 

\begin{definition}[点模型可定义性]\index{模态可定义性!点模型可定义性}
设$K$是一些点模型的集合,$\Sigma$是一些模态公式的集合. 我们说\textbf{$K$可由公式集$\Sigma$定义},指的是对任意点模型$\mathcal M,w$,$\mathcal M,w\in K$当且仅当任何$\Sigma$中的公式在$\mathcal M,w$都是可满足的. 如果$\Sigma={\phi}$,我们就说\textbf{$K$可以由公式$\phi$定义}. 
\end{definition}
对于框架可定义性,我们有类似的定义.

模态可定义性:如果定义$K$的公式集$\Sigma$有限,那么$K$也可以由单个公式$\bigwedge_{\phi\in\Sigma}\phi$定义. 如果$K$由公式$\phi$定义,那么它的补$\overline{K}$就可以由$\neg\phi$定义. 然而,如果$K$由无穷个公式定义,它的补$\overline{K}$不一定可以用无穷个公式定义. 这是因为形式上来说,$\overline{K}$可以被$\bigvee_{\phi\in\Sigma}\neg\phi$定义,然而这是一个无穷析取,不一定能等价于某一个集合的公式.

我们来看框架类可定义性的更多例子:
\begin{example}
\begin{itemize}
    \item $\Box p\to\Diamond p$.
 
    定义了每一个点都有后继的框架,与$\Diamond\top$定义了一样的框架类.
    
    \item $\Box p\to\Box\Box p$.
    
    定义了传递的框架类.
    
    \item $\Diamond p\to\Diamond \Diamond p$.
    
    定义了稠密的框架类,即如果$x\to y$,那么存在$z$满足$x\to z$且$z\to y$.
\end{itemize}
注意,如果将这些模态算子放在时序逻辑中理解,我们实际上已经得到了关于时间的公理!% HW:Lob公式
\end{example}

\begin{remark}
我们注意到,所有以上的例子,模型的结构都是可以用一阶公式描述的.
\begin{itemize}
    \item 每个点都有后继:$\forall x\exists y(xRy)$.
    \item 传递:$\forall x,y,z(xRy\wedge yRz\to xRz)$.
    \item 稠密:$\forall x,y(xRy\to\exists z(xRz\wedge zRy))$.
\end{itemize}
 将一阶公式中的变元$x,y,\dots$看成模型的点,类似模态公式,我们可以讨论一阶公式可定义的模型类/框架类. 
 
 更一般地,我们可以问,给定一个点模型类$K$,是否存在模态公式(集)可以定义$K$?类似地,给定一个框架类$K$,是否存在模态公式(集)可以定义$K$?对于点模型来说,可以被模态公式集定义以及可以被模态公式定义,都有充分必要的刻画定理. 对于框架来说,如果限制框架是一阶可定义的框架,我们有GoldBlatt-Thomason定理,这是一个充分必要条件.
\end{remark}

\endgroup