\chapter{共同知识,Bayes博弈,Aumann知识算子}\label{chap:bayesian-game}

真实世界是一个巨大的游戏,参与者之间的信息不对称是一个普遍存在的现象,这些不对称往往形成了“丛林法则”。“丛林法则”从来不是一个书面的规则。刚来美国旅游的人可能会非常担心自己被抢劫;然而,“盗亦有道”,绝大部分时候,抢劫犯只会抢二十美元,以便吃一顿饭。如果坏了规矩,反而会被惩罚。

我们可以考虑一个看似非常疯狂的想法:既然有丛林法则,何不把“一次抢劫最多二十美元”写入法律,这会有什么区别呢?抛开道德和法律的问题,这样做似乎不无道理。

1982年,经济学家Alvin E. Roth和J. Keith Murnighan对讨价还价这一经典的市场博弈现象进行了同样的实验。实验中,有两个玩家,他们可能会收到特定价值的奖品,每个玩家的奖品价值可以不同。然而玩家并不总是能获得奖品,他有一个概率获得这一奖品。他们需要在规定的时间内就他们各自获得奖品的概率进行讨价还价。

具体来说,玩家其实在商讨如何分配一张100\%概率的“彩票券”,这个券决定了每个玩家赢得奖品的概率。例如,甲如果获得40\%的彩票券,就有40\%的概率赢得奖品,60\%的概率一无所获;而乙则完全相反,有60\%的概率赢得奖品,40\%的概率一无所获。

然而,如果在规定时间内未能达成协议,那么所有玩家都将一无所获。因此,只有在双方就彩票券的分配达成了协议,并且该玩家在随后的抽奖中中奖时,玩家才能获得相应的奖品。否则,他将得不到任何奖励。我们将这种每个玩家只有两种可能金钱收益的游戏称为“二元彩票游戏”。

在实际的实验中,玩家甲的奖品价值20美元,玩家乙的奖品价值为5美元,谈判时间限制为12分钟。在实验开始前,实验人员会公开告知甲乙完整的游戏规则,包括奖品价值和谈判时间限制。实验人员还会公开告知甲乙关于他们私有信息的情况,例如,
\begin{quotation}
    “在游戏开始后,你们奖品的价值会被告诉对方。”

    “在游戏开始后,甲的奖品价值会被告诉乙,但乙的不会被告诉甲。”
\end{quotation}
实验人员也可以选择不告知这些信息。

实验开始后,实验人员会告知甲乙他们各自的奖品价值,并且选择性地告知甲(或乙)乙(或甲)的奖品价值,这些信息要和实验开始前的信息保持一致。然后,甲乙开始商讨如何分配彩票券。如果在规定时间内达成了协议,那么他们将按照协议分配彩票券。如果未能达成协议,那么他们将一无所获。

上面的实验设置中,在游戏开始前是否公开宣告双方的信息结构,和上面那个疯狂的想法是一样的。然而,在讨价还价的情景下,我们似乎不会觉得这有什么区别。然而,实验结果却显示,在信息不对称的情况下,双方信息结构的公开宣告会对博弈结果产生显著的影响。

谈判过程展现出了明显的策略性行为。例如,甲(20美元的玩家)通常不会提及自己奖品的价值;但如果游戏开始前宣布过“乙(5美元的玩家)不清楚甲的奖品价值”,20美元的玩家往往会虚报自己的奖品。一个典型的例子是:“我知道你的奖品是5美元。我的只有2美元。所以我应该得到超过50\%的份额。”另一方面,当乙知道甲的奖品价值时,往往会透露信息。这两种策略通常都不被对方相信。

更重要的是,实验结果显示,当实验人员在游戏开始前宣布双方的信息结构时,玩家变得更加没有策略性,博弈的结果很可能没有达到Nash均衡:当只有20美元的玩家知道两个奖品时,其总体平均收益(包括达成协议和未达成协议的情况)为34.9,显著低于5美元玩家的相应收益53.6。

另一方面,当双方的信息结构在游戏开始前不被公开宣告时,玩家表现出更多的策略性。在这种情况下,因为他们不能确定对手是否不知道自己的奖品,所以玩家无法像前一种情况那样自由地谎报自己的奖品价值。然而,因为玩家都不知道对方是否指导双方奖品的价值,所以,如果一个玩家知道两个奖品的价值,他完全可以假装自己只知道自己的奖品价值。由于更复杂的策略性行为,博弈的结果往往达到Nash均衡!

上面的故事告诉我们,信息结构(也就是我们\emph{知道}什么)是否被公开宣告,对于博弈的结果有着重要的影响。实际上,这一问题是一个关于\emph{共同知识}的问题。

在本章,我们将更系统地讨论共同知识,并定义\emph{Bayes博弈},它是我们研究博弈论中信息结构的基本语言。最后,我们将介绍\emph{Aumann知识算子},它是从Bayes博弈中起源的,一个关于“知识”的数学模型。


\section{“泥泞的孩童”谜题}

我们先从一个经典的谜题开始。有$n$个孩子在玩泥巴,他们互相泼泥巴. 母亲告诉孩子们,如果他们脸上沾上了泥巴,会受到严厉的惩罚. 孩子们不能看到自己的脸,但是可以看到其他所有人的脸. 所有孩子都希望保持自己的脸干净,但是弄脏别人的脸. 

此时,孩子的父亲出现了,于是,孩子们停止泼泥巴. 孩子们互相不说话. 父亲看到了$k$($k\geq 1$)个人脸上有泥巴,于是宣布:“\emph{你们至少有一个人脸上沾了泥巴.}” 之后,父亲会公开地问若干轮如下问题: “\emph{你们知道自己脸上有泥巴了吗?}” 孩子们回答“知道”或者“不知道”. 

假设孩子们观察力敏锐、聪慧且诚实,并且每一轮他们都同时回答. 接下来会发生什么?乍一看,似乎大家每次都会回答“不知道”,因为他们不能从这句话知道自己脸上有泥巴. 但是,这个谜题远比想象的要复杂。读者可以先不看后文,先自己试一试。

假设有$k$个孩子脸上有泥巴. 这个问题的谜底是:在前$k-1$轮中,所有孩子都会说“不知道”,在第$k$轮中,所有脸上有泥巴的孩子都会说“知道”. 

这一结论的论证来源于对较小$k$的归纳总结:
\begin{itemize}
    \item 当$k=1$时,脸上沾满泥巴的孩子看到其他人都没有泥巴. 既然他知道至少有一个孩子的脸上有泥巴,他就能推出那个人肯定是他自己. 
    \item 现在假设$k=2$,脸上沾满泥巴的孩子是$a$和$b$. 一开始,因为他们分别看到了对方的脸上有泥巴,所以他们每个人都回答“不知道”. 
    
    但是,当$b$回答“不知道”时,$a$可以代入$b$的角色,意识到$b$看到了$a$脸上有泥巴. 否则,$b$在第一轮中就会知道泥巴在$b$的脸上,并回答“知道”. 因此,$a$在第二轮回答“知道”. $b$也会通过同样的推理得出相同的结论. 

    \item 现在假设$k=3$,脸上沾满泥巴的孩子分别是$a$,$b$和$c$. 孩子$a$的论证如下. 假设我没有泥巴落在脸上. 根据$k=2$的情况,$b$和$c$在第二轮都会回答“是”. 他们没有这样做,我意识到假设是错误的,我的脸上也有泥巴. 因此在第三轮我会回答“知道”. $b$和$c$的论证也是类似的.
\end{itemize}
容易看出,$k=3$的论证具有一般性,对一般的$k$也成立.

\begin{remark}
“泥泞的孩童”还有其他流行的陈述方式,比如“蓝眼睛红眼睛”. 一个岛上有100个人,其中有5个红眼睛,95个蓝眼睛. 这个岛有三个奇怪的宗教规则.
    \begin{enumerate}
        \item 他们不能照镜子,不能看自己眼睛的颜色. 
        \item 他们不能告诉别人对方的眼睛是什么颜色. 
        \item 一旦有人知道了自己的眼睛是红色,他就必须在当天夜里自杀.
    \end{enumerate}
岛民不知道具体有几个红眼睛. 

某天,有个旅行者到了这个岛上. 由于不知道这里的规矩,所以他在和全岛人一起狂欢的时候,一不留神说了一句话:“\emph{你们这里有红眼睛的人. }”假设这个岛上的人足够聪明,每个人都可以做出缜密的逻辑推理. 请问这个岛上将会发生什么?
\end{remark}

谜题就解到这里。然而,一个好的谜题,知道答案之后一定会带来更多的谜题。为什么迷题的答案是这样的呢?比如,如果$k>1$,那么所有人本来就都知道$p$:“至少有一个人脸上有泥巴”. 那么父亲说这句话的意义是什么?

我们可以设想,如果父亲没有说$p$,会发生什么?容易发现,无论父亲问多少轮,所有孩子都只会回答“不知道”!(见习题\lhysays{出一下})。因此,这里会产生极其反直觉的事实:即便大家都知道$p$,父亲说不说$p$,会导致完全不同的结果. 为什么会这样?

我们重新审视这一问题的过程。
\begin{itemize}
    \item 
    假设$k=2$,脸上沾满泥巴的孩子是$a$和$b$. 在父亲宣布$p$之前,$a$和$b$都知道$p$. 然而,他们并不知道对方知道$p$. $a$可能会有两种想法:
        \begin{itemize}
            \item 我的脸上有泥巴,所以$b$知道$p$.
            \item 我的脸上没有泥巴,$b$是唯一一个有泥巴的,$b$看不到其他人脸上有泥巴,所以$b$不知道$p$.
        \end{itemize}
    当父亲宣布$p$之后,\emph{$a$知道了$b$知道$p$}. 当第一轮$b$回答“不知道”之后,$a$可以用“$b$知道$p$”这一事实排除第二种情况,从而推出自己脸上有泥巴.    
    \item 假设$k=3$,脸上沾满泥巴的孩子是$a$,$b$和$c$. 在父亲宣布$p$之前,$a$,$b$和$c$不仅知道$p$,而且知道彼此知道$p$. 比如说,以$a$的视角看,$b$能看到$c$脸上有泥巴,所以$a$知道$b$知道$p$. 
    
    但是,$a$不知道$b$知道$c$知道$p$,因为此时有两种情况:
    \begin{itemize}
        \item $a$脸上有泥巴,$b$能看到$c$和$a$脸上有泥巴,所以$b$知道$c$能看到$a$脸上有泥巴,从而知道$c$知道$p$.
        \item $a$脸上没有泥巴,$b$能看到$c$脸上有泥巴,$a$脸上没有泥巴,但因为$b$不知道自己脸上有没有泥巴,所以$c$不一定知道$p$,$b$不知道$c$知道$p$.
    \end{itemize}
    更一般地,\emph{$a$,$b$,$c$都不知道所有人知道所有人知道$p$}!然而,当父亲宣布$p$之后,$a$,$b$,$c$都\emph{知道了所有人知道所有人知道$p$}.
\end{itemize}

用$E^m p$表示所有人知道所有人知道……所有人知道($m$次)$p$. 在一般情况下,父亲没有宣布$p$之前,$E^k p$并不成立. 父亲宣布了$p$之后,对任意$m\geq 1$,$E^m p$都成立!因此,父亲宣布$p$带来了\emph{共同知识}. 有了共同知识,这一谜题就可以按照我们所讨论的方式进行下去.

我们曾经假设过所有人“观察力敏锐、聪慧且诚实”. 然而,这一假设并不足够. 上面的论证其实暗含了,所有人都知道所有人“观察力敏锐、聪慧且诚实”,所有人都知道所有人都知道所有人“观察力敏锐、聪慧且诚实”,……换言之,我们需要假设“所有人观察力敏锐、聪慧且诚实”是共同知识. 

如果没有这样的假设,上面的论证都将不成立。例如,还是只有两个孩子$a,b$脸上有泥巴. 假如$a$不知道$b$是诚实的,即便$b$回答了“不知道”,$a$也无法从$b$的回答中得到任何额外的知识!

除了假设“所有人观察力敏锐、聪慧且诚实”是共同知识,我们还需要假设以下陈述是共同知识:
    \begin{itemize}
        \item 每个人都能看到所有除自己外的人.
        \item 每个人都听到了父亲说的话.
        \item 父亲是诚实的.
        \item 每个人都在每一轮进行了充分的推理.
        \item ……
    \end{itemize}
任何假设的破坏都会导致之前的讨论失效. 那么,为什么父亲宣布$p$就可以让$p$变成共同知识呢?

所有人都\emph{听到}父亲说$p$并不能产生共同知识. 假如父亲只是对每一个孩子单独宣布$p$. 所有人并不知道所有人都知道$p$,因而仅仅可以做到$E p$,没有共同知识。

那么,所有人都\emph{知道}所有人听到父亲说$p$会如何呢?进一步假设每个孩子给每一个孩子都安装了窃听器,每个人都能够偷听每个人与父亲的谈话内容. 所有人并不知道所有人都知道所有人都知道$p$,因而仅仅有$E^2 p$. 

所以关键在于,父亲宣布$p$的过程是\emph{公开的},每个人都可以仔细观察别人有没有听到父亲说$p$,也可以观察到别人有没有观察到别人有没有听到父亲说$p$,等等. 此时对每一个$m$都有$E^m p$.

“泥泞的孩童”谜题足以表明,关于“知道”的讨论远比想象的复杂. 关于“知道”和知识的研究在哲学中划归为\emph{知识论}. 我们将介绍处理知识的两种数学模型:
\begin{itemize}
    \item 一种源自Aumann,Harsanyi和Rubinstein等人,以Bayes概率论为基础,是偏经济学的学术风格;
    \item 另一种源自Kripke,Hintikka和Halpern等人,以模态逻辑为基础,是偏计算机科学和哲学的学术风格。
\end{itemize}
在这一章,我们主要讨论Bayes概率论的方法.

\section{不完全信息博弈(Bayes博弈)}

接下来,我们介绍讨论“知识”的博弈论语言。我们先从正则形式博弈和Nash均衡开始说起。正则形式博弈隐含了一个重要的假设:所有玩家对整个世界有一致、完全的共同知识. Nash均衡建立在这一假设之上:每个玩家可以在博弈结束后,根据其他玩家的策略,确定自己的最优反应. 

然而,现实世界中,玩家对世界的认识是有限的,不能获得完全的信息. 比如,我们可能不知道对手的收益函数,然而这在现实中极其普遍. 

另一个问题是,Nash均衡只有假设\emph{混合策略}的情况下才能保证存在. 我们在现实中并不真的在选择混合策略:所有的交易其实都是“一锤子买卖”,绝对不可能有人说“我今天以0.5的概率花一块钱买你的苹果,0.5的概率花5块钱买你的苹果”. 英语也有一句谚语:
\begin{quotation}
“Decision makers do not flip coins in the real world.”
\end{quotation}

相比之下,纯策略Nash均衡更加符合实际。然而,即便是纯策略Nash均衡也可能是不合理的状态. 考虑如下的二人博弈:
\[\begin{pmatrix}
1,1&0,0\\
0,0&0,0
\end{pmatrix}.\]
显然,两个人玩家都选择第二策略就达到了纯策略Nash均衡.

然而,当行玩家对列玩家的选择有任意小的不确定性时,他都更倾向于选择第一个策略. 因此,我们给出的这个纯策略Nash均衡实际上描述了一种不太可能出现的状态.

因此,一种Nash均衡的修正概念被提出:\emph{颤抖的手完美化},它指的是,$s$是一个纯策略Nash均衡,并且当对手玩家的策略有任何微小不确定性的时候,$s$中的策略依然是最优反应.

“颤抖的手”给了我们一个例子,说明对对手的不确定性会影响玩家的决策. 因此,进一步的问题是,如何量化对对手的不确定性?Harsanyi给出了一个现在已经是“标准答案”的解决方案:引入玩家的“类型”(可能世界)以及其他玩家的对此的先验的\emph{信念}. 他的采用了Bayes解释的概率论,信念在数学上被建模为对可能世界的概率分布.

我们先给出不完全信息博弈的定义:

\begin{definition}[不完全信息博弈,Bayes博弈]
    一个\textbf{不完全信息博弈}(\textbf{Bayes博弈})包含了以下组成部分:
\begin{itemize}
    \item 玩家集合:$I$.
    \item 行动空间:$A=(A_i)_{i\in I}$,$A_i$表示玩家$A_i$的所有可能行动.
    \item 类型空间:$\Theta=(\Theta_i)_{i\in I}$,$\Theta_i$表示玩家$i$的所有可能类型.
    \item 收益函数:$u_i:A\times\Theta\to\R$,当所有人的行动和类型都确定的时候,玩家$i$能拿到的收益.
\end{itemize}
所有玩家的行动$a=(a_i)_{i\in I}$形成了一个\textbf{行动组合},所有玩家的类型$\theta=(\theta_i)_{i\in I}$形成了一个\textbf{类型组合}.
\end{definition}

$P_i\in\Delta(\Theta_i)$是玩家$i$类型的概率分布. 比较不直观的一点是,$P_i$表示了\emph{其他玩家}对玩家$i$类型的信念. 因此,Bayes博弈其实做了简化:
\begin{itemize}
    \item 不论哪个玩家,对特定玩家$i$的信念是一致的.
    \item 所有$P_i$是相互独立的,因此玩家的类型之间不会有相互的关联。
\end{itemize}
因此,玩家$i$在博弈中的\emph{全部}不确定性都来自其他玩家的类型,他对此的信念是
\[P_{-i}=\prod_{j\neq i}P_j.\]

最后,我们假设玩家$i$知道自己的类型.

更进一步,在一般情况下,玩家$i$对这个世界的信念应该包含:
\begin{itemize}
    \item 博弈中其他玩家都有谁,
    \item 自己的可能行动,
    \item 自己的类型,
    \item 自己的收益函数,
    \item ……
\end{itemize}
在一个真实的博弈中,以上信息都会是不确定的:对手可以藏在暗处,我们会不知道自己本来拥有的选择,我们可以不了解自己的性格,我们甚至也不知道自己究竟在追求什么。尽管有很多的不确定性最终都可以归结为“类型”(见后面正文的若干例子和习题\lhysays{出一下}),Bayes博弈依然是一个高度理想化的模型. 

然而,Bayes博弈的成功,正如Robert Aumann所说,不在于多么贴合实践,而是给了一种系统的方法,让我们可以建模不确定性和信念,从而理解这些概念在博弈中的作用:
\begin{quotation}
    “一个典型的例子是 Roth 和 Murnighan(1982)在完全信息和不完全信息讨价还价方面的实验工作\footnote{也就是本章开头讲的故事。}……他们将这些结果与早期 Fouraker 和 Siegel(1960)的实验进行了比较。

    “Fouraker 和 Siegel 进行了类似的实验,但由于缺乏 Harsanyi 的模型,只能将不完全信息的情况描述为双方都没有被告知对方的收益。
    
    “然而,Roth 和 Murnighan 则从类型的角度详细阐述了不完全信息,并明确考虑了博弈的共同知识方面。”
\end{quotation}

接下来,我们看一个Bayes博弈的例子。

\begin{example}[“工作还是偷懒”博弈]
    在这个博弈中,有两个玩家,他们共同完成一个项目。两个玩家的行动都是“工作”($W$)或“偷懒”($S$). 行玩家的类型集合是单点集,列玩家的类型是“勤奋”($D$)或“懒惰”($L$).收益矩阵为
    \[\begin{array}{c|cc}
        \multicolumn{3}{l}{\theta_2=D,}\\
         &W&S  \\\hline
         W&3,3&-1,0\\
         S&2,1&0,0\\
    \end{array}\qquad \begin{array}{c|cc}
        \multicolumn{3}{l}{\theta_2=L,}\\
         &W&S  \\\hline
         W&1,1&-1,2\\
         S&2,-1&0,0\\
    \end{array}\]
    换言之,我们其实不确定的只有玩家$2$是喜欢偷懒还是努力工作,但他具体是什么人又会影响双方的收益,从而影响双方的决策。
\end{example}

至此,我们已经建立了博弈的语言,下一任务就是定义一个玩家的\emph{策略}。和\Cref{chap:game}的博弈极为不同,Bayes博弈中的玩家要面对不确定性。这给我们定义策略带来了一定的困难。为此,我们先要讨论清楚,究竟什么是“不确定性”。

设想如下的情景:你选了一门课,期末要考试。于是,你认真学习,并且把往年题c都找出来,认真做完,老师也划定了考纲和难度:你对这个考试胸有成竹。真正上场考试的时候,你的心理有一个对难度和考点的预期,因此,你在考试时候面对的是具有明确\emph{风险}的不确定性。

然而,当你考的不是期末考,而是英语四级考试的时候,你可能就会变得非常随性:挂了还可以重新考,所以考前根本没有学习,甚至连题型都不知道有什么。这个时候,你甚至连四级总分是多少都不知道,所以,你甚至连考试结果的预期都没有。这个时候,你面对的是\emph{模糊}的不确定性。

还有一种情景,你现在不是考四级,而是考GRE,这是上机考试。GRE的每道题的难度都不一样,题目是随机出现的,难度分布极其不均匀,并且他会根据你答题情况来自动调整题目的难度。此时,你面对的是\emph{不稳定}的不确定性。

上面讲到了三种不确定性。那么,Bayes博弈是他们中的是哪一种呢?显然,在Bayes博弈中,玩家依然只能做一次行动,所以不是不稳定的不确定性。此外,Bayes博弈中,所有玩家的类型集是固定的,甚至连类型的似然也是确定的,因此,这是具有明确风险的不确定性。

接下来,我们可以定义玩家的策略和理性了。注意,玩家知道自己的类型,但不知道其他人的类型,所以,玩家只能根据自己的类型来决定自己的行为。因此,我们应该定义玩家的策略如下:

\begin{definition}[策略]
    玩家$i$的\textbf{策略}是一个映射
    \[s_i:\Theta_i\to A_i,\]
    其中$\Theta_i$是玩家$i$的类型集合,$A_i$是玩家$i$的行动集合. $s_i(\theta_i)$表示玩家$i$在类型$\theta_i$下的行动.
\end{definition}

\begin{remark}
    自然,我们也可以定义\emph{混合策略},此时,$s_i$是一个$\Theta_i$到$\Delta(A_i)$的映射. 不过,在Bayes博弈中,混合策略会让情况(不论概念上还是计算上)变得复杂,所以我们避免混合策略的讨论.
\end{remark}

当面对具有明确风险(似然)不确定性的时候,我们可以遵循von Neumann-Morgenstern的期望效用理论,定义玩家的理性。首先,我们定义玩家的\emph{事中期望收益}:

\begin{definition}[事中期望收益]
    玩家$i$在类型$\theta_i$下,采取策略$s_i$,对手的策略是$s_{-i}$时,$i$的\textbf{事中期望收益}为
    \[\tilde u_i(s_i,\theta_i,s_{-i})=\E_{\theta_{-i}\sim P_{-i}}[u_i(s_i(\theta_i),s_{-i}(\theta_{-i}),\theta_i,\theta_{-i})].\]
\end{definition}

于是,按照期望理论,玩家的\emph{事中理性}就是在知道对手的策略之后,会最大化自己的事中期望收益。

据此,我们就可以定义最优反应:

\begin{definition}[事中最优反应]
    考虑玩家$i$的策略$s_i$,对手的策略是$s_{-i}$,如果对任意的其他策略$s'_i$和任意类型$\theta_i$,都有t
    \[\tilde u_i(s_i,\theta_i,s_{-i})\geq \tilde u_i(s'_i,\theta_i,s_{-i}),\]
    那么$s_i$是玩家$i$的\textbf{事中最优反应}.
\end{definition}

用最优反应,我们可以很容易定义Bayes Nash均衡:

\begin{definition}[Bayes Nash均衡,BNE]
    考虑策略组合$s=(s_i)_{i\in I}$,如果对任意$i,\theta_i,a_i$都有
    \[\tilde u_i(s(\theta_i),\theta_i,s_{-i})\geq \tilde u_i(a_i,\theta_i,s_{-i}),\]
    那么$s$是一个\textbf{Bayes Nash均衡}(\textbf{BNE}).
\end{definition}

接下里,我们来看一个BNE的例子。

\begin{example}[猜硬币游戏]
    考虑经典的猜硬币游戏(见\Cref{ex:matching-pennies}),这是一个正则形式博弈,收益矩阵为
    \[
    \begin{array}{c|cc}
         &H&T  \\\hline
         H&1,-1 &-1,1\\
         T&-1,1&1,-1\\
    \end{array}
    \]
    如果两个人都出$H$的时候收益有独立微小的扰动,我们就得到了一个Bayes博弈:
    \[
    \begin{array}{c|cc}
         &H&T  \\\hline
         H&1+\epsilon\theta_1,-1+\epsilon\theta_2 &-1,1\\
         T&-1,1&1,-1\\
    \end{array}
    \]
    其中$\theta_i\sim\mathcal U[-1,1]$且相互独立,这就是玩家$i$的类型。

    那么,这个博弈的BNE会是什么呢?我们可以猜一个。考虑策略:$s_i:[-1,1]\to\{H,T\}$满足
    
    \[s_i(\theta_i)=\begin{cases}
    H,&\theta_i\in[0,1],\\
    T,&\theta_i\in[-1,0).
    \end{cases}\]
    我们来验证,这的确是一个BNE.

    固定列玩家的策略$s_2$,我们来计算行玩家的最优反应。对于行玩家来说,不论他的类型是什么,他面对的是一个$50\%$的概率选择$H$和$T$的对手,因此,如果选择$H$,行玩家的期望收益是
    \[\frac{1}{2}\cdot (1+\epsilon\theta_1)+\frac{1}{2}\cdot(-1)=\frac{\epsilon\theta_1}{2}. \]
    如果选择$T$,行玩家的期望收益是
    \[\frac{1}{2}\cdot(-1)+\frac{1}{2}\cdot(1)=0. \]
    因此,如果$\theta_1>0$,行玩家应该选择$H$,如果$\theta_1<0$,行玩家应该选择$T$,这是一个最优反应,恰好是我们猜测的策略.

    对于列玩家的最优反应,我们可以做类似的计算,也可以得到,$s_2$是列玩家的最优反应. 因此,$(s_1,s_2)$是一个BNE.
\end{example}

以上例子有极其特殊的含义:策略$(s_1,s_2)$导致的结果实际上是,每个玩家计算最优反应的时候,面对的对手其实仿佛是一个混合策略玩家,他以等概率选择$H$和$T$. 注意,当$\epsilon\to 0$,这个博弈收益矩阵回到了原始博弈. 因此,Bayes博弈里行动概率分布其实可以被视作原始博弈的混合策略.

猜硬币游戏的例子其实说明,正则形式博弈的混合策略Nash均衡被理解为:当不确定性趋于消失时候,BNE形成的行动概率分布. 这不是偶然的,实际上,所有的正则形式博弈的混合策略均衡都可以用一系列Bayes博弈的BNE\emph{纯化}.

下面我们把猜硬币游戏的过程一般化。考虑一个正则形式博弈$(I,A,u)$,其中$I$是玩家集合,$A$是行动空间,$u$是收益函数. 我们可以定义一个Bayes博弈,被称为\emph{扰动博弈}:
\begin{itemize}
\item 给定一个扰动参数$\epsilon>0$,定义类型组合为$\theta=(\theta_i)_{i\in I}$,$\theta_i\in[-1,1]$,然后,将收益扰动为:
    \[u_i'(s,\theta)=u_i(s)+\epsilon\theta_i.\]
    \item 假设$\theta_i\sim F_i$,相互独立,$F_i$是具有连续可微密度的分布.
\end{itemize}

定义一个从Bayes博弈策略到正则形式博弈混合策略的映射$\phi$,给定一个Bayes博弈的策略组合$s$,$\phi(s)$是一个混合策略,满足
\[\phi(s)_i(a_i)=\Pr_{\theta_i\sim F_i}[s_i(\theta_i)=a_i].\]
换言之,在原本的Bayes博弈中,概率定义在了玩家的类型上,这个映射的作用是将这个概率转化为了玩家的行动上.

接下来,我们可以正式给出纯化定理的表述:

\begin{theorem}[Harsanyi纯化定理]
给定玩家集$I$和行动空间$A$. 对于一般的收益函数$u$和连续分布族$\{F_i\}_{i\in I}$,对任意完全信息正则形式博弈$(I,A,u)$的混合策略Nash均衡$\sigma$,存在一列扰动博弈纯策略BNE $s_\epsilon$,当扰动参数$\epsilon\to 0$,$\phi(s_\epsilon)\to \sigma$.
\end{theorem}
这一定理的证明比较长,并且需要用到较为复杂的数学技巧。由于证明本身与本章的讨论无关,所以这里略去.

这一定理给了Nash均衡(也即混合策略)一种新的解释:混合策略定义的Nash均衡可以被看作不确定性趋于消失的时候的(纯策略)BNE。

尽管我们说,“Decision makers do not flip coins in the real world.”然而,如果玩家对自己的收益有微小的不确定性,他的行为就会仿佛在抛硬币. 这就是混合策略的\emph{似然解释}.

注意,在Bayes博弈中,玩家对自己的类型是确定的,所以玩家在决策时不应该对自己的收益有微小的不确定,因而上面这一解释并不完全正确。然而,我们可以重新定义理性的概念,它会产生等价的BNE定义,但是玩家此时要面对自己类型的不确定性。

为了说明这一理性的概念,我们先看一个例子。假如你是一个顶尖斗地主玩家,知道自己拿到身份、拿到手牌之后要如何行动,这是你一贯的策略。当然,在游戏开始前,你其实并不知道自己的身份和手牌,这是你对自己的不确定性。不过,这并不影响你评估自己的胜率:你只需要知道其他玩家的水平,你就可以大概估计一个总体的胜率。

上面的例子里,玩家其实在博弈开始前就已经选好了一个类型到行动的策略,但是他要面临自己类型的不确定性。此时,我们计算的胜率其实是\emph{事前}期望收益:

\begin{definition}[事前期望收益]
    给定玩家集$I$、行动空间$A$和类型上的联合分布$P$,如果玩家$i$的策略是$s_i$,对手的策略是$s_{-i}$,那么$i$的\textbf{事前期望收益}为
\[\hat u_i(s_i,s_{-i})=\E_{\theta\sim P}[u_i(s_i(\theta_i),s_{-i}(\theta_{-i}),\theta_i,\theta_{-i})].\]
\end{definition}

在Bayes博弈中,\emph{事前理性}指的就是,玩家在不知道自己的类型的情况下,最大化自己的事前期望收益. 根据事前期望收益,我们可以定义事前最优反应:

\begin{definition}[事前最优反应]
    考虑玩家$i$的策略$s_i$,对手的策略是$s_{-i}$,如果对任意的其他策略$s'_i$和任意类型$\theta_i$,都有
    \[\hat u_i(s_i,s_{-i})\geq \hat u_i(s'_i,s_{-i}),\]
    那么$s_i$是玩家$i$的\textbf{事前最优反应}.
\end{definition}

注意,事前理性和事中理性考虑的都是同样的策略,但是玩家面对的不确定性是不同的。然而,事前理性和事中理性其实是等价的:

\begin{theorem}
    给定玩家集$I,$行动空间$A$,类型空间$\Theta$、收益函数$u$和类型空间的联合分布$P$,考虑策略组合$s=(s_i)_{i\in I}$,对任意玩家$i$,$s_i$是$s_{-i}$的事前最优反应当且仅当$s_i$是$s_{-i}$的事中最优反应.
\end{theorem}
这一定理的证明类似正则形式博弈的无差别原理(\Cref{thm:indifference-principle})的证明,见习题\lhysays{出一下}.

这一定理其实有一些反直觉:在Bayes博弈中,如果玩家面对的仅仅只是具有确定风险的不确定性,那么他是否不确定自己的类型并不会影响他的理性决策. 这其实是因为,玩家面对的仅仅是\emph{具有确定风险的不确定性},而不是更加复杂的不确定性,因此,无论博弈如何进行,他在开始前就已经可以确定自己的最优策略.

这一定理的直接推论是,我们可以根据事前最优反应来定义BNE:
     \[\hat u_i(s_i,s_{-i})\geq u_i(s'_i,s_{-i})\]
对任意$i$和任意策略$s_i'$成立.

当所有的不确定性都消失的时候,我们得到的收益是真实的,被称为\emph{事后收益},此时不再有任何的概率,因此博弈退化为了正则形式博弈. 事前、事中、事后分别表明了信息的确定(披露)程度。

\section{电子邮件博弈}\label{sec:emailing-game}

作为一个例子,接下来我们使用Bayes博弈来研究知识的性质。这个例子由Ariel Rubinstein给出,它说明在二人正则形式博弈中,共同知识对到底实现哪一个Nash均衡非常关键.

考虑两个玩家和两个可能的收益矩阵:
\begin{table}[ht]
    \centering
\begin{tabular}{c|cc}
&$A$ & $B$ \\
\hline
$A$ & $(0, 0)$ & $(-10, 1)$ \\
$B$ & $(1, -10)$ & $(8, 8)$ \\
\end{tabular}
\qquad
\begin{tabular}{c|cc}
&$A$ & $B$ \\
\hline
$A$ & $(8, 8)$ & $(-10, 1)$ \\
$B$ & $(1, -10)$ & $(0, 0)$ \\
\end{tabular}
\end{table}

在左边的矩阵中,$(B,B)$是唯一的Nash均衡. 在右边的矩阵中有多个Nash均衡:$(A,A)$和$(B,B)$. $(A,A)$给出比$(B,B)$更高的收益,但行动$A$比$B$更有风险.

左边矩阵是真实矩阵概率是$p>1/2$. 玩家$1$知道真实的矩阵,而玩家$2$不知道. 如果选择了右边矩阵,玩家$1$会给玩家$2$发送一条消息. 如果玩家$2$收到了消息,她会回复. 如果玩家$1$收到了回复,她会发送第二条消息来确认她收到了玩家$2$的回复. 以此类推.  每条消息都以$\epsilon$的概率独立等可能丢失.\footnote{注意:发送电子邮件不是一个行动,而是一个规则.}  

以上传信的过程可以用Bayes博弈的类型来刻画. 具体来说,两个玩家的类型集合为$\Theta_i = \{\theta_i^0, \theta_i^1, \theta_i^2, \dots\}$. $\theta_i^m$表示玩家$i$发了$m$封邮件. $\theta_i^m$有直观的含义. 例如,类型$\theta_1^0$表示真实收益矩阵是左边的.  而类型$\theta_1^1$表示真实收益矩阵是右边的,$1$发送了一封电子邮件,但$2$没有收到. 

实际上,$\theta$包含了所有可能的情况:
\begin{itemize}
\item $(\theta_1^0, \theta_2^0)$:真实收益矩阵是左边的. 
\item $(\theta_1^1, \theta_2^0)$:真实收益矩阵是右边的,$1$发送了一封电子邮件,但$2$没有收到. 
\item $(\theta_1^1, \theta_2^1)$:真实收益矩阵是右边的,$2$收到了第一封电子邮件,但$1$没有收到$2$的回复. 
\item ……
\end{itemize}

我们可以算出来,当真实矩阵为左边矩阵时,每个类型出现的概率. 首先,以概率$p$选择左边的矩阵,而且没有人发送消息. 因此,$(\theta_1^0,\theta_2^0)$的概率是$p$,其他项概率都是$0$. 我们得到如下的概率表:
\[\begin{array}{c|cccc}
\text{左}& \theta_2^0 & \theta_2^1 & \theta_2^2 & \cdots \\
\hline
\theta_1^0 & p & 0 & 0 & \cdots \\
\theta_1^1 & 0 & 0 & 0 & \cdots \\
\theta_1^2 & 0 & 0 & 0 & \cdots \\
\vdots & \vdots & \vdots & \vdots & \ddots
\end{array}
\]

同样可以算出来,当真实矩阵为右边矩阵时,每个类型出现的概率. 首先,以概率$1 - p$选择右边的矩阵,玩家$1$发送一条消息,它会以概率$\epsilon$丢失. 因此,$(\theta_1^1,\theta_2^0)$的概率是$\epsilon(1 - p)$. 以此类推,可以得到计算. 我们得到如下的概率表:
\[
\begin{array}{c|cccc}
\text{右} & \theta_2^0 & \theta_2^1 & \theta_2^2 & \cdots \\
\hline
\theta_1^0 & 0 & 0 & 0 & \cdots \\
\theta_1^1 & \epsilon(1 - p) & \epsilon(1 - \epsilon)(1 - p) & 0 & \cdots \\
\theta_1^2 & 0 & \epsilon(1 - \epsilon)^2(1 - p) & \epsilon(1 - \epsilon)^3(1 - p) & \cdots \\
\theta_1^3 & 0 & 0 & \epsilon(1 - \epsilon)^4(1 - p) & \cdots \\
\vdots & \vdots & \vdots & \vdots & \ddots
\end{array}
\]


容易看出来,当真实类型为$\theta_i^m$时,收益矩阵是到第$m$层的共同知识,即$E^m$. 所以对于很大的$m$,收益矩阵是“几乎公共知识”. 所以,这个模型在研究的问题是:如果收益矩阵是”几乎共同知识“,那么Nash均衡是什么?为此,我们需要求出来这个Bayes博弈的BNE.

我们需要弄清楚对每个类型$\theta_i^m$,玩家会做什么. 假设玩家$1$的类型为$\theta_1^0$. 玩家$1$知道$(\theta_1^0,\theta_2^0)$是真实的类型,所以左边的矩阵被选择. 据此推理:玩家$1$选择占优策略$B$.

假设玩家$2$的类型为$\theta_2^0$. 我们对玩家$1$的所有可能类型分类讨论:
\begin{itemize}
    \item 如果玩家$1$的类型为$\theta_1^0$,那么左边的矩阵被选择,对于玩家$2$来说,这种情况的概率为
    \[\Pr(\theta_1^0|\theta_2^0) = \frac{p}{p+\epsilon(1-p)} := \mu_2^0.\] 
    \item 如果玩家$1$的类型为$\theta_1^1$,那么右边的矩阵被选择,对于玩家$2$来说,这种情况的概率为
    \[\Pr(\theta_1^1|\theta_2^0) = 1 - \mu_2^0.\]
\end{itemize}

现在我们来分析玩家$2$的两种选择:$A$和$B$.
\begin{itemize}
    \item 选择$B$的期望收益至少是$8\mu_2^0$. 推理如下:
    \begin{itemize}
        \item 玩家$1$的类型是$\theta_1^0$时,这是左边的矩阵,玩家$1$肯定选择$B$,此时玩家$2$选择$B$的收益是$8$.
        \item 玩家$1$的类型是$\theta_1^1$时,这是右边的矩阵,无论玩家$1$怎么选,此时玩家$2$选择$B$的收益至少是$0$.
    \end{itemize}
    因此,按照全概率公式计算,$B$的期望收益至少是$8\mu_2^0$.
    \item 选择$A$的期望收益至多是$-10\mu_2^0 + 8(1 - \mu_2^0)$. 推理如下:
    \begin{itemize}
        \item 玩家$1$的类型是$\theta_1^0$时,这是左边的矩阵,玩家$1$肯定选择$B$,此时玩家$2$选择$A$的收益是$-10$.
        \item 玩家$1$的类型是$\theta_1^1$时,这是右边的矩阵,无论玩家$1$怎么选,此时玩家$2$选择$A$的收益至多是$8$.
    \end{itemize}
    因此,按照全概率公式计算,$A$的期望收益至多是$-10\mu_2^0 + 8(1 - \mu_2^0)$.
\end{itemize}

注意,
\begin{align*}
    &8\mu_2^0-(-10\mu_2^0 + 8(1 - \mu_2^0))\\
    =& 10\mu_2^0 - 8\\
    =& \frac{10p-8(p+\epsilon(1-p))}{p+\epsilon(1-p)}\\
    =& \frac{(2+\epsilon)p-8\epsilon}{p+\epsilon(1-p)}\\
    >& \frac{1-8\epsilon}{p+\epsilon(1-p)}.
\end{align*}
对充分小的$\epsilon>0$,这个值是正的. 因此,$B$更好.

假设玩家$1$的类型为$\theta_1^1$,于是,右边的矩阵被选择. 同样,对玩家$2$的类型分类:
\begin{itemize}
    \item 如果玩家$2$的类型为$\theta_2^0$,对于玩家$1$来说,这种情况的概率为
    \[\Pr(\theta_2^0|\theta_1^1) = \frac{\epsilon(1-p)}{\epsilon(1-p)+\epsilon(1-\epsilon)(1-p)} := \mu_2^1.\]
    \item 如果玩家$2$的类型为$\theta_2^1$,对于玩家$1$来说,这种情况的概率为
    \[\Pr(\theta_2^1|\theta_1^1) = 1 - \mu_2^1.\]
\end{itemize}
同样可以计算玩家$1$的两种选择的期望收益:
\begin{itemize}
    \item 选择$B$的期望收益至少为$0$. 推理如下:玩家$2$是类型$\theta_2^0$时肯定选择$B$(上面已经证明),因此最坏的情况是玩家一类型为$\theta_2^1$. 
    \item 选择$A$的期望收益至多为$-10\mu_2^1 + 8(1 - \mu_2^1)$. 推理如下:玩家$2$是类型$\theta_2^0$时肯定选择$B$(上面已经证明),因此最好的情况是玩家一类型为$\theta_1^2$.
\end{itemize}

综合两方面,$B$更好,因为对于所有$\epsilon$,$\mu_1^1 > 1/2$. 

逐步迭代上述过程,我们发现,在唯一的BNE中,所有玩家在所有类型下都选择$B$. 

然而,如果右边的矩阵是共同知识,$(A,A)$也是一个真正的Nash均衡,然而,因为对于收益矩阵的不确定性,即便收益矩阵是“几乎共同知识”,这个均衡也不会被实现!这个例子说明了,共同知识对于均衡的实现是非常关键的. 我们在习题\lhysays{出一下}中会进一步讨论Nash均衡与共同知识的关系.

\section{Aumann知识算子}

\section{习题}

    {关于均衡的进一步思考}
\begin{itemize}
    \item 用$Nash(x)$表示“$x$是Nash均衡”,那么$\exists x C(Nash(x))$和$C(\exists x C(Nash(x)))$的含义是否一样?
    \item 如果不引入不确定性,在完全信息下,实现特定的Nash均衡是否还需要共同知识?
    \item 如果玩家不是逻辑全知的,或者说她的推理、计算能力是有限的,那么Nash均衡还是否会达到?是否可接近?
\end{itemize}

\section{章末注记}