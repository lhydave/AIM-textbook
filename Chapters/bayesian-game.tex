\chapter{共同知识,Bayes博弈,Aumann知识算子}\label{chap:bayesian-game}

真实世界是一个巨大的游戏,参与者之间的信息不对称是一个普遍存在的现象,这些不对称往往形成了“丛林法则”。“丛林法则”从来不是一个书面的规则。刚来美国旅游的人可能会非常担心自己被抢劫;然而,“盗亦有道”,绝大部分时候,抢劫犯只会抢二十美元,以便吃一顿饭。如果坏了规矩,反而会被惩罚。

我们可以考虑一个看似非常疯狂的想法:既然有丛林法则,何不把“一次抢劫最多二十美元”写入法律,这会有什么区别呢?抛开道德和法律的问题,这样做似乎不无道理。

1982年,经济学家Alvin E. Roth和J. Keith Murnighan对讨价还价这一经典的市场博弈现象进行了同样的实验。实验中,有两个玩家,他们可能会收到特定价值的奖品,每个玩家的奖品价值可以不同。然而玩家并不总是能获得奖品,他有一个概率获得这一奖品。他们需要在规定的时间内就他们各自获得奖品的概率进行讨价还价。

具体来说,玩家其实在商讨如何分配一张100\%概率的“彩票券”,这个券决定了每个玩家赢得奖品的概率。例如,甲如果获得40\%的彩票券,就有40\%的概率赢得奖品,60\%的概率一无所获;而乙则完全相反,有60\%的概率赢得奖品,40\%的概率一无所获。

然而,如果在规定时间内未能达成协议,那么所有玩家都将一无所获。因此,只有在双方就彩票券的分配达成了协议,并且该玩家在随后的抽奖中中奖时,玩家才能获得相应的奖品。否则,他将得不到任何奖励。我们将这种每个玩家只有两种可能金钱收益的游戏称为“二元彩票游戏”。

在实际的实验中,玩家甲的奖品价值20美元,玩家乙的奖品价值为5美元,谈判时间限制为12分钟。在实验开始前,实验人员会公开告知甲乙完整的游戏规则,包括奖品价值和谈判时间限制。实验人员还会公开告知甲乙关于他们私有信息的情况,例如,
\begin{quotation}
    “在游戏开始后,你们奖品的价值会被告诉对方。”

    “在游戏开始后,甲的奖品价值会被告诉乙,但乙的不会被告诉甲。”
\end{quotation}
实验人员也可以选择不告知这些信息。

实验开始后,实验人员会告知甲乙他们各自的奖品价值,并且选择性地告知甲(或乙)乙(或甲)的奖品价值,这些信息要和实验开始前的信息保持一致。然后,甲乙开始商讨如何分配彩票券。如果在规定时间内达成了协议,那么他们将按照协议分配彩票券。如果未能达成协议,那么他们将一无所获。

上面的实验设置中,在游戏开始前是否公开宣告双方的信息结构,和上面那个疯狂的想法是一样的。然而,在讨价还价的情景下,我们似乎不会觉得这有什么区别。然而,实验结果却显示,在信息不对称的情况下,双方信息结构的公开宣告会对博弈结果产生显著的影响。

谈判过程展现出了明显的策略性行为。例如,甲(20美元的玩家)通常不会提及自己奖品的价值;但如果游戏开始前宣布过“乙(5美元的玩家)不清楚甲的奖品价值”,20美元的玩家往往会虚报自己的奖品。一个典型的例子是:“我知道你的奖品是5美元。我的只有2美元。所以我应该得到超过50\%的份额。”另一方面,当乙知道甲的奖品价值时,往往会透露信息。这两种策略通常都不被对方相信。

更重要的是,实验结果显示,当实验人员在游戏开始前宣布双方的信息结构时,玩家变得更加没有策略性,博弈的结果很可能没有达到Nash均衡:当只有20美元的玩家知道两个奖品时,其总体平均收益(包括达成协议和未达成协议的情况)为34.9,显著低于5美元玩家的相应收益53.6。

另一方面,当双方的信息结构在游戏开始前不被公开宣告时,玩家表现出更多的策略性。在这种情况下,因为他们不能确定对手是否不知道自己的奖品,所以玩家无法像前一种情况那样自由地谎报自己的奖品价值。然而,因为玩家都不知道对方是否指导双方奖品的价值,所以,如果一个玩家知道两个奖品的价值,他完全可以假装自己只知道自己的奖品价值。由于更复杂的策略性行为,博弈的结果往往达到Nash均衡!

上面的故事告诉我们,信息结构(也就是我们\emph{知道}什么)是否被公开宣告,对于博弈的结果有着重要的影响。实际上,这一问题是一个关于\emph{共同知识}\index{共同知识}的问题。

在本章,我们将更系统地讨论共同知识,并定义\emph{Bayes博弈}\index{Bayes博弈},它是我们研究博弈论中信息结构的基本语言。最后,我们将介绍\emph{Aumann知识算子}\index{Aumann知识算子},它是从Bayes博弈中起源的,一个关于“知识”的数学模型。


\section{“泥泞的孩童”谜题}

我们先从一个经典的谜题开始。

有$n$个孩子在玩泥巴,他们互相泼泥巴. 母亲告诉孩子们,如果他们脸上沾上了泥巴,会受到严厉的惩罚. 孩子们不能看到自己的脸,但是可以看到其他所有人的脸. 所有孩子都希望保持自己的脸干净,但是弄脏别人的脸. 此时,孩子的父亲出现了,于是,孩子们停止泼泥巴. 孩子们互相不说话. 父亲看到了$k$($k\geq 1$)个人脸上有泥巴,于是宣布:“\emph{你们至少有一个人脸上沾了泥巴.}” 之后,父亲会公开地问若干轮如下问题: “\emph{你们知道自己脸上有泥巴了吗?}” 孩子们回答“知道”或者“不知道”. 假设孩子们观察力敏锐、聪慧且诚实,并且每一轮他们都同时回答. 接下来会发生什么?

假设有$k$个孩子脸上有泥巴. 谜底:在前$k-1$轮中,所有孩子都会说“不知道”,在第$k$轮中,所有脸上有泥巴的孩子都会说“知道”. 这一结论的论证来源于对$k$的归纳.

当$k=1$时,脸上沾满泥巴的孩子看到其他人都没有泥巴. 既然他知道至少有一个孩子的脸上有泥巴,他就能推出那个人肯定是他自己. 

现在假设$k=2$,脸上沾满泥巴的孩子是$a$和$b$. 一开始,因为他们分别看到了对方的脸上有泥巴,所以他们每个人都回答“不知道”. 但是,当$b$回答“不知道”时,$a$意识到他自己肯定是脸上有泥巴的那个孩子,否则$b$就会在第一轮中知道泥巴在他的脸上,并回答“知道”. 因此,$a$在第二轮回答“知道”. $b$也会通过同样的推理得出相同的结论. 

 现在假设$k=3$,脸上沾满泥巴的孩子分别是$a$,$b$和$c$. 孩子$a$的论证如下. 假设我没有泥巴落在脸上. 根据$k=2$的情况,$b$和$c$在第二轮都会回答“是”. 他们没有这样做,我意识到假设是错误的,我的脸上也有泥巴. 因此在第三轮我会回答“知道”. $b$和$c$的论证也是类似的.

$k=3$的论证具有一般性,对一般的$k$也成立.

\begin{remark}
“泥泞的孩童”还有其他流行的陈述方式,比如“蓝眼睛红眼睛”. 一个岛上有100个人,其中有5个红眼睛,95个蓝眼睛. 这个岛有三个奇怪的宗教规则.
    \begin{enumerate}
        \item 他们不能照镜子,不能看自己眼睛的颜色. 
        \item 他们不能告诉别人对方的眼睛是什么颜色. 
        \item 一旦有人知道了自己的眼睛是红色,他就必须在当天夜里自杀.
    \end{enumerate}
岛民不知道具体有几个红眼睛. 

某天,有个旅行者到了这个岛上. 由于不知道这里的规矩,所以他在和全岛人一起狂欢的时候,一不留神说了一句话:“\emph{你们这里有红眼睛的人. }”假设这个岛上的人足够聪明,每个人都可以做出缜密的逻辑推理. 请问这个岛上将会发生什么?
\end{remark}

那么,为什么会这样呢?如果$k>1$,那么所有人都知道$p$:“至少有一个人脸上有泥巴”. 那么父亲说这句话的意义是什么?如果父亲没有说$p$,那么会发生什么?无论父亲问多少轮,所有孩子都只会回答“不知道”!(为什么)因此,父亲公开说了$p$,这是谜题的关键.

假设$k=2$,脸上沾满泥巴的孩子是$a$和$b$. 在父亲宣布$p$之前,$a$和$b$都知道$p$. 然而,他们并不知道对方知道$p$. $a$可能会有两种想法:
    \begin{itemize}
        \item 我的脸上有泥巴,所以$b$知道$p$.
        \item 我的脸上没有泥巴,$b$是唯一一个有泥巴的,所以$b$不知道$p$.
    \end{itemize}
当父亲宣布$p$之后,$a$知道了$b$知道$p$. 当第一轮$b$回答“不知道”之后,$a$可以用“$b$知道$p$”这一知识推出自己脸上有泥巴.

假设$k=3$,脸上沾满泥巴的孩子是$a$,$b$和$c$. 在父亲宣布$p$之前,$a$,$b$和$c$不仅知道$p$,而且知道彼此知道$p$. 以$a$的视角看,$b$能看到$c$脸上有泥巴,所以$a$知道$b$知道$p$. 但是,$a$,$b$,$c$都不知道所有人知道所有人知道$p$!


用$E^m p$表示所有人知道所有人知道……所有人知道($m$次)$p$. 在一般情况下,父亲没有宣布$p$之前,$E^k p$并不成立. 父亲宣布了$p$之后,对任意$m\geq 1$,$E^m p$都成立!因此,父亲宣布$p$带来了\emph{共同知识}\index{共同知识}. 有了共同知识,这一谜题就可以按照我们所讨论的方式进行下去.

我们曾经假设过所有人“观察力敏锐、聪慧且诚实”. 然而,这一假设并不足够. 我们必须假设所有人都知道所有人“观察力敏锐、聪慧且诚实”,所有人都知道所有人都知道所有人“观察力敏锐、聪慧且诚实”,……换言之,我们需要假设“所有人观察力敏锐、聪慧且诚实”是共同知识. 假设还是只有两个孩子$a,b$脸上有泥巴. 假如$a$不知道$b$是诚实的,即便$b$回答了“不知道”,$a$也无法从$b$的回答中得到任何额外的知识!

除了假设“所有人观察力敏锐、聪慧且诚实”是共同知识,我们还需要假设以下陈述是共同知识:
    \begin{itemize}
        \item 每个人都能看到所有除自己外的人.
        \item 每个人都听到了父亲说的话.
        \item 父亲是诚实的.
        \item 每个人都在每一轮进行了充分的推理.
        \item ……
    \end{itemize}
任何假设的破坏都会导致之前的讨论失效. 那么,为什么父亲宣布$p$就可以让$p$变成共同知识呢?

所有人都\emph{听到}父亲说$p$并不能产生共同知识. 假如父亲只是对每一个孩子单独宣布$p$. 所有人并不知道所有人都知道$p$,因而仅仅可以做到$E p$. 那么,所有人都\emph{知道}所有人听到父亲说$p$会如何呢?进一步假设每个孩子给每一个孩子都安装了窃听器,每个人都能够偷听每个人与父亲的谈话内容. 所有人并不知道所有人都知道所有人都知道$p$,因而仅仅有$E^2 p$. 因此,父亲宣布$p$会产生共同知识的核心原因是\emph{公开宣布},此时对每一个$m$都有$E^m p$.

“泥泞的孩童”谜题足以表明,关于“知道”的讨论远比想象的复杂. 关于“知道”和知识的研究在哲学中划归为\emph{知识论}\index{知识论}. 接下来,我们将使用模态逻辑来形式化关于“知道”和知识的讨论,这被称之为\emph{认知逻辑}\index{认知逻辑}.

我们将介绍处理知识的两种数学模型:
\begin{itemize}
    \item 一种源自Aumann,Harsanyi和Rubinstein等人,以Bayes概率论为基础,是偏经济学的学术风格;
    \item 另一种源自Kripke,Hintikka和Halpern等人,以模态逻辑为基础,是偏计算机科学和哲学的学术风格。
\end{itemize}
在这一章,我们主要讨论Bayes概率论的方法.

\section{不完全信息博弈(Bayes博弈)}

\section{Rubinstein电子邮件博弈}


\section{Aumann知识算子}

\section{习题}


\section{章末注记}