\chapter{Johnson-Lindenstrauss引理}\label{cha:J-L-Lemma}

我们已经在上一章看到,使用概率分布建模的信息论在机器学习中起到了举足轻重的作用。基于概率论的信息论总是考虑一个集合的对象的信息量,因此\textbf{数据}成为了这种方法论的核心前提:数据表征了一个集合的对象的某一特征。在这一章,我们将探讨机器学习中数据的特性,以及一种重要的数据压缩的原理:Johnson-Lindenstrauss引理。证明这一引理所用到的概率论技术是\textbf{矩法}\index{矩法},这是\emph{机器学习理论}\index{机器学习理论}中最为核心的几个技术之一。因此本章也可以看做机器学习理论的一个引论。

\section{机器学习中的数据}
从编码的角度来说,数据最简单的表示方法是使用固定长度的字符串。比如说,人的生理性别有男或者女两种,于是我们可以用字符串$0$表示男,$1$表示女。这样,我们就可以用一个长度为$1$的字符串来表示人的生理性别。人的属性还有很多,比如说年龄、身高、体重、学历、职业等等,这些属性都可以分别用固定长度的字符串来表示。于是,一个人就被抽象为了一个固定长度的字符串。

然而,这种表示方式必须要假定数据只取有限个值。有时候,为了简化建模和计算,我们还会考虑可以取无限个值的数据。我们看一个具体的例子,人的身高。从现代物理的角度来说,身高的变化是离散的,它有一个最小变化的单位。从生物学的角度来说,身高是有上界的,比如说所有人的身高都不会超过十米。因此,从理论上说,身高也只能取有限个值,所以也可以用字符串来表示。然而,更加方便的方式是假定身高是一个\emph{非负实数},因此用一个数而不是一个字符串来表示。

因此,更加常见的情况下,我们会用\emph{实数}或者\emph{整数}来编码数据。此时,将对象的多种属性按顺序排在一起,我们就得到了一个\emph{向量}。总而言之,在机器学习的框架,\emph{数据被表示成数值向量}. 例如,要表示一个人的年龄、身高、体重、学历、职业,我们需要
\begin{enumerate}
    \item 身高使用厘米作为单位;体重用千克作为单位;给学历编一个编号,比如$0$是高中,$1$是本科,$2$是硕士,$3$是博士,$-1$是其他;给职业也编号,例如$1$表示提示词工程师。
    \item 用一个五维向量来表示年龄、身高、体重、学历、职业。例如,$(20, 180, 70, 2, 1)$表示一个年龄为$20$岁,身高$180$厘米,体重$70$千克,学历为硕士,职业为提示词工程师的人。
\end{enumerate}

\begin{remark}
    \lhysays{介绍一下计算机中对数值的编码}。
\end{remark}

机器学习中,如此表示数据具备了独特的性质,一言以蔽之:\textbf{维数高,但是稀疏}。

\lhysays{介绍一下高维高斯分布的特点,以及图像处理中数据稀疏的特点。}

\section{矩法与集中不等式}
我们先引入示性函数的概念.
\begin{definition}[示性函数]\label{def:indicator-function}
    对事件$A$,定义$A$的\emph{示性函数}为一个从样本空间$\Omega$到$\R$的随机变量:
\begin{equation*}
    I(A)(\omega) := 
    \begin{cases}
        1,& \omega \in A. \\
        0,& \omega \notin A.
    \end{cases}
\end{equation*}
\end{definition}

从定义就可以得到如下基本性质:
\begin{proposition}\label{prop:indicator-function}
    设$A, B$是两个事件,则
    \begin{enumerate}
        \item $I(AB) = I(A) I(B)$.
        \item $I(A)^2 = I(A)$.
        \item $I(A\cup B) = I(A) +I(B) - I(AB) $.
    \end{enumerate}
\end{proposition}
\begin{proof}
    这里只作为一个示意,证明第三点,其他都类似。我们需要证明,对任意样本点$\omega\in\Omega$,我们有
    \[
        I(A\cup B)(\omega) = I(A)(\omega) + I(B)(\omega) - I(AB)(\omega).
    \]
    假设$\omega\in A\cup B$,那么左边等于$1$。我们分类讨论:
    \begin{itemize}
        \item 如果$\omega\in A$,那么右边第一项为$1$.
        \begin{itemize}
            \item 如果$\omega\in B$,那么右边第二项为$1$。此时自然也有$\omega\in AB$,所以右边第三项为$1$,因此右边等于$1$,等于左边。
            \item 如果$\omega\notin B$,那么右边第二项为$0$。此时自然也有$\omega\notin AB$,所以右边第三项为$0$,因此右边等于$1$,等于左边。
        \end{itemize}
        \item 如果$\omega\notin A$,那么右边第一项为$0$。此时必须有$\omega\in B$,所以右边第二项为$1$。但是此时自然也有$\omega\notin AB$,所以右边第三项为$0$,因此右边等于$1$,等于左边。
    \end{itemize}
    如果$\omega\notin A\cup B$,讨论类似,这里不再赘述。
\end{proof}

示性函数之所以重要,是因为它联系了期望与概率。我们先来看一个显然的命题:
\begin{proposition}\label{prop:expectation-of-indicator-function}
    设$A$是一个事件,则
    \[
        \E[I(A)] = \Pr(A).
    \]
\end{proposition}

示性函数可以把对概率的计算变成对期望的计算。回忆期望的线性性:设$a,b\in\R$,$X,Y$是有期望的随机变量,那么成立
\[\E(aX+bY)=a\E(X)+b\E(Y).\]
利用期望的线性性,示性函数可以导出很多概率恒等式与不等式. 例如:容斥公式
\begin{align*}
    \Pr(A\cup B) &= \E[I(A\cup B)] = \E[I(A) + I(B) - I(AB)] \\
    &= \E[I(A)] +  \E[I(B)] -  \E[I(AB)]\\
    &= \Pr(A) + \Pr(B) - \Pr(AB).
\end{align*}

对于概率论以及机器学习理论来说,下面的这个不等式非常重要:

\begin{theorem}[Markov不等式]\label{thm:markov-inequality}\index{Markov不等式}
    如果$X$是非负有期望的随机变量,$a>0$,那么
        \[
            \Pr(X\geq a) \leq \frac{\E[X]}{a}.
        \]
\end{theorem}

\begin{proof}
直接利用示性函数,我们有:
    \begin{align*}
        \E[X] = &\E[X I(X\geq a)+X I(X<a)]\\
        =&\E[\underbrace{X I(X\geq a)}_{\geq aI(X\geq a)}]+\E[\underbrace{XI(X<a)}_{\geq 0}]\\
        \geq& a\E[I(X\geq a)]=a\Pr(X\geq a).
    \end{align*}
\end{proof}

\begin{remark}
    为了使得证明有效,我们必须要假设上面的推导中出现的期望都是存在的,当然这实际上很容易验证。为了避免不必要的技术细节,在后面的所有证明以及推导中,我们都会默认写出来的期望是存在的,不再赘述。
\end{remark}

我们利用Markov不等式可以直接得到以下结果. 
\begin{corollary}[Chebyshev不等式]\label{cor:chebyshev-inequality}\index{Chebyshev不等式}
设$X$是任意有方差的随机变量,那么对任意$a>0$,成立
    \[
        \Pr(|X - \E[X]| \geq a) \leq \frac{\var(X)}{a^2}.
    \]
\end{corollary}
\begin{proof}
设$Y=(X-\E[X])^2$,$t=a^2$,那么$Y$是非负随机变量,且$\E[Y]=\var(X)$,于是由Markov不等式,我们有
\begin{align*}
    \Pr(|X - \E[X]| \geq a) &= \Pr(|X - \E[X]|^2 \geq a^2) \\
    &=\Pr(Y \geq t) \\
    &\leq \frac{\E[Y]}{t}=\frac{\var(X)}{a^2}.
\end{align*}
\end{proof}

Chebyshev不等式告诉我们采样到偏离其期望的概率有一个上界. 像这样利用矩(即$\E[f(X)]$)来估计概率上界的方法被称为\emph{矩法}\index{矩法}. 

实际上,很多情况下,偏离期望是非常小概率的事件,远小于上面的估计值. 为了得到更精确的上界,我们需要一些技巧. 考虑任意随机变量$X$,对$\lambda >0$,
\[
X\geq a \iff \lambda X \geq \lambda a \iff e^{\lambda X} \geq e^{\lambda a}.
\]
由Markov不等式(如何得到?),
\[
\Pr(X\geq a) = \Pr\left(e^{\lambda X} \geq e^{\lambda a}\right) \leq e^{- \lambda a}\cdot \E\left[e^{\lambda X}\right]. 
\]
注意到这个不等式应该对任意$\lambda > 0$ 成立,所以
\[
\Pr(X\geq a) \leq \inf_{\lambda >0} e^{- \lambda a}\cdot \E \left[e^{\lambda X}\right].
\]

以上方法可以得到概率更精确的上界。这样用指数进行推导的方法称为\emph{指数矩}\index{指数法}或\emph{Cramér-Chernoff方法}\index{Cramér-Chernoff方法}. 

利用指数矩,我们可以更加精确地研究Chebyshev不等式中随机变量所表现出来的性质,这种性质被称为概率的\emph{集中性}\index{集中性}. 我们可以用\emph{集中不等式}\index{集中不等式}来刻画这样的性质。这样的不等式描述随机变量$X$有多大概率偏离某个值$\mu$多少值($t$),它表现为
\[
\Pr(| X - \mu| \geq t) \leq \text{ 小量}.
\]
通常来说,$\mu$是随机变量的期望或者中位数,在这本书中,只会讨论关于期望的集中性. 我们可以看到Chebyshev不等式就是一种特殊的集中不等式,但是它的界太松. 利用指数矩,我们将证明界更紧的Hoeffding不等式和Chernoff不等式. 

\begin{theorem}[Hoeffding 不等式]\label{thm:hoeffding-inequality}\index{Hoeffding不等式}
    设$X_1, \dots, X_n$相互独立且服从对称Bernoulli分布,即$X_i$满足$\Pr(X_i=1)=1-\Pr(X_i=-1)=1/2$. 考虑向量$a = (a_1, \dots, a_n) \in \R^n$,对任意$t\geq0$,我们有
    \[
        \Pr\left(\sum_{i=1}^n a_i X_i \geq t\right)\leq \exp \left( - \frac{t^2}{2\norm{a}^2_2}\right).
    \]
\end{theorem}
\begin{proof}
由指数矩,我们有
\[
\begin{aligned}
    \Pr\left(\sum_{i=1}^n a_i X_i \geq t\right) &= \Pr\left(\exp\left(\lambda \sum_{i=1}^n a_i X_i\right) \geq \exp\left(\lambda t\right)\right) \\
    &\leq e^{-\lambda t} \E \left[\exp\left(\lambda \sum_{i=1}^n a_i X_i\right)\right]\\
    &=  e^{-\lambda t}\prod_{i=1}^n \E\left[\exp\left(\lambda a_i X_i\right)\right].
\end{aligned}
\]
这个不等式对任意$\lambda > 0$都成立. 
利用$X_1, \dots, X_n$服从对称Bernoulli分布,得到(习题)\lhysays{习题}
\begin{equation}
e^{-\lambda t}\prod_{i} \E\left[\exp\left(\lambda a_i X_i\right)\right]\leq\exp \left(-\lambda t + \frac{\lambda^2}{2} \sum_{i} a_i^2\right).\label{eq:exponential-moment}
\end{equation}
由于这一不等式对任意$\lambda > 0$都成立,根据二次函数的性质,取$\lambda = t / \sum_i a_i^2$,可得
\[
\begin{aligned}
    \inf_{\lambda>0}\exp \left(-\lambda t + \frac{\lambda^2}{2} \sum_{i} a_i^2\right) &=\exp\left( -\frac{t}{\sum_i a_i^2} t + \frac12 \left(\frac{t}{\sum_i a_i^2}\right)^2 \sum_{i} a_i^2\right) \\
    &= \exp \left( - \frac{t^2}{2\norm{a}_2^2}\right).
\end{aligned}
\]
\end{proof}

利用相同的证明技巧,我们可以证明一般形式的Hoeffding不等式,我们把证明留作习题.\lhysays{习题}

\begin{theorem}[Hoeffding 不等式,一般情形]\label{thm:hoeffding-inequality-general}\index{Hoeffding不等式}
    设$X_1, \dots, X_n$是相互独立的随机变量,对任意$i$都成立$X_i \in [m_i, M_i]$。那么对任意$t\geq0$,我们有
    \[
        \Pr\left(\sum_{i=1}^n (X_i - \E [X_i]) \geq t\right) \leq \exp \left( - \frac{2t^2}{\sum_{i=1}^n (M_i - m_i)^2}\right).
    \]
\end{theorem}

下面我们介绍 Chernoff 不等式. 
\begin{theorem}[Chernoff 不等式]\label{thm:chernoff-inequality}\index{Chernoff不等式}
    设$X_1, \dots, X_n$是相互独立的随机变量,分别服从于参数为$p_1, \dots, p_n$的Bernoulli分布.  记$\sum_{i=1}^n X_i$的期望为$\mu = \sum_{i=1}^n p_i$,对于任意$t > \mu$,我们有
    \[
        \Pr\left( \sum_{i=1}^n X_i \geq t\right) \leq e^{-\mu} \left(\frac{e\mu}{t}\right)^t.  
    \]
    这里$e$是自然对数的底数.
\end{theorem}

\begin{proof}
和证明Hoeffding 不等式的第一步相同,我们先利用指数矩,对任意$\lambda > 0$有
    \[
    \begin{aligned}
        \Pr\left( \sum_{i=1}^n X_i \geq t\right) &\leq e^{-\lambda t}\prod_{i=1}^n \E\left[\exp (\lambda X_i)\right].
    \end{aligned}
    \]
然后,将$\prod_{i=1}^n\E[\exp (\lambda X_i)]$进一步放缩:
    \[
    \begin{aligned}
        \prod_{i=1}^n \E \left[\exp (\lambda X_i)\right]&= \prod_{i=1}^n \left(e^\lambda p_i + (1 - p_i) \right) \\
         &\leq \prod_{i=1}^n \exp\left((e^\lambda -1)p_i \right). \\
    \end{aligned}
    \]
因此
    \[
    \begin{aligned}
        \Pr\left( \sum_{i=1}^n X_i \geq t\right) &\leq e^{-\lambda t}\prod_{i=1}^n \exp\left((e^\lambda -1)p_i \right) \\
        &= e^{-\lambda t} \exp\left((e^\lambda -1)\sum_{i=1}^n p_i \right)\\
        &=\exp\left(\mu e^\lambda -t\lambda-\mu \right).
    \end{aligned}
    \]
右边的最小值在$\lambda = \log (t/\mu)$取得,代入得到:
    \[
    \begin{aligned}
        \Pr\left( \sum_{i=1}^n X_i \geq t\right) \leq e^{-\mu} \left(\frac{e\mu}{t}\right)^t.
    \end{aligned}
    \]
\end{proof}

\section{J-L引理的陈述与证明}
有了上面矩法的准备,我们可以陈述并证明J-L引理了.
\begin{theorem}[Johnson-Lindenstrauss 引理]\label{thm:johnson-lindenstrauss-lemma}\index{Johnson-Lindenstrauss引理}
给定$N$个单位向量$v_1,\dots,v_N\in\R^m$和$n >24\log N/\epsilon^2$,随机矩阵$A\in \R^{n\times m}$每个元素独立重复采样自$\mathcal N(0,1/n)$,$\epsilon \in (0,1)$是给定常数,那么至少有$(N-1)/N$的概率,使得对所有的$i\neq j$,都成立
    \[
        (1 - \epsilon)\norm{v_i-v_j}_2^2 <\norm{Av_i-Av_j}_2^2 < (1 + \epsilon)\norm{v_i-v_j}_2^2.
    \]
\end{theorem}

我们可以把$n$理解成降维后的维度,$Av_i$是降维后的向量. 这个引理告诉我们只要$n > 24\log N/\epsilon^2$,我们就可以用变换$A$把原本$m$维的向量映射到$n$维空间,并且保证它们相对距离的偏离不超过$\epsilon$,因此我们可以把$A$看成一个损失率很低的压缩变换。不严格地说,\emph{塞下$N$个向量,只需要$\O(\log N)$维空间.}

下面我们开始证明J-L引理。为了看出来证明的思路,我们第一个任务是算出压缩后$Av_i$的分布. 我们首先回忆一些正态向量的基本性质。关于正态向量的讨论,可以参考\Cref{???}。

\begin{proposition}\label{prop:gaussian-vector}
假设$u\sim\mathcal N(\mu,\Sigma)$是一个$n$维正态向量,$M$是一个$m\times n$矩阵,那么$Mu$是一个$m$维正态向量,并且$Au\sim \mathcal N(M\mu,M\Sigma M^T)$.
\end{proposition}

利用这一个命题,很容易可以得到$Av_i$的分布:

\begin{lemma}\label{lemma:gaussian-vector}
    假设$u\in\R^m$是一个单位向量,那么$Au\sim \mathcal N(0,n^{-1}I_n)$.
\end{lemma}
\begin{proof}
    将$A$视作一个$mn$维的正态向量,注意到,$(Au)_i = \sum_{j=1}^m A_{ij}u_j$,所以$Au$是一个从向量$A$线性变换得到的向量。根据\Cref{prop:gaussian-vector},$Au$是一个正态向量,只需计算它的期望和协方差矩阵。
   
   注意到,对不同的$i$,向量$(A_{ij})_{j}$相互是独立的,所以分量$(Au)_i$相互也是独立的,因此只需要计算正态变量$(Au)_i$的期望与方差。其期望为$\sum_{j=1}^m 0\cdot u_j = 0$,方差为
   \[\sum_{j=1}^m \left(\frac{1}{n}\cdot u_j^2\right) = \frac{1}{n}.\] 
   所以$Au$的期望是$0$,协方差矩阵是$n^{-1}I_n$.
\end{proof}

然而,我们关心的其实不单单是$Av_i$的分布,更重要的其实是$Av_i-Av_j$的分布,即压缩后的向量之间的相对距离,幸运的是,我们并不需要做额外的什么计算,我们直接有如下结果:

\begin{lemma}\label{lemma:gaussian-vector-diff}
    向量$u=\frac{v_i-v_j}{\norm{v_i-v_j}_2}$是一个单位向量,因此$Au\sim \mathcal N(0,n^{-1}I_n)$.
\end{lemma}

J-L引理实际上在说,$\norm{Au}_2$偏离$1$的一定程度的概率是非常小的。于是,为了证明J-L引理,我们最重要的任务是给出$Au$这样向量模长的集中不等式:
\begin{lemma}[单位模引理]\label{lemma:unit-mod-lemma}\index{单位模引理}
    设 $u\sim \mathcal N(0,n^{-1}I_n)$,$\epsilon \in (0,1 )$是给定的常数,那么我们有
    \[
        \Pr(|\norm{u}_2^2 - 1| \geq \epsilon) \leq 2\exp \left(-\frac{\epsilon^2 n}{8}\right).
    \]
\end{lemma}
注意到$\E[\norm{u}_2^2]=n\cdot(1/n)=1$,所以这个引理在说高维空间中,如果正态向量具有单位模长平方期望,那么它的模长就会集中在单位长度附近,因此称为单位模引理.

\begin{proof}
$|\norm{u}_2^2 - 1| \geq \epsilon$发生有两种可能,$ \norm{u}_2^2 - 1 \geq \epsilon$和$ 1 - \norm{u}_2^2 \geq \epsilon$. 我们先来计算$ \norm{u}_2^2 - 1 \geq \epsilon$的概率,根据指数矩,
    \[
    \Pr\left(\norm{u}_2^2 - 1 \geq \epsilon\right) \leq \inf_{\lambda > 0} \left\{e^{- \lambda (\epsilon + 1)} \E \left[e^{\lambda \norm{u}_2^2}\right]\right\}. 
    \]
因为$u$的各个分量是相互独立的,所以我们可以把$\norm{u}_2^2$展开
    \[
    \E\left[e^{\lambda \norm{u}_2^2}\right] = \E\left[e^{\lambda \sum_i u_i^2}\right] = \E\left[\prod_i e^{\lambda u_i^2}\right] = \prod_i \E\left[e^{\lambda u_i^2}\right]. 
    \]
可以算得$\E\left[e^{\lambda u_i^2}\right]= \sqrt{n/(n-2\lambda)}$\lhysays{习题},所以
    \[
    \Pr\left(\norm{u}_2^2 - 1 \geq \epsilon\right) \leq \inf_{\lambda > 0} \left\{e^{- \lambda (\epsilon + 1)} \left(\frac{n}{n-2\lambda} \right)^{n/2}\right\}. 
    \]
可以验证最小值在$\lambda = n\epsilon/(2(1+\epsilon))$处取到,代入可得
    \[
    \Pr\left(\norm{u}_2^2 - 1 \geq \epsilon\right) \leq e^{n(\log (1+\epsilon) - \epsilon)/2} \leq e^{-n\epsilon^2/8}. 
    \]
这里最后一个不等号使用了不等式$\log(1+\epsilon)\leq\epsilon-\epsilon^2/4$. 

计算$1 - \norm{u}_2^2 \geq \epsilon$的概率的过程和$\norm{u}_2^2 - 1 \geq \epsilon$几乎完全相同的,可以得到
    \[
    \Pr\left(1 - \norm{u}_2^2 \geq \epsilon\right) \leq e^{n(\log (1-\epsilon) + \epsilon)/2} \leq e^{-n\epsilon^2/8}. 
    \]
    \begin{align*}
        \Pr\left(| \norm{u}_2^2 - 1| \geq \epsilon\right) &\leq \Pr\left(\norm{u}_2^2 - 1 \geq \epsilon\right) + \Pr\left(1 - \norm{u}_2^2 \geq \epsilon\right) \\
        &\leq 2e^{-n\epsilon^2/8}. 
    \end{align*}
\end{proof}

有了单位模引理,我们就可以很容易证明J-L引理了。将\Cref{lemma:gaussian-vector-diff} 中的$u$带入单位模引理,得到
\[
    \Pr\left(\left|\norm{\frac{A(v_i-v_j)}{\norm{v_i-v_j}_2}}_2^2 - 1 \right| \geq \epsilon\right)\leq  2\exp \left(-\frac{\epsilon^2 n}{8}\right). 
\]
这个结论对任意$i\neq j$成立,因此遍历所有$i,j$对,可得
\[
\begin{aligned}
    \Pr\left(\exists (i,j): \left|\norm{\frac{A(v_i-v_j)}{\norm{v_i-v_j}_2}}_2^2 - 1 \right| \geq \epsilon\right)
    &\leq 2 \sum_{i\neq j} \exp \left(-\frac{\epsilon^2 n}{8}\right) \\
    &=  2  \binom{N}{2}\exp \left(-\frac{\epsilon^2 n}{8}\right). 
\end{aligned}
\]
换言之,对任意$i,j$,$\left|\norm{\frac{A(v_i-v_j)}{\norm{v_i-v_j}_2}}_2^2 - 1 \right| < \epsilon$都成立的概率不小于
\[
1 - 2  \binom{N}{2}\exp \left(-\frac{\epsilon^2 n}{8}\right) = 1 - N(N-1)\exp \left(-\frac{\epsilon^2 n}{8}\right). 
\]
代入$n > \frac{24\log N}{\epsilon^2}$,可得这一概率
\[
1 - N(N-1)\exp \left(-\frac{\epsilon^2 n}{8}\right) \geq 1 - N(N-1)N^{-3}\geq 1 - N^{-1} = \frac{N-1}{N}. 
\]

很多时候,我们关心的并不是向量间的距离,而是向量的内积(比如使用\emph{余弦度量}\index{余弦度量}的时候),这时候我们可以使用内积版本的J-L的引理:
\begin{theorem}[J-L引理,内积形式]\label{thm:johnson-lindenstrauss-lemma-inner-product}\index{Johnson-Lindenstrauss引理}
    给定$N$个单位向量$v_1,\dots,v_N\in\R^m$和$n > 24\log N/\epsilon^2$,随机矩阵$A\in \R^{n\times m}$每一个元素都独立重复采样自$\mathcal N(0,1/n)$,$\epsilon \in (0,1)$是给定常数,那么至少有$(N-1)/N$的概率,使得对所有的$i\neq j$,都成立
    \[
        |\langle Av_i, Av_j\rangle - \inner{v_i}{v_j}|< \epsilon.
    \]
\end{theorem}    
\begin{proof}
由原始J-L引理可知,至少有$\frac{N-1}{N}$的概率满足对于任意$i\neq j$有:
    \[
    \begin{aligned}
        &(1 - \epsilon)\norm{v_i-v_j}_2^2 < \norm{Av_i-Av_j}_2^2< (1 + \epsilon)\norm{v_i-v_j}_2^2, \\
        &(1 - \epsilon)\norm{v_i+v_j}_2^2 < \norm{Av_i+Av_j}_2^2 < (1 + \epsilon)\norm{v_i+v_j}_2^2. 
    \end{aligned}
    \]
我们将第一行乘$-1$加到第二行可以得到
    \[
        4\inner{v_i}{v_j} - 2 \epsilon (\norm{v_i}_2^2 + \norm{v_j}_2^2) <4\langle Av_i, Av_j\rangle < 4\inner{v_i}{v_j} + 2 \epsilon (\norm{v_i}_2^2 + \norm{v_j}_2^2).
    \]
因为$v_i,v_j$是单位向量,所以上式等价于$|\langle Av_i, Av_j\rangle - \inner{v_i}{v_j}|<\epsilon$. 
\end{proof}

\section{J-L引理的应用}
回顾:J-L引理描述的是对于$N$个向量,我们可以将它们降到$\O(\log N)$维空间,并将相对距离的误差控制在一定范围内. 它的内容本身就和降维相关,所以最基本的应用就是直接作为降维方法. 许多其它算法例如局部敏感哈希(LSH)、随机SVD,本质上也都依赖J-L引理. 除此之外,J-L引理对机器学习模型中维度的选择提供了一些理论解释. 下面我们将介绍两个具体的应用案例. 

\begin{example}[词向量维度]
在NLP的发展中产生了像Word2Vec、GloVe这样经典的词向量模型和基于注意力机制\index{注意力机制}的各种大语言模型。这里一个问题自然的问题是,当我们对$N$个单词进行建模,词向量的维度选择多少比较合适? 如果维度过高,会使得后续的计算变得更加复杂. 如果维度过低,会无法完全表达出这些单词本身的信息. 对于这一问题,J-L引理给出了一个比较直接的结论,$\O(\log N)$空间足以容纳下$N$个单词. 但是要注意,这一结论成立的前提是正态随机矩阵,然而单词的空间是否符合正态分布是不知道的,所以这一结果只是从理论上给了一个直观,选择什么样的$n$还是由具体的实验效果来决定. 
\end{example}

\begin{example}[多头注意力]
在\textbf{注意力机制}\index{注意力机制}中,我们往往会先把 \verb#head_size# 降低到$64$再做内积. 那么一个很自然的问题是,\verb#head_size# 为$64$的注意力机制是否足以拟合任何概率分布?具体来说,注意力的计算公式为
    \[
    a_{ij} = \frac{e^{\inner{q_i}{k_j}}}{\sum_{j=1}^L e^{\inner{q_i}{k_j}}}.
    \]
其中$q_i,k_j \in \R^d$。

我们希望能够做到:给定任意的概率矩阵$(p_{ij})$,上述$(a_{ij})$都能够很好的逼近$(p_{ij})$.换言之,给定$(p_{ij})$和维度$d$,我们是否能找到一组$q_1,\dots,q_L,k_1,\dots,k_L\in \R^d$,使得对应项$a_{ij}$与$p_{ij}$足够接近. 其实这就和词向量模型的维度选择问题是等价的. 词向量的维度变成了\verb#head_size#,词表大小变成了序列长度. J-L引理告诉我们的答案依然是只需要$\O(\log N)$的空间就足以容纳下$N$个单词,一个很粗糙的计算,
\end{example}
\lhysays{这两个例子写得更详细一些。}
\section{习题}

\section{章末注记}