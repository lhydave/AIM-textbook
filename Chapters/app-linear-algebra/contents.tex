\chapter{线性代数基础}\label{chap:linear-algebra}

\section{线性空间}

从动机上说,线性空间试图将$\R^n$或者$\C^n$这样的集合连同他们上面的代数结构抽象出来。除此之外,函数和无穷数列的集合也是非常重要的对象,比如说$\R$上的连续函数组成的集合$C(\R)$,或者具有“模长”的无穷实数列($\ell^2(\R)$-空间):
\[\ell^2(\R)=\left\{x=(x_1,x_2,\cdots)\in\R^\infty:\sum_{i=1}^\infty x_i^2<\infty\right\}.\]

我们将这些对象的共性抽象出来,得到线性空间的概念。线性空间都是基于某个域定义的,我们先给出域的定义。

\begin{definition}[域]\index{域}
一个\textbf{域}是一个集合$F$,其上定义了两种二元运算:加法$+$和乘法$\cdot$,他们都是$F\times F$到$F$的映射,满足下面的公理:
\begin{enumerate}
    \item (结合律)对于任意的$a,b,c\in F$,有$(a+b)+c=a+(b+c)$和$(a\cdot b)\cdot c=a\cdot (b\cdot c)$;
    \item (交换律)对于任意的$a,b\in F$,有$a+b=b+a$和$a\cdot b=b\cdot a$;
    \item(分配律)对于任意的$a,b,c\in F$,有$a\cdot(b+c)=a\cdot b+a\cdot c$。
    \item (单位元)存在唯一的两个元素$0,1\in F$,使得对于任意的$a\in F$,有$a+0=a$和$a\cdot 1=a$;
    \item (加法逆元)对于任意的$a\in F$,存在唯一$b\in F$,使得$a+b=0$,记$b$作$-a$;
    \item (乘法逆元)对于任意的$a\in F$,如果$a\neq 0$,则存在唯一$b\in F$,使得$a\cdot b=1$,记$b$作$a^{-1}$;
\end{enumerate}
通常将$a\cdot b$写作$ab$,并且乘法的优先级高于加法,即$ab+c=(ab)+c$。
\end{definition}

域的重要例子包括有理数域$\Q$,实数域$\R$和复数域$\C$,他们都是无限域。我们将在后面的内容中使用这些域。

\begin{definition}[线性空间,向量空间]\index{线性空间}\index{向量空间}
设$V$是一个集合,$F$是一个域。如果在$V$上定义了两种运算:加法$+$和数乘$\cdot$,使得$V$满足下面的公理:
\begin{enumerate}
    \item ($V$的结合律)对于任意的$x,y,z\in V$,有$(x+y)+z=x+(y+z)$;
    \item ($V$的交换律)对于任意的$x,y\in V$,有$x+y=y+x$;
    \item (加法零元)存在唯一的元素$0\in V$,使得对于任意的$x\in V$,有$x+0=x$;
    \item (加法逆元)对于任意的$x\in V$,存在唯一$y\in V$,使得$x+y=0$,记$y$作$-x$;
    \item 对于任意的$x\in V$,有$1\cdot x=x$;
    \item 对于任意的$a,b\in F$和$x\in V$,有$(ab)\cdot x=a\cdot (b\cdot x)$;
    \item 对于任意的$a\in F$和$x,y\in V$,有$a\cdot(x+y)=a\cdot x+a\cdot y$;
    \item 对于任意的$a,b\in F$和$x\in V$,有$(a+b)\cdot x=a\cdot x+b\cdot x$。
\end{enumerate}
则称$V$是一个\textbf{$F$-线性空间},简称\textbf{线性空间},也称\textbf{向量空间}。$V$中的元素被称为\textbf{向量}\index{向量}. 通常将数乘$a\cdot x$写作$ax$,并且乘法的优先级高于加法,即$a\cdot x+y=(a\cdot x)+y$。
\end{definition}

“线性”一词的含义是指的$ax+by$这种形式的数学对象,线性代数就是研究这种对象的学科。线性空间的典型例子包括:
\begin{itemize}
    \item $\R^n$和$\C^n$.
    \item $M_{m\times n}(F)$,即所有$m\times n$矩阵组成的集合.
    \item $C(\R)$,即$\R$上的连续函数组成的集合.
    \item $C^k(\R)$,即$\R$上的$k$次连续可微函数组成的集合.
    \item $\ell^2(\R)$,即所有二次可和的实数序列组成的集合.
\end{itemize}

如同所有其他的代数结构,线性空间也有各式各样构造新的线性空间的方法。为了看出来线性空间本质的特性,我们有如下引理:

\begin{lemma}\label{lemma:linear-space-subspace}
设$V$是$F$-线性空间,$W$是$V$的一个子集。则$W$是一个线性空间当且仅当对任意$a,b\in F$和$x,y\in W$,有$ax+by\in W$。
\end{lemma}
\begin{proof}
    按照定义即可验证。
\end{proof}

我们给$ax+by$这样的对象一个正式的定义。

\begin{definition}[线性组合]\index{线性组合}
设$V$是$F$-线性空间,$x_1,\cdots,x_n\in V$,$a_1,\cdots,a_n\in F$,则称$a_1x_1+\cdots+a_nx_n$是$x_1,\cdots,x_n$的一个\textbf{线性组合}。
\end{definition}

接下来,基于某些特定的线性空间,我们构造各种新的线性空间。

\begin{definition}[线性子空间]\index{线性子空间}
设$V$是$F$-线性空间,$W$是$V$的一个子集。如果$W$是一个线性空间,则称$W$是$V$的一个\textbf{线性子空间}。
\end{definition}

例如,$\Q$是$\R$的一个线性子空间,但$\Z$不是$\R$的一个线性子空间。再比如,当$k<l$,$C^k(\R)$是$C^l(\R)$的一个线性子空间。

\begin{definition}[乘积空间]\index{乘积空间}
设$V_1,\cdots,V_n$是$F$-线性空间,则$V_1\times\cdots\times V_n$是一个$F$-线性空间,其中加法和数乘分别定义为
\begin{align*}
    &(x_1,\cdots,x_n)+(y_1,\cdots,y_n)=(x_1+y_1,\cdots,x_n+y_n),\\
    &a(x_1,\cdots,x_n)=(ax_1,\cdots,ax_n).
\end{align*}
\end{definition}

例如,$\R^n$就是$n$个$\R$的乘积空间,$M_{m\times n}(F)$就是$m\times n$个$F$的乘积空间。

接下来,我们按照\emph{表示论}\index{表示论}的观点,引入基的概念。线性空间是一个非常抽象的数学概念,因此我们需要一些具体的元素去表示这整个空间。

\begin{definition}[生成集]\index{生成集}
设$V$是$F$-线性空间,$S\subseteq V$,如果$V$中的每一个元素都是$S$的线性组合,则称$S$是$V$的一个\textbf{生成集}。

更一般地,任意一个$S\subseteq V$,我们可以定义 \textbf{$S$生成的线性子空间}为所有$S$的线性组合的集合,记为$\Span(S)$。
\end{definition}

我们希望用尽可能少的元素来表示整个线性空间,为此,我们需要把“可表示”这样的概念严格化。

\begin{definition}[线性相关]\index{线性相关}
设$V$是$F$-线性空间,$S\subseteq V$,如果存在$x_1,\cdots,x_n\in S$,$a_1,\cdots,a_n\in F$,使得$a_1x_1+\cdots+a_nx_n=0$,且至少有一个$a_i\neq 0$,则称$S$是\textbf{线性相关}的,否则称$S$是\textbf{线性无关}的。
\end{definition}

$S$线性相关意味着$S$中的一些元素可以被另一些元素的线性组合表示出来,因而$S$中有一些冗余。线性无关意味着$S$中的元素都是必要的,没有冗余。由此,我们可以给出基的定义。

\begin{definition}[基]\index{基}
设$V$是$F$-线性空间,$S\subseteq V$,如果$S$是线性无关的,并且$\Span(S)=V$,则称$S$是$V$的一个\textbf{基}。
\end{definition}

线性空间的一个核心定理是基的存在性定理。

\begin{theorem}[基的存在性定理]\label{thm:existence-of-basis}
设$V$是$F$-线性空间,则$V$中存在一个基。
\end{theorem}

要注意,这一定理并不是平凡的。首先,基是线性无关的集合,所以$V$本身通常就不是基。此外,这一定理要求有一个线性无关的集合$S\subseteq V$,任意向量$x\in V$都可以用$S$中\emph{有限个}元素的线性组合来表示,这样的$S$并不容易找到。该定理的证明是构造性的,这一构造依赖于选择公理(或者Zorn引理),我们在此略去。

基的典型例子包括:
\begin{itemize}
    \item $\R^n$的标准基是$\{e_1,\cdots,e_n\}$,其中$e_i$是第$i$个分量为$1$,其余分量为$0$的向量;
    \item $M_{m\times n}(F)$的标准基是$\{E_{ij}:1\leq i\leq m,1\leq j\leq n\}$,其中$E_{ij}$是第$i$行第$j$列为$1$,其余元素为$0$的矩阵;
    \item $\ell^2(\R)$的标准基是$\{e_1,e_2,\cdots\}$,其中$e_i$是第$i$个分量为$1$,其余分量为$0$的实数列。
\end{itemize}

给定一个基,我们可以用基来表示线性空间中的元素,容易证明,这一表示是唯一的。因此,我们可以把线性空间中的元素看成基的线性组合,因而有了下面的定义。

\begin{definition}[坐标]\index{坐标}
设$V$是$F$-线性空间,$S$是$V$的一个基,$x\in V$,如果$x=\sum_{v\in S} a_v v$,则称$(a_v)_{v\in S}$是$x$在基$S$下的\textbf{坐标}。特别地,如果$S$是有限集,那么$(a_v)_{v\in S}$是一个有限元组,我们也称其为$x$在基$S$下的坐标。
\end{definition}

例如,$\R^3$的标准基是$\{e_1,e_2,e_3\}$,那么任意$x\in\R^3$都可以表示为$x=a_1e_1+a_2e_2+a_3e_3$,其中$a_i$是$x$的第$i$个分量。因此,我们可以把$x$看成一个三元组$(a_1,a_2,a_3)$,这就是$x$在标准基下的坐标。这样的讨论也适用于$\R^n$或$\C^n$.

线性空间的基可以衡量线性空间的复杂程度,基元素越少,线性空间越简单。我们可以定义维数来衡量线性空间的复杂程度。

\begin{definition}[维数]\index{维数}
设$V$是$F$-线性空间,如果$V$的一个基有限,则称$V$是\textbf{有限维}的,否则称$V$是\textbf{无限维}的。有限维线性空间的基的元素个数称为$V$的\textbf{维数},记为$\dim V$。
\end{definition}
这一定义隐含的事实是,如果$V$有有限基,那么所有基都是有限的,并且任意两个基的元素个数相同。我们这里略去证明。

例如,$\R^n$的维数是$n$,$M_{m\times n}(F)$的维数是$mn$,$C^k(\R)$和$\ell^2(\R)$都是无穷维的。

\section{线性映射}