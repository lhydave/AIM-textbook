\chapter{不动点理论}\label{chap:fixed-point-theory}

如果有一个长满毛发的球体,你能够把它所有的毛发都梳理平顺吗?做个实验就会发现,这好像是做不到的,总会有一根毛发直立不倒,或某个地方没有毛发覆盖. 实际上,早在1912年,Brouwer就从数学上严格证明了上述现象,我们现在称之为\emph{毛球定理}.

你是否在大型商场或者公园里经常看到“您在此处”的地图标识牌?为什么可以有这样的标识,它真的表明了你的位置吗?

你是否相信,地球上有两个地方,它们分别位于地球的对径点,并且温度和湿度完全相同?

这些问题看似毫无关联,但它们都有一个共同的数学背景:\emph{不动点理论}.

不动点的定义是非常直接的,考虑一个集合$X$以及它到自身的映射$f:X\to X$,元素$a\in X$称为映射$f:X\to X$的\textbf{不动点},如果$f(a)=a$. 

除了生活中,不动点理论对于优化来说也是非常重要的. 考虑优化算法$A$,它在函数$f$上的的收敛性如何?算法运行所产生的点列记为$\{x_n\}$,它满足
\[x_{n+1}=A(x_n).\] 

如果关注序列$x_n$本身,要分析收敛性,我们需要通过寻找不同量之间的联系,比如$f(x_n)$和$f(x_{n+1})$之间的关系. 在数学中,这样的思路被归类到了\emph{数学分析}中.  

一种更为抽象的做法是,我们直接看算法$A$本身的性质. 此时,要想说明$A$收敛,我们要说明$A$有一个“吸收点”,即不管从何处出发,经过若干次迭代,都会收敛到这个点附近. 这样的思路是更加现代的数学方法,它被归类到了\emph{算子法}和\emph{泛函分析}中. 

我们将看到,从算子的角度来理解收敛性,最终问题就归结到了\emph{不动点理论}. 本章将介绍两种不动点存在性定理,并介绍他们的应用.

\section{Banach不动点定理}

首先,我们需要引入一些度量空间相关的概念,更系统的的讨论请参阅\Cref{chap:calculus}.

\begin{definition}[度量与度量空间]
集合$X$上的\textbf{度量}(或\textbf{距离})$d$是映射
\[d:X\times X\to \R\]
满足条件
\begin{itemize}
\item 非负性:$d(x_1,x_2)\geq 0$,并且$d(x_1,x_2)=0\iff x_1=x_2$.
\item 对称性:$d(x_1,x_2)=d(x_2,x_1)$.
\item 三角不等式:$d(x_1,x_3)\leq d(x_1,x_2)+d(x_2,x_3)$.
\end{itemize}
其中$x_1,x_2,x_3$是$X$的任意元素. 

此时,$(X,d)$或$X$被称为\textbf{度量空间}.
\end{definition}

度量是一个非常直观的概念. 实际上,它就是Euclid空间中“距离”概念的抽象化. 

下面,我们不加证明地给出一些度量的例子,他们的证明见习题\lhysays{出一下}.
\begin{example}
    考虑实数集$\R$,要成为度量空间,可以装备以下度量:
    \begin{itemize}
            \item 平凡的离散度量:$\forall x_1\neq x_2\ d(x_1,x_2)\equiv 1, d(x,x)=0$. 
            \item $d(x_1,x_2)=|x_1-x_2|$. 
        \end{itemize}
        这一例子告诉我们,尽管我们熟悉的绝对值度量是最常见的度量,但实数也可以具备其他度量.

        考虑向量空间$\R^n$,要成为度量空间,可以装备以下度量:
        \begin{itemize}
            \item Minkowski度量($L^p$度量):
            \[d(x_1,x_2)=\left(\sum_{i=1}^n|x_1^i-x_2^i|^p\right)^{1/p}\ (p\geq 1).\] 
            \item Manhattan度量($L^1$度量):
            \[d(x_1,x_2)=\sum_{i=1}^n|x_1^i-x_2^i|.\]
            \item Euclid度量($L^2$度量):
            \[d(x_1,x_2)=\sqrt{\sum_{i=1}^n|x_1^i-x_2^i|^2}.\]
            \item Chebyshev度量($L^\infty$度量):
            \[d(x_1,x_2)=\max_i|x_1^i-x_2^i|=\lim_{p\to\infty}\left(\sum_{i=1}^n|x_1^i-x_2^i|^p\right)^{1/p}.\]
        \end{itemize}
\end{example}

我们的目标是找到一类和实数集非常像的度量空间. 实数集一个非常重要的性质是实数列收敛当且仅当它是Cauchy列. 我们把这一性质抽象出来,就得到了如下定义:

\begin{definition}[Cauchy列,完备度量空间]
    考虑度量空间$(X,d)$的点列$\{x_n\}_{n\in \N}$,如果对于任何$\epsilon>0$,都可以找到序号$N\in\N$,使得对于任何大于$N$的序号$m,n\in\N$,
    \[d(x_m,x_n)<\epsilon,\]
    那么我们称$\{x_n\}$是\textbf{Cauchy列}.

    如果度量空间$(X,d)$的任意Cauchy列$\{x_n\}_{n\in \N}$都收敛,即存在点$a\in X$,使得
    \[\lim_{n\to\infty}d(a,x_n)=0,\]
    那么,我们称度量空间$(X,d)$是\textbf{完备的},
\end{definition}

为了理解Cauchy列的含义,我们先要理解序列的收敛性(也就是极限). 一列实数$a_n$有极限$a$,指的是对任何$\epsilon>0$,都可以找到序号$N\in\N$,使得对于任何大于$N$的序号$n\in\N$,
\[
    |a_n-a|<\epsilon.
\]
更直观一些的说法是,不论给多小的精度,除了有限项,$a_n$都可以以这一精度逼近$a$. 

而Cauchy列描述了另一种形式的收敛性,此时,我们虽然不知道$a_n$离哪个实数比较近,但是我们知道除了有限项,$a_n$相互之间的差异都会小于这个精度. 直观上,这说明$a_n$在靠近某个东西,也就是收敛. 

完备性这一概念就是说,这两个收敛性的定义是等价的,因此$a_n$的确是在靠近某个东西. 我们将它写作定理的形式:

\begin{theorem}\label{thm:complete-metric-space-convergence}
    设$(X,d)$是一个完备度量空间,对任意序列$\{x_n\}_{n\in\N}$,以下两个条件等价:
    \begin{itemize}
        \item $\{x_n\}$是Cauchy列.
        \item $\{x_n\}$收敛.
    \end{itemize}
\end{theorem}
\begin{proof}
    我们只需要证明收敛序列是Cauchy列. 设$\{x_n\}$收敛到$a$,即对任意$\epsilon>0$,存在$N\in\N$,使得对于任意$n>N$,有
    \[d(x_n,a)<\epsilon/2.\]
    于是对于任意$m,n>N$,有
    \[d(x_m,x_n)\leq d(x_m,a)+d(a,x_n)<\epsilon/2+\epsilon/2=\epsilon.\]
    因此$\{x_n\}$是Cauchy列.
\end{proof}

下面,我们不加证明地给出一些完备度量空间的例子,证明见习题\lhysays{出一下}.
\begin{example}
\begin{itemize}
\item $L^p$度量下下$\R^n$是完备的.
\item 使用度量$d(x_1,x_2)=|x_1-x_2|$,则$X=\R\setminus\{0\}$不是完备度量空间. 考虑
\[\left\{x_n=\frac1n\right\}_{n\in\N},\]
它是Cauchy列,但该点列在$X$中没有极限(极限是$0$).

\item $[0,1]$到自身的连续函数空间$C([0,1])$在$L^\infty$度量下是完备的.
此时
\[d(f,g)=\sup_{x\in[0,1]}|f(x)-g(x)|.\]
\end{itemize}
\end{example}

特别注意最后一个例子,我们这里给出了一类抽象的度量空间:它的元素是函数. 这类空间是\emph{泛函分析}中最主要的研究对象,它关注的不再是函数局部的性质,而是整体上研究函数之间的关系.

有了度量的概念,我们就可以研究两个度量空间之间映射的性质:连续性. 

\begin{definition}[连续映射]
设$X$和$Y$是度量空间$(X,d_X),(Y,d_Y)$,考虑映射$f:X\to Y$个点$a\in X$,如果对于任意$\epsilon>0$,存在$\delta>0$,使得对于任意$x\in X$,有
    \[d_X(a,x) < \delta\Rightarrow d_Y(f(a),f(x))<\epsilon,\]
那么我们称$f$在点$a$是\textbf{连续的}.

如果$f$在每个点$x\in X$连续,则称$f$为\textbf{连续映射}.  
\end{definition}

连续映射的定义也是非常直观的,它的意思是,如果$x$和$y$很接近,那么$f(x)$和$f(y)$也应该很接近,说明$f$变化得非常小.

下面我们给出与Banach不动点定理相关的概念:

\begin{definition}[压缩映射]
考虑度量空间$(X,d)$到自身的映射$f:X\to X$. 如果存在$q\in(0,1)$,使得$X$中的任何两个点$x_1,x_2$都成立不等式
    \[d(f(x_1),f(x_2))\leq q\cdot d(x_1,x_2),\]
那么我们称$f$是一个\textbf{压缩映射}.
\end{definition}

压缩映射也是一个非常直观的概念,它的意思是,映射$f$的每次作用都会按照某个比例$q$缩小任意两点之间的距离. 比如,考虑点$x_0$和$f(x_0)$,当压缩次数足够多之后,两点之间的距离就会趋于零,也就是
\[f(\textcolor{Orchid}{\underbrace{f(f(\cdots f(x_0)\cdots))}_{n\text{次}}})\approx \textcolor{Orchid}{\underbrace{f(f(\cdots f(x_0)\cdots))}_{n\text{次}}}.\]
这就是压缩映射具有不动点的原因. 下面我们来严格证明这一点. 

首先,我们说明,证明压缩映射一定是连续映射:
\begin{lemma}\label{lemma:contraction-continuous}
    压缩映射$f:X\to X$是连续映射.
\end{lemma}
\begin{proof}
    对于任意$\epsilon>0$,取$\delta=\epsilon/q$,则对于任意$x_1,x_2\in X$,有
    \[d(x_1,x_2)<\delta\implies d(f(x_1),f(x_2))\leq qd(x_1,x_2)<\epsilon.\]
    因此$f$是连续的.
\end{proof}

接下来,我们说明,度量本身也是一个连续映射:
\begin{lemma}\label{lemma:metric-continuous}
    度量$d:X\times X\to\R$是连续映射.
\end{lemma}
\begin{proof}
    对于任意$x_1,x_2,y_1,y_2\in X$,有
    \[|d(x_1,y_1)-d(x_2,y_2)|\leq d(x_1,y_1)+d(x_2,y_2)\leq 2\max\{d(x_1,x_2),d(y_1,y_2)\}.\]
    因此,对于任意$\epsilon>0$,取$\delta=\epsilon/2$,则对于任意$x_1,x_2,y_1,y_2\in X$,有
    \[d(x_1,x_2)<\delta,d(y_1,y_2)<\delta\implies |d(x_1,y_1)-d(x_2,y_2)|<\epsilon.\]
    因此$d$是连续映射.
\end{proof}

接下来,我们证明压缩映射一定有不动点,这就是Banach不动点定理:
\begin{theorem}[Banach不动点定理,压缩映像原理]\label{thm:banach-fixed-point}

完备度量空间$(X,d)$到自身的压缩映射$f:X\to X$具有唯一的不动点$a$. 

此外,对于任何点$x_0\in X$,迭代序列$x_0,x_1=f(x_0),\cdots,x_{n+1}=f(x_n),\cdots$收敛到$a$. 收敛速度由以下估计给出:
\[d(a,x_n)\leq \frac{q^n}{1-q}d(x_1,x_0).\]
\end{theorem}

\begin{proof}
首先证明存在性. $d(x_{n+1},x_n)\leq qd(x_n,x_{n-1})\leq \cdots\leq q^nd(x_1,x_0).$
从而
\begin{align*}
d(x_{n+k},x_n)&\leq d(x_n,x_{n+1})+\cdots+d(x_{n+k-1},x_{n+k})\\
&\leq(q^n+\dots+q^{n+k-1})d(x_1,x_0)\leq \frac{q^{n}}{1-q}d(x_1,x_0).
\end{align*}
这一不等式对任意$k$都成立,而因此$\{x_n\}$是Cauchy列,根据完备性的定义存在极限
\[\lim_{n\to\infty} x_n=a\in X.\]
结合压缩映射的连续性,有
\[a=\lim_{n\to\infty}x_{n+1} = \lim_{n\to\infty}f(x_n)=f\left(\lim_{n\to\infty}x_n\right)=f(a).\]

然后证明唯一性. 若$f$还有其他不动点$a_1,a_2$,则
\[0\leq d(a_1,a_2)=d(f(a_1),f(a_2))\leq qd(a_1,a_2).\]
而这当且仅当$d(a_1,a_2)=0$,即$a_1=a_2$时才可能成立.

最后证明收敛速度. 对
\[d(x_{n+k},x_n)\leq \frac{q^n}{1-q}d(x_1,x_0),\]
取$k\to\infty$,根据$d$的连续性,有
\[d(a,x_n)\leq \frac{q^n}{1-q}d(x_1,x_0).\]
\end{proof}

在进入应用之前,我们指出压缩映射在\emph{算子法}中的表述,这一部分的系统讨论需要线性代数的知识,请参阅\Cref{chap:linear-algebra}. 我们这里只做一个简单介绍. 

首先,如果我们把压缩映射$f$看成一个算子$\mathcal A$,即把$X$中的元素变换到$X$中的元素,那么我们可以定义这一算子的\emph{范数}:

\begin{definition}[算子范数]
    设$X=\R^n$,对于算子$\mathcal A:X\to X$,它的\textbf{范数}定义为
    \[\norm{\mathcal A}=\sup_{x\neq 0}\frac{\norm{\mathcal A x}}{\norm{x}}.\]
    其中$\norm{\cdot}$是$X$上的$L^2$范数,即
    \[\norm{x}=\sqrt{\sum_{i=1}^n|x_i|^2}.\]
\end{definition}

在这一概念下,我们可以改写压缩映射的定义. 对任意$x,y\in \R^n$,有
\[\norm{\mathcal A x-\mathcal A y}\leq q\norm{x-y}\implies \frac{\norm{\mathcal A(x-y)}}{\norm{x-y}}\leq q.\]
根据$x,y$的任意性,这其实就是说
\[\norm{\mathcal A}\leq q<1.\]
所以,压缩映射其实就是算子范数小于$1$的算子. 

反之,如果一个算子$\mathcal A$的范数$q$小于$1$,那么对任意$x,y\in \R^n$,有
\[\norm{\mathcal A x-\mathcal A y}\leq \norm{\mathcal A}\norm{x-y}\leq q\norm{x-y}.\]
因此,$\mathcal A$是一个压缩映射. 我们将这一讨论总结如下:

\begin{theorem}\label{thm:operator-fixed-point}
    设$X=\R^n$,对于算子$\mathcal A:X\to X$,以下两个条件等价:
    \begin{itemize}
        \item $\mathcal A$是压缩映射.
        \item $\norm{\mathcal A}<1$.
    \end{itemize}
\end{theorem}

对于很多算子,直接验证压缩映射的定义比较困难,而验证算子范数小于$1$则相对容易. 因此这是一个特别实用的表述方式. 

\begin{example}[落在地面上的地图]
将一座公园的地图铺开在公园地面上,则地面上恰有唯一一点与地图上对应的点重合. 

设公园可以用有界的面闭区域$\Omega$表示. 设地图的压缩比是$\lambda\in(0,1)$. 现在固定一个平面直角坐标系,把地图铺在区域$\Omega$内,则从$\Omega$内的点$x$(公园中的地点)到地图上对应点$x'$的变换由下面的公式给出:
\[x' = f(x) := \lambda Rx + b.\]
其中$R$和$b$分别为旋转和平移变换. 

根据旋转的定义,容易看出$\norm{Rx}=\norm{x}$,因此
\[\norm{\lambda R}=\sup_{\norm{x}=1}\norm{\lambda Rx}=\lambda<1,\]
所以对任意$x,y\in\Omega$,有
\[\norm{f(x)-f(y)}=\norm{\lambda Rx-\lambda Ry}=\lambda\norm{Rx-Ry}= \lambda\norm{x-y}.\]
因此$f$是一个压缩映射.

由Banach不动点定理可知,压缩映射$f(x)$有唯一不动点$a=f(a)$.
\end{example}

\begin{example}[梯度下降的收敛性]\label{ex:gradient-descent}
这个例子研究如何利用算子法证明梯度下降的收敛性. 它需要较多微积分和线性代数的知识,请参阅\Cref{chap:calculus} 和\Cref{chap:linear-algebra}. 不过,理解整个思路并不需要这些知识.

我们优化目标是寻找二阶可微凸函数$f(x),x\in \R^n$的最小值. 使用梯度下降方法:
\[x_{k+1}=x_k-\alpha_k f'(x_k),\]
其中$\alpha_k$是第$k$步的\textbf{步长},在这个例子中,我们假设$\alpha_k=\alpha$是一个常数.


接下来,我们给出对$f$的假设:存在常数$L>0$,对任意$x\in \R^n$,
    \[\lambda_{\min}(\nabla^2 f(x))\geq L,\]
其中
    \begin{itemize}
        \item $\nabla^2f(x)$是$f$的Hessian矩阵(二次导数),
        \item $\lambda_{\min}(A)$表示矩阵$A$的最小特征值.
    \end{itemize}

我们要证明:对于足够小的$\alpha$,梯度下降能收敛到最小值点,且具有指数收敛速度.

先看一下证明的思路. 定义梯度下降算子:
    \[\mathcal T^{(\alpha)}: x\mapsto x - \alpha\nabla f(x).\]
我们要设法证明梯度下降算法是完备度量空间中的一个压缩映射. 
\begin{enumerate}
    \item 首先,根据\Cref{thm:differential-convex-equivalence},可微凸函数$f$的最小值点充分必要地满足
    \[\nabla f(x)=0.\]
    \item 其次,显然有
    \[\nabla f(x^*)=0 \iff \mathcal T^{(\alpha)}x^*=x^*.\]
    因而最小值点是梯度下降算子的不动点.
    \item 所以,我们只需要说明$\mathcal T^{(\alpha)}$是一个完备度量空间的压缩映射,就可以用Banach不动点定理证明梯度下降的收敛性.
\end{enumerate}

我们只需要证明$\mathcal T^{(\alpha)}$是压缩映射,并给出压缩系数. 由有限增量原理(\Cref{thm:lagrange-finite-multi}):
\[\norm{\mathcal T^{(\alpha)}x - \mathcal T^{(\alpha)}y} \leq \sup_{z\in(x,y)}\norm{I - \alpha \nabla^2 f(z)}_2\cdot\norm{x-y}_2.\]

注意到$\norm{I-\alpha\nabla^2f(z)}_2$等于$I-\alpha\nabla^2f(z)$特征值的最大模,根据条件可知特征值的最大模 $\leq 1-L\alpha$. 因此,只要$\alpha<L^{-1}$,$\mathcal T^{(\alpha)}$就是一个压缩映射.
\end{example}

\section{Brouwer不动点定理}

下面我们考虑另一类不动点定理. 在Banach不动点定理中,我们对映射的性质做出了限制. 在这一部分,我们只要求映射是连续的,但是对映射所在的集合做出了限制. 因此,我们下面不加解释地给出几个技术性的概念,更系统的讨论请参阅\Cref{chap:calculus}.

\begin{definition}[开集、闭集和紧集]
    考虑度量空间$(X,d)$,定义$a\in X$的邻域为
    \[B(a,\delta):=\{x\in X|d(a,x)<\delta\}.\]
    考虑一个集合$K\subseteq X$,
\begin{itemize}
    \item 如果对任意$x\in G$,都存在邻域$B(x,\delta)\subseteq G$,那么$G$是\textbf{开集}.
    \item 如果$X\setminus G$是开集,那么$G$是\textbf{闭集}.
    \item 如果对任何开集族$\{G_\alpha\}$,只要满足
    \[K\subseteq \bigcup_\alpha G_\alpha,\]
    就存在$G_{\alpha_1},\cdots,G_{\alpha_n}$使得
    \[K\subseteq G_{\alpha_1}\cup\cdots\cup G_{\alpha_n},\]
    那么$K$是\textbf{紧集}. 换言之,如果任何可以覆盖$K$的开集族都有一个有限子族可以覆盖$K$,那么$K$是紧集.
\end{itemize}
\end{definition}

在Euclid空间中,我们有如下性质:

\begin{theorem}\label{thm:compact-set-iff-bounded-closed}
    考虑集合$K\subseteq\R^n$,以下两个定义等价:
    \begin{itemize}
        \item $K$是紧集.
        \item $K$是有界闭集.
    \end{itemize}
\end{theorem}
这里,有界的意思就是,存在一个半径$R$,使得$K$包含在半径为$R$的球内. 

注意,\Cref{thm:compact-set-iff-bounded-closed} 只在$\R^n$中成立,对于一般的度量空间,紧集和有界闭集不一定等价(见习题\lhysays{出一下}).

有了上面的准备,我们就可以叙述Brouwer不动点定理了:

\begin{theorem}[Brouwer不动点定理]
    设$M\subseteq\R^n$是一个非空紧凸集,而$F:M\to M$是一个连续函数. 则存在$x\in M$使得$F(x)=x$成立.
\end{theorem}

Brouwer不动点定理可以通过该实际的例子来理解:将一张白纸平铺在桌面上,再将它揉成一团(不撕裂),放在原来白纸所在的地方,那么只要它不超出原来白纸平铺时的边界,那么白纸上一定有一点在水平方向上没有移动过. 这个断言依据Brouwer不动点定理在$\R^2$的情况,因为把纸揉皱是一个连续的变换过程.

另一个例子:大商场等地方可以看到的平面地图,上面标有“您在此处”的红点. 如果标注足够精确,那么这个点就是把实际地形映射到地图的连续函数的不动点.

下面我们看一个Brouwer不动点定理的应用例子,这一例子需要线性代数和Markov链的知识,请参阅\Cref{chap:linear-algebra} 和\Cref{chap:markov-chain}.

首先引入矩阵\textbf{不可约}的概念:

\begin{definition}[不可约矩阵]
    考虑方阵$A$,定义操作$O_{ij}$:
    \begin{itemize}
        \item 将$A$的第$i$列和第$j$列交换,
        \item 同时将$A$的第$i$行和第$j$行交换.
    \end{itemize}

    如果经过有限次操作$O_{ij}$(不同的$i,j$)后,$A$变成分块上三角矩阵,那么$A$是\textbf{可约的};否则,$A$是\textbf{不可约的}.
\end{definition}

下面我们来解释不可约矩阵在Markov链中的含义. 设$A$是某个Markov链的转移矩阵,假如$A$可约,通过行列交换的方法变成了分块上三角矩阵:
\[\begin{pmatrix}
    A_{11} & A_{12}\\
    O & A_{22}
\end{pmatrix},\]
设前半对应的状态集是$S_1$,后半对应的状态集是$S_2$,那么,这一转移矩阵的形式意味着,从$S_2$的任意状态出发,达到$S_1$的任意状态的概率都是$0$. 因此,这个Markov链的流动性是比较差的. 

反之,如果$A$是不可约的,那么,不论从哪个状态出发,经过有限次转移,都可以到达任何一个状态. 所以,这一Markov链的流动性是比较好的.

接下来,我们说明,如果Markov链不可约(也就是流动性很好),它会有一个平稳遍历分布(即所有状态都是正概率). 这个结论由以下定理给出:

\begin{theorem}[Perron-Frobenius定理]\label{thm:perron-frobenius}
    设$A=(a_{ij})$为$n\times n$不可约实矩阵,所有元素均非负,$a_{ij}\geq 0$,则下列结论成立.
    \begin{itemize}
        \item 存在一个实特征值$r$,其他(左右)特征值$\lambda$的模均不超过$r$,即$|\lambda|\leq r$.
        \item 存在一个与$r$对应的左特征向量和右特征向量,其所有元素恒正.
        \item $\min_i\sum_{j}a_{ij}\leq r\leq \max_i\sum_j a_{ij}$.
    \end{itemize}
\end{theorem}

在开始证明之前,我们先说明它如何导出Markov链的性质. 

\begin{corollary}
    不可约有限状态Markov链必然存在平稳遍历分布. 换言之,如果$P$是一个不可约有限状态Markov链的转移矩阵,那么存在一个分布$\pi$,使得$\pi=\pi P$并且对任意$i$都有$\pi_{i}>0$.
\end{corollary}
\begin{proof}
    根据定义,$P$是非负实不可约方阵. 由Perron-Frobenius定理,$P$存在一个特征值$r$使得
    \[1=\min_i\sum_jP_{ij}\leq r\le\max_i\sum_j P_{ij}=1,\]
    即$r=1$,并且,它对应一个正的左特征向量
    \[\pi_0\in \left\{x\in\R^n|x\geq 0,\sum_ix_i=1\right\}.\]
    因此,
    \[\pi_0 P = \pi_0.\]
    即$\pi_0$是平稳遍历分布. 
\end{proof}

接下来,我们证明\Cref{thm:perron-frobenius}.
\begin{proof}[\Cref{thm:perron-frobenius} 的证明]
首先证明$A$存在一个正的特征值$r>0$. 考虑单纯形
\[S:=\left\{x\in\R^n|x\geq 0,\sum_i x_i=1\right\}.\]
任意$x\in S$,有$Ax\geq 0$. 

我们断言$Ax>0$. 若不然,$A$存在某一列全$0$(由$x\geq 0$和$A$非负可得). 此时可将该$0$列交换到第一列,对应的行也交换,得到的矩阵为分块上三角,与不可约性矛盾.

可以在$S$上定义映射
\[T(x) = \frac1{\rho(x)}Ax,\]
其中$\rho(x) > 0$使得$T(x)\in S$. 具体来说,
\[\rho(x) = \sum_i (Ax)_i = \sum_{i,j} a_{ij}x_j.\]

显然$T(x)$是$S\to S$的连续映射. $S$是一个有界凸闭集. 由Brouwer不动点定理,存在$x_0\in S$使得
\[x_0 = T(x_0)=\frac1{\rho(x_0)}Ax_0.\]

令$r=\rho(x_0)$,则可得$r$为$A$的一个正的特征值.

我们接下来证明,与$r$对应的右特征向量所有元素恒正. 由之前的证明,与$r$对应的特征向量$x_0\in S$,则$x_0\geq 0$. 我们证明$x_0>0$.

我们将$A$的行列进行交换,使得$Ax_0$非零的元素在上方. 具体来说,设$A = PBP^{-1}$,其中$P$是置换矩阵,则
\[PBP^{-1}x_0=rx_0\implies B(P^{-1}x_0)=r(P^{-1}x_0).\]

记$\tilde{x}_0 = P^{-1}x_0$.取$B$使得$\tilde{x}_0 = (\xi,0)^\t, \xi > 0$. 则
\begin{align*}
    \left(                 %左括号
        \begin{array}{cc}   
        B_{11} & B_{12}\\  
        B_{21} & B_{22}\\  
        \end{array}
    \right)
    \left(                 
        \begin{array}{c}   
        \xi\\  %第一行元素
        0\\
        \end{array}
    \right)=
    \left(                 
        \begin{array}{c}   
        r\xi\\  %第一行元素
        0\\
        \end{array}
    \right).
\end{align*}
此时$B_{21}\xi=0$,由$\xi>0$可得$B_{21}=0$. 这与不可约矛盾,因此$x_0 > 0$.

以上过程可以对左特征值$r_1$和对应的左特征向量$x_1$重复,得到$r_1>0$且$x_1>0$.

然后我们证明:若$\lambda$是$A$的任意右特征值,有$|\lambda|\leq r$. 

设$0\leq B\leq A$,也就是$0\leq B_{ij}\leq A_{ij}$,则$B$的特征值$\beta$和对应的特征向量$y$满足
\[|\beta|\leq r,\quad By=\beta y.\]
记$y^\star = |y|=(|y_i|)_i$. 于是有
\[|\beta|y^\star = |\beta y| = |By| \leq By^\star\leq Ay^\star.\]
左乘$x_1^\t$,有
\[|\beta|x_1^\t y^\star \leq x_1^\t Ay^\star = r_1x_1^\t y^\star.\]
由$x_1^\t y^\star>0$可得$|\beta|\leq r_1$. 

令$B=A$可得$|\lambda|\leq r_1$,特别地$r\leq r_1$. 

如果$\lambda$是左特征值,用同样的证明可以得到$|\lambda|\leq r$,特别地$r_1\leq r$. 

综合以上两点,$r=r_1$,于是我们说明了$x_0$和$x_1$是与$r$对应的左右特征向量,并且其他左右特征值的模都不超过$r$. 

最后证明:
\[\min_i\sum_j a_{ij}\leq r\leq \max_i\sum_j a_{ij}.\]

以这样的方式获得$\tilde A$: 将$A$的每一行都扩增(不减小某个元素),使得每一行都达到$\max_i\sum_j a_{ij}$. 此时$\max_i\sum_j a_{ij}$成为$A$的一个正特征值,且有右特征向量
\[\tilde{x}_0=\frac 1n\cdot\mathbf{1}\in S.\] 

由之前的证明,根据$0\leq A\leq \tilde A$,可以得到$\tilde A$的正特征值$\tilde r\geq r$. 因此
\[r \leq \max_i\sum_j a_{ij}.\]
同理缩小$A$可得
\[\min_i\sum_j a_{ij}\leq r.\]
\end{proof}


\section{习题}

\lhysays{TODO}

\section{章末注记}

\lhysays{TODO}