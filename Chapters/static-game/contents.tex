\chapter{静态博弈}\label{chap:static-game}
本章我们讨论静态博弈的基本概念和分析方法,并以此为基础,讨论博弈中认知相关的问题。

\section{正则形式博弈}
动态博弈通常被建模为\emph{\index{博弈!扩展形式~}扩展形式博弈}. 与之相对的是\emph{\index{博弈!正则形式~}正则形式博弈},即玩家只有一次行动的机会,所有玩家同时操作. 正则博弈通常要求信息是完全的. 这种博弈的过程与时间无关,属于\emph{\index{博弈!静态~}静态博弈}.


一个正则形式博弈有如下构成要素
\begin{itemize}
    \item 玩家集合:$I$,我们总是假设这是一个有限集合.
    \item 玩家的行动集(纯策略集):$A_i$,$i\in I$.
    \item 玩家的收益:$u_i:\prod_j A_j\to\R$.
    \item 完全信息:以上内容是所有玩家的共同知识.
\end{itemize}
所有人的策略拼在一起,即$s=(s_i)_{i\in I}$,构成博弈的\emph{\index{策略组合}策略组合}。有以下特殊的正则博弈:
\begin{itemize}
    \item 当$A_i$有限,我们称之为\emph{\index{博弈!矩阵~}矩阵博弈}.
    \item 当$A_i$和$u_i$都是连续的,我们称之为\emph{\index{博弈!连续~}连续博弈}.
    \item 当$\sum_i u_i=0$,我们称之为\emph{\index{博弈!零和~}零和博弈},
    当所有策略组合,收益和都是常数时,解概念的分析可以保持一致,我们也可以按零和处理. 
\end{itemize}

如何定义正则博弈的均衡?首先要明确均衡的概念。假设所有人之间是不能交流的,每个人独立做决策. 因此玩家之间不能协调彼此的决策. 因为只能行动一次,所以所谓\emph{均衡},指的是没有人对自己的决策感到后悔的状态,没有人可以通过改变自己现在的策略来获得更多的收益. 因此我们有如下定义:

\begin{definition}[Nash均衡]
(纯策略)\textbf{Nash均衡}指的是策略组合$s$,满足
    \[\forall i\in I\,\forall a_i\in A_i:u_i(s_i,s_{-i})\geq u_i(a_i,s_{-i}).\]
\end{definition}

我们也可以用不动点来理解Nash均衡。首先定义\emph{最优反应}\index{最优反应}:给定对手的策略$s_{-i}$,玩家$i$选择的最大化自己收益的策略$s_i$. Nash均衡的等价定义是每个人都达到了自己的最优反应,即最优反应的不动点.


\begin{example}[囚徒困境]\index{囚徒困境}
考虑一个经典的非合作博弈,\emph{囚徒困境}. 一共有两个玩家,行玩家和列玩家. 玩家的第一个选择是保持沉默,第二个选择是认罪并检举对方. 它有如下收益矩阵:
\[
\begin{pmatrix}
-1,-1&-10,0\\
0,-10&-5,-5\\
\end{pmatrix}.
\]
矩阵每一项第一个元素是行玩家的收益,第二个是列玩家的收益. 这个博弈有唯一的Nash均衡:每个人都认罪. 思考:打破Nash均衡的假设,有没有可能得到更好的结果?
\end{example}

然而,纯策略Nash均衡并不一定存在. 考虑如下的输赢(零和)博弈:猜硬币游戏\index{猜硬币游戏}. 行列玩家分别有一枚硬币,他们秘密地抛掷. 如果两个玩家的硬币上面相同,行玩家获胜;否则列玩家获胜. 收益矩阵为:
    \[
    \begin{pmatrix}
    1,0&0,1\\
    0,1&1,0\\
    \end{pmatrix}.
    \]
容易验证,这个博弈没有纯策略Nash均衡. 更一般地,二人正则输赢博弈中纯策略Nash均衡往往不存在. 我们有如下定理:
\begin{theorem}
设$G=(I,\{A_i\}_{i\in I}, \{u_i\}_{i\in I})$是一个二人正则输赢博弈,其中$I=\{1,2\}$. 那么,$G$存在纯策略Nash均衡当且仅当其中一个玩家存在必胜策略. % HW:证明这个定理的矩阵博弈形式
\end{theorem}
对比动态博弈中的Zermelo定理,静态的二人完全信息输赢博弈已经不能够保证必胜策略的存在性。因此,静态输赢博弈的结局往往比动态输赢博弈更加不确定. 我们可以利用这一事实去理解生成对抗网络模型的不稳定性.


\subsection{生成对抗网络}

\emph{\index{生成对抗网络}生成对抗网络}(GAN)\index{GAN} 有两个子模型组成,一个被称为\emph{\index{生成模型}生成模型},一个被称为\emph{\index{判别模型}判别模型}. 生成模型的任务是生成看似真实的数据,二判别模型的任务是识别给定的数据是真实的还是伪造的.

假设真实数据的分布为$F_{data}$. 生成模型为$G(x;\theta_g)$,参数为$\theta_g$,输入向量$x$,输出数据向量$z$. 当$x$服从分布$F_x$,$G$的输出会形成一个分布$F_g$. 判别模型为$D(z;\theta_d)$,参数为$\theta_d$,接受一个数据向量$z$,输出一个$[0,1]$中的实数,表示$z$来自分布$F_{data}$的概率. 我们假设$F_{data}$和$F_x$都是连续型分布,有密度函数$p_{data}$和$p_x$. 我们再假设$D$和$G$都是连续的.

将$G$和$D$看成两个玩家,于是GAN可以被看成一个二人零和博弈,收益函数为:
    \[
        V(G,D)=\E_{z\sim F_{data}}(\log D(z))+\E_{x\sim F_x}(\log(1-D(G(x)))).
    \]
$D$最大化$V$,$G$最小化$V$.

从博弈论角度出发,一个基本的问题是Nash均衡是否存在?假设$D$和$G$都可以任意选择连续函数. 我们将展示一种通用的方式求解连续博弈的Nash均衡. 注意到$G(x)$形成了一个连续分布,密度记为$p_g$.  首先证明密度函数存在性定理:
\begin{theorem}
设$X\sim \mathcal U(0,1)$. 对于任意密度函数$p$,存在一个连续函数$F$使得$F(X)$具有密度$p$.
\end{theorem}
\begin{proof}
设$F_p$是$p$对应的分布函数,它是一个单调的连续函数. 取$F(x)=\inf\{y\in \R: F_p(y)\geq x\}$即可.
\end{proof}
因此,$G$的行动等价于选择$p_g$.

给定$G$的选择$p_g$,我们来求$D$的最优反应$D^*$.
    \[V(G,D)=\int (p_{data}(x)\log D(x)+p_g(x)\log(1-D(x)))\d x.\]
函数$a\log x+b\log(1-x)$最大值在$x=a/(a+b)$的时候取得. 因此,
    \[D^*(x)=\frac{p_{data}(x)}{p_{data}(x)+p_g(x)}.\]
    
现在,给定最优反应$D^*=p_{data}(x)/(p_{data}(x)+p_g(x))$,我们来求$G$的最优反应. 直观上,$G$能做到的最好选择就是$p_g=p_{data}$. 此时,$D^*(x)=1/2$,因此对任意$G$,$V(G,D^*)=-\log 4$. $G$选任何策略都是一样的收益,因此这是一个Nash均衡. 我们证明了:
\begin{theorem}[GAN的Nash均衡存在性]
在GAN的博弈中,$G$选择$p_{data}$,$D$选择$1/2$是一个Nash均衡.
\end{theorem}

我们刚刚的分析过于理想化,需要考虑一些问题。首先,神经网络的大小是有限的,因此$G$不能选择任何$p_g$。因此,我们刚刚找到的Nash均衡可能不存在。其次,$p_{data}$是一个未知的量,我们只有一些样本。因此,$G$和$D$都需要一个算法来找到它们的最优策略。这就是训练GAN的过程。

我们接下来给出一种更符合实际的均衡概念。

\emph{\index{局部Nash均衡}局部Nash均衡}$(G^*, D^*)$是指在$G^*$和$D^*$的一个邻域内$(G^*, D^*)$形成了一个Nash均衡。\emph{\index{稳定局部Nash均衡}稳定局部Nash均衡}$(G^*, D^*)$是指$(G^*, D^*)$是一个局部Nash均衡,并且在$(G^*,D^*)$的一个邻域内,对任意$(G,D)$都有$V(G,D^*)\geq V(G,D)$和$V(G^*,D)\leq V(G,D)$。GAN的训练实际上就是在寻找稳定局部Nash均衡的过程。

稳定局部Nash均衡表明了,即便对手的策略具有(很小的)不确定性,玩家的策略依然是最优反应。在训练过程中,这样的不确定性很可能出现,源自精度或者误差. 因此,稳定局部Nash均衡是一个更有可能被找到的解,不稳定局部Nash均衡则很容易偏离. 然而,我们刚刚在理想条件下找到的Nash均衡其实也是不稳定的. 实际上,GAN的训练是一个非常不稳定的过程. 我们有如下结果:

\begin{theorem}
设GAN博弈的收益函数$V$是解析的,$(0,0)$是稳定局部Nash均衡,在$(0,0)$的一个邻域内,$V(G,D)=C+V^2f(V)+D^2g(D)+V^2D^2h(G,D)$,其中$f,g,h$都是解析函数,满足$f(0),g(0)\geq0 $,$C$是常数.
\end{theorem}

$V$要具备这种形式才可能有稳定局部Nash均衡. 然而一般的神经网络并不能具备这样的形式,所以很多情况下根本不存在稳定局部Nash均衡!

\subsection{混合策略}
我们已经看到,在相当普遍的情况下,纯策略Nash均衡并不存在. 所以我们需要允许玩家进行随机行动,这就是\textbf{混合策略}\index{混合策略}。混合策略就是建立在纯策略空间$S$上的一个概率分布. 混合策略空间记为$\Delta(S)$. 当$S$有$n$个元素(有限),$\Delta(S)$可以被表示为标准的$n$-单纯形:
    \[\Delta(S)=\left\{x\in\R^n:\sum_{i=1}^n x_i=1,x_j\geq 0,\forall j\right\}.\]

那么,有了混合策略,玩家的决策思考过程是怎么样的?一个非常标准的回答是\emph{期望效用理论}\index{期望效用理论},它由Von Neumann和Morgenstern提出. 该理论认为,在面对不确定性时,人按照期望效用进行决策. 因此,我们需要计算玩家的期望效用。为此,引入混合\emph{策略组合}\index{策略组合}:$\sigma=(\sigma_i)_{i\in I}$,其中$\sigma_i\in\Delta(A_i)$. $\sigma$是一个$(A_i)_{i\in I}$上的概率分布,每一维相互独立. 当所有玩家选定策略之后,玩家$i$的期望收益是:
    \[u_i(\sigma)=\E_{a\sim\sigma} u_i(a).\]


\begin{definition}[Nash均衡]\index{Nash均衡}
对于一个博弈$G=(I,\{A_i\}_{i\in I},\{u_i\}_{i\in I})$,混合策略Nash均衡$\sigma$满足对于任意玩家$i$和任意$\sigma_i'\in \Delta(A_i)$,都有
\[u_i(\sigma_i,\sigma_{-i})\geq u_i(\sigma_i',\sigma_{-i}).\]
\end{definition}
Nash著名的定理是:
\begin{theorem}[Nash均衡存在性定理]\index{Nash均衡存在性定理}
对于任意有限正则形式博弈,都存在一个混合策略Nash均衡.
\end{theorem}

我们来看一个例子。
\begin{example}
继续考虑猜硬币游戏\index{猜硬币游戏},收益矩阵为
\[
\begin{pmatrix}
1,0&0,1\\
0,1&1,0\\
\end{pmatrix}.
\]
容易证明,唯一的均衡是两个玩家都选择$(1/2,1/2)$.
\end{example}

尽管在数学上,混合策略是导出了漂亮的结果,但是混合策略并不是一个非常合理的概念. 如何理解混合策略?我们将在后面通过似然、知识论等方式来解释混合策略.


\section{不完全信息博弈(Bayes博弈)}

即便是纯策略Nash均衡也可能是不合理的状态. 考虑如下的二人博弈:
    \[\begin{pmatrix}
    1,1&0,0\\
    0,0&0,0
    \end{pmatrix}.\]
显然,两个人玩家都选择第二策略达到了Nash均衡. 然而,当行玩家对列玩家的选择有任意小的不确定性时,他都更倾向于选择第一个策略. 因此,我们给出的这个Nash均衡实际上描述了一种不太可能出现的状态. 这促使我们提出了所谓的\emph{颤抖的手完美化}\index{颤抖的手完美化}:$s$是一个纯策略Nash均衡,并且当对手玩家的策略有任何微小不确定性的时候,$s$中的策略依然是最优反应.

“颤抖的手”给了我们一个例子说明不确定性会影响玩家的决策. 那么如何量化不确定性?经济学的解决方案是\emph{Bayes解释}\index{Bayes解释}的概率论:每一个玩家对世界有一个先验的\emph{信念}\index{信念},信念在数学上被建模为对可能世界的概率分布.

利用这样的建模,我们可以给不完全信息博弈一个正式定义。一个不完全信息博弈有如下组成部分:
\begin{itemize}
\item 玩家集合:$I$.
\item 行动空间:$A=(A_i)_{i\in I}$,$A_i$表示玩家$A_i$的所有可能行动.
\item 类型空间:$\Theta=(\Theta_i)_{i\in I}$,$\Theta_i$表示玩家$i$的所有可能类型.
\item 收益函数:$u_i:A\times\Theta\to\R$,当所有人的行动和类型都确定的时候,玩家$i$能拿到的收益.
\end{itemize}
所有玩家的行动$a=(a_i)_{i\in I}$形成了一个行动组合. 所有玩家的类型$\theta=(\theta_i)_{i\in I}$形成了一个类型组合.

 $P_i\in\Delta(\Theta_i)$是玩家$i$类型的概率分布. $P_i$表示了其他玩家对玩家$i$类型的信念. 我们假设$P_i$是相互独立的,因此玩家$i$对其他玩家的信念是$P_{-i}=\prod_{j\neq i}P_j$. 玩家$i$知道自己的类型.

在使用这一定义的时候需要非常小心,在一般情况下,玩家$i$对这个世界的信念应该包含:
\begin{itemize}
    \item 其他玩家有谁;
    \item 自己和对手可能的行动;
    \item 自己的类型;
    \item 自己的收益函数;
    \item ……
\end{itemize}
然而,在上述标准的经济学模型中,我们做了如下严格的限制:
\begin{itemize}
    \item 玩家、可能行动、可能类型、收益函数是所有人的共同知识,没有人对这些东西有不一样的信念. 
    \item 玩家对世界的不确定性仅仅在于其他玩家的类型,而且所有人关于每个玩家类型的信念是一致且独立的.
    \item 自己的类型自己知道并且只有自己知道.
\end{itemize} 

下面我们看一个例子


\begin{example}[合作者]
考虑一个二人博弈,称为“工作-偷懒”博弈\index{博弈!工作-偷懒~}。两个人的行动都是“工作”($W$)或“偷懒”($S$)。行玩家的类型集合是单点集,列玩家的类型是“勤奋”($D$)或“懒惰”($L$)。收益矩阵为
\[\begin{array}{c|cc}
    \multicolumn{3}{l}{\theta_2=D,}\\
     &W&S  \\\hline
     W&3,3&-1,0\\
     S&2,1&0,0\\
\end{array}\qquad \begin{array}{c|cc}
    \multicolumn{3}{l}{\theta_2=L,}\\
     &W&S  \\\hline
     W&1,1&-1,2\\
     S&2,-1&0,0\\
\end{array}\]
\end{example}

在具有不确定性的世界中,玩家的策略如何定义?玩家如何决策?因为玩家知道自己的类型,但在决策的时候不能知道其他人的类型,所以一个完整的(纯)策略应该是$s_i:\Theta_i\to A_i$,即在给定自己的类型时,应该采取的行动.

关于收益,我们依然沿用期望效用理论. 当玩家$i$具有类型$\theta_i$,采取行动$a_i$,对手的策略是$s_{-i}$时,$i$的\emph{中期}期望收益为\index{中期收益}:
    \[\tilde u_i(a_i,\theta_i,s_{-i})=\E_{\theta_{-i}\sim P_{-i}}[u_i(a_i,s_{-i}(\theta_{-i}),\theta_i,\theta_{-i})].\]
利用期望效用理论,我们很容易定义均衡的概念:
\begin{definition}[Bayesian Nash均衡,BNE]\index{Bayesian Nash均衡}\index{BNE}
$s=(s_i)_{i\in I}$被称为\emph{Bayesian Nash均衡},如果
    \[\tilde u_i(s(\theta_i),\theta_i,s_{-i})\geq \tilde u_i(a_i,\theta_i,s_{-i})\]
对任意$i,\theta_i,a_i$都成立.
\end{definition}

我们也可以考虑\emph{前期}期望收益\index{前期收益},此时玩家$i$并不知道自己是什么类型,因此他也要对自己的类型求期望:
     \[\hat u_i(s_i,s_{-i})=\E_{\theta\sim P}[u_i(s_i(\theta_i),s_{-i}(\theta_{-i}),\theta_i,\theta_{-i})].\]
根据前期期望收益,我们也可以定义BNE为:
     \[\hat u_i(s_i,s_{-i})\geq u_i(s'_i,s_{-i})\]
对任意$i$和任意策略$s_i'$成立.

两个定义是等价的. 首先,前期期望收益是中期期望收益的加权平均. 然后,最大化前期期望收益等价于最大化平均中的每一项中期期望收益,也就是最大化中期收益. 所以这两者是等价的.

当所有的不确定性都消失的时候,我们得到的收益是真实的,被称为\emph{后期}收益\index{后期收益}. 前期、中期、后期分别表明了信息的确定程度.

\begin{remark}
    自然,我们也可以定义混合策略的BNE,此时策略$s_i$是一个$\Theta_i$到$\Delta(A_i)$的映射.
\end{remark}


\begin{example}[猜硬币游戏的BNE]\index{猜硬币游戏}
考虑猜硬币游戏:
\[
\begin{array}{c|cc}
    &H&T  \\\hline
    H&1,-1 &-1,1\\
    T&-1,1&1,-1\\
\end{array}
\]
如果两个人都出$H$的时候收益有微小的扰动,我们就得到了一个Bayes 博弈:
\[
\begin{array}{c|cc}
    &H&T  \\\hline
    H&1+\epsilon\theta_1,-1+\epsilon\theta_2 &-1,1\\
    T&-1,1&1,-1\\
\end{array}
\]
其中$\theta_i\sim\mathcal U[-1,1]$.

考虑策略:$s_i:[-1,1]\to\{H,T\}$满足
\[s_i(\theta_i)=\begin{cases}
H,&\theta_i\in[0,1],\\
T,&\theta_i\in[-1,0).
\end{cases}\]
容易证明,$(s_1,s_2)$是一个BNE. 
\end{example}

注意到,在上面的例子中,策略$(s_1,s_2)$导致的结果实际上是,每个玩家以等概率选择$H$和$T$. 当$\epsilon\to 0$,这个博弈收益矩阵回到了原始博弈. BNE形成的行动概率分布则趋于原始博弈的混合策略. 通过这样的办法,正则博弈的混合策略均衡被理解为:当不确定性趋于消失时候,BNE形成的行动概率分布. 这不是偶然的,实际上所有的正则博弈的混合策略均衡都可以用一系列(纯策略的)BNE\emph{纯化}.\index{纯化}

考虑一个正则博弈$(I,A,u)$. 给定一个扰动参数$\epsilon>0$,定义类型为$\theta=(\theta_i)_{i\in I}$,将收益扰动为:
\[\tilde u_i(s,\theta)=u_i(s)+\epsilon\theta_i,\quad\theta_i\in[-1,1].\]
假设$\theta_i\sim F_i$,相互独立,$F_i$是具有连续可微密度的分布. 如此就形成了一个\emph{扰动博弈}\index{博弈!扰动~}. 当扰动参数$\epsilon\to 0$时候,扰动博弈的BNE趋于正则博弈的混合策略均衡,这正是下面的\emph{Harsanyi纯化定理}\index{Harsanyi纯化定理}。

\begin{theorem}[Harsanyi纯化定理]\label{thm:har}\index{Harsanyi纯化定理}
给定玩家集$I$和行动空间$A$. 对于一般的收益函数$u$和连续分布族$\{F_i\}_{i\in I}$,对任意完全信息正则博弈$(I,A,u)$的混合策略Nash均衡$\sigma$,存在一列扰动博弈纯策略BNE $s_\epsilon$,当扰动参数$\epsilon\to 0$,$s_\epsilon\to \sigma$.
\end{theorem}
混合策略均衡可以被看作不确定性趋于消失的时候的纯策略均衡. 这一定理的原始证明需要用到Brouwer不动点定理和隐函数定理,并且比较长,这里略去.

人们常说
\begin{quotation}
``Decision makers do not flip coins in the real world.''
\end{quotation}

然而,如果玩家对收益的信念有微小的不确定性的时候,他的行为就仿佛在抛硬币. 这是混合策略的似然解释(主观概率论).

\lhysays{细化这一部分。
{混合策略的进一步讨论}
\begin{itemize}
    \item 我们之前说过,Bayes博弈对于玩家信念的刻画是相当受限制的.
    \begin{itemize}
        \item 当引入不确定性、知识、信念的概念的时候,几乎不可避免需要加入限制条件.
    \end{itemize}
    \item 另一方面,概率论的Bayes学派解释在哲学上也有很多争议.
    \begin{itemize}
        \item 一旦使用Bayes学派的概率论研究不确定性、知识、信念,这一问题也是不可避免的.
    \end{itemize}
    \item 因而,我们可以考虑完全理性、完全耐心玩家在无穷轮重复的完全信息博弈中的决策行为.
    \begin{itemize}
        \item 玩家做出行动$a$的极限频率就是行动$a$在混合策略中的概率.
    \end{itemize}
    \item 这一角度并不涉及不确定性、信念等数学上模糊的概念,单纯讨论混合均衡达到的方式.
\end{itemize}
}