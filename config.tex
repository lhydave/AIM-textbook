% font set
\usepackage{fontspec}
\usepackage[T1]{fontenc}
\usepackage[sc]{mathpazo}
\usepackage{anyfontsize}
\setmainfont{Source Serif 4}
\setsansfont{Source Sans 3}
\setmonofont{Menlo}
\setCJKmainfont[BoldFont=黑体-简 中等,ItalicFont=楷体-简 常规体]{宋体-简 常规体}

% colors
\usepackage[dvipsnames]{xcolor}
\definecolor{pku-red}{RGB}{139,0,18}
\usepackage{colortbl}
\newcommand{\light}[1]{\textcolor{red}{#1}}

% environments
\usepackage{array}
\usepackage{caption}
\usepackage[shortlabels,inline]{enumitem}
\usepackage{booktabs}
\usepackage{makecell}
\usepackage{float}
\usepackage{multicol}
\usepackage{etoolbox}
\usepackage{graphicx}
\usepackage{multirow}
\usepackage{inputenx}
\usepackage{subcaption}


\usepackage[normalem]{ulem}

% math package
\let\Bbbk\relax
\usepackage{amsmath}
\usepackage{mathrsfs}
\usepackage{amssymb}
\usepackage{amsfonts}
\usepackage{stmaryrd}
\usepackage{latexsym}
\usepackage{extarrows}
\SetSymbolFont{stmry}{bold}{U}{stmry}{m}{n}


% math theorems
\usepackage{mdframed}
\usepackage[thmmarks,amsmath]{ntheorem}
\newtheorem{construction}{构造}[chapter]
\newtheorem{observation}{观察}[chapter]
\newtheorem{lemma}{引理}[chapter]
\newtheorem{proposition}{命题}[chapter]
\newtheorem{corollary}{推论}[chapter]
\newtheorem{theorem}{定理}[chapter]
\newtheorem{principle}{规则}[chapter]
\newtheorem{axiom}{公理}[chapter]
{
\theoremstyle{plain}
\theoremheaderfont{\bfseries}
\theorembodyfont{\normalfont}
\newtheorem{definition}{定义}[chapter]
}
{
\theoremstyle{plain}
\theoremheaderfont{\bfseries}
\theorembodyfont{\normalfont}
\newtheorem{example}{例}[chapter]
}
{
\theoremstyle{nonumberplain}
\theorembodyfont{\normalfont}
\newmdtheoremenv{remark}{注. }
\AtBeginEnvironment{remark}{\small}
}
{
\theoremstyle{nonumberplain}
\theoremheaderfont{\bfseries}
\theorembodyfont{\normalfont}
\theoremsymbol{\ensuremath{\Box}}
\newtheorem{proof}{证明. }
}

% chap/section set
\ctexset{
    chapter={format={\raggedright\bfseries\Huge}},
    section={format={\raggedright\bfseries\LARGE},name={\S,}}
}
% appendix section
\newenvironment{appsec}{
    \ctexset{section={format={\raggedright\bfseries\Large},name={\S,}}}\footnotesize
}{}

% page style set
\usepackage[a4paper,scale=0.7]{geometry}
\usepackage{nonumonpart}

% automation
\usepackage{makeidx}
\makeindex

% cleveref
\usepackage{crossreftools}
% \pdfstringdefDisableCommands{%
%     \let\Cref\crtCref
%     \let\cref\crtcref
% }
\usepackage{hyperref}
\hypersetup{
    colorlinks=true,
    linkcolor=pku-red,     
    urlcolor=violet,
    citecolor=BlueViolet,
    pdffitwindow=true,
}
\usepackage[capitalize, nameinlink]{cleveref}

% customize Chinese cref name
\Crefformat{ALC@unique}{第#2#1#3行}
\Crefformat{construction}{构造\,#2#1#3}
\Crefformat{part}{第#2#1#3部分}
\Crefformat{chapter}{第#2#1#3章}
\Crefformat{section}{第\,#2#1#3\,节}
\Crefformat{theorem}{定理\,#2#1#3}
\Crefformat{lemma}{引理\,#2#1#3}
\Crefformat{proposition}{命题\,#2#1#3}
\Crefformat{corollary}{推论\,#2#1#3}
\Crefformat{observation}{观察\,#2#1#3}
\Crefformat{definition}{定义\,#2#1#3}
\Crefformat{remark}{注\,#2#1#3}
\Crefformat{example}{例\,#2#1#3}
\Crefformat{figure}{图\,#2#1#3}
\Crefformat{table}{表\,#2#1#3}
\Crefformat{appendix}{附录\,#2#1#3}

% remarks
\newcommand\lhysays[1]{\textcolor{blue}{[lhy: #1]}}

% exercise set
\newenvironment{hint}{% hint
\par\itshape 提示:
}{}

\renewcommand{\theenumii}{\arabic{enumii}}

% algorithm environment
\usepackage{algorithm}
\usepackage{algorithmic}
\renewcommand{\algorithmicrequire}{\textbf{输入:}}
\renewcommand{\algorithmicensure}{\textbf{输出:}}
\floatname{algorithm}{算法}

% plots
\usepackage{tikz}
\usepackage{tikz-cd}
\usetikzlibrary{arrows}
\usetikzlibrary{arrows.meta,positioning}
\usepackage{pgfplots}
\pgfplotsset{compat=1.18}

% math notations
\newcommand{\LHS}{\mathrm{LHS}}
\newcommand{\RHS}{\mathrm{RHS}}
\newcommand{\Z}{\mathbb{Z}}
\newcommand{\N}{\mathbb{N}}
\newcommand{\R}{\mathbb{R}}
\newcommand{\Q}{\mathbb{Q}}
\newcommand{\C}{\mathbb{C}}
\renewcommand{\O}{\mathcal{O}}
\newcommand{\id}{\mathrm{id}}
\DeclareMathOperator*{\Span}{Span}
\DeclareMathOperator*{\im}{Im}
\DeclareMathOperator*{\rank}{rank}
\DeclareMathOperator*{\card}{card}
\DeclareMathOperator*{\grad}{grad}
\DeclareMathOperator*{\argmax}{argmax}
\DeclareMathOperator*{\epi}{epi}
\renewcommand{\d}{\mathrm{d}}
\newcommand{\Pow}{\mathcal{P}}
\newcommand{\E}{\mathbb{E}}
\newcommand{\cov}{\mathsf{Cov}}
\newcommand{\var}{\mathsf{Var}}
\newcommand{\Nor}{\mathcal{N}}
\newcommand{\U}{\mathcal{U}}
\renewcommand{\t}{\mathsf{T}}
\newcommand{\T}{\top}
\newcommand{\F}{\bot}
\newcommand{\norm}[1]{\left\|#1\right\|}
\newcommand{\inner}[2]{\left\langle{#1},{#2}\right\rangle}
\newcommand{\e}{\mathrm{e}}
\newcommand{\const}{\mathrm{const}}

\DeclareSymbolFont{symbolsC}{U}{txsyc}{m}{n}
\DeclareMathSymbol{\strictif}{\mathrel}{symbolsC}{74}