\chapter{引言}

何谓智能?从历史的进程来看,生命的演化似乎就是智能的演化. 生命最初的形态很可能是单细胞生物,它能够维持自身的结构,并通过吸收外界的物质来维持自己这样的结构. 此时,生物已经可以选择性吸收外界的物质,而将不需要的物质排除在外,这已经可以被看成一种极其简单的智能. 

随着的演化进行,生命的形态变得越来越复杂. 在人的这条演化线上,我们经历了多细胞的、类似线虫的动物,到了爬行动物,然后变成哺乳动物,再到了灵长类动物,最后是人类. 在这个演化的过程中,生命对于外界的反应变得越来越复杂,智能也变得越来越强大. 

那么,应该怎么定义智能呢?困难在于,智能并不是一个非黑即白的东西,我们不能画一条线然后说,这边的是智能,那边的是非智能. 正如同生命演化的历程,智能也是一个多层次渐变的过程. 

从生命的观点来看,\textit{智能意味着一种对不断变化的外界的适应能力},越高级的智能能够考虑的外界因素就越复杂,适应变化的能力也越强. 例如,秀丽隐杆线虫是一种简单的有神经系统的生物,它能够机械地对外界刺激做出反应,一切行为似乎是写好的,没有任何自主性. 当我们把目光转移到蜥蜴,一种爬行动物,我们发现他们的行为似乎更加灵活,蜥蜴似乎可以为了某个目标(比如进食)而做出一些行为. 而到了人类,我们的行为准则中会把其他人纳入考虑,我们会考虑自己的行为对他人的影响,会去设身处地地考虑他人的感受,形成\textit{社会规范}. 

从演化的这个路径来看,智能有着两个维度:一个是\textit{从认知到决策},另一个是\textit{从个体到多体}. 
\begin{itemize}
    \item 从认知到决策:智能体为了适应外界,必须要先知道外界的情况,然后根据自己的目标做出相应的选择. 例如,如果我们感到口渴,我们会先扫视桌面,找到水杯,然后才会拿起水杯喝水. 因此,智能需要解决的第一个问题是\textit{认知},即如何获取外界的信息,然后是\textit{决策},即如何根据自己的目标利用这些信息做出选择. 
    \item 从个体到多体:在自然界中,生命面对的最大变数就是其他生命. 因此,演化过程中,最大的选择压力其实来自其他生命. 对于秀丽隐杆线虫,他们似乎只会对外界刺激做出反应,而不会考虑其他生命的行动. 而到了人类(甚至猩猩),我们会根据自己的想象,去揣测其他人的意图,从而做出自己的选择. 因此,智能不仅需要有能够面对变化环境的适应能力,更需要有能够面对其他生命的适应能力. 
\end{itemize}

因此,如果我们要实现“人工智能”(AI,Artificial Intelligence),我们需要面对的问题就是,如何实现这两个维度的智能. 本书的组织结构就是按照这两个维度来展开的. 书一共被分为了五个部分:
\begin{itemize}
    \item \Cref{part:AI-logic},AI的逻辑:这部分探究AI的认知是什么样的,如何被数学建模. 我们将会探究如何用Bayes概率论建立AI的推理模型(\Cref{chap:plausible-reasoning}),以及如何用Markov链建立AI包含时间的认知与决策模型(\Cref{chap:markov-chain}). 
    \item  \Cref{part:information-data},信息与数据:这部分探究AI如何从环境中获得认知,而现实中,这就是关于信息与数据的问题. 我们将会展示信息论的基本事实(\Cref{chap:information-theory}),AI面对数据的特性(\Cref{chap:J-L-Lemma}),以及如何在保护个人隐私的前提下让AI利用数据(\Cref{chap:differential-privacy}). 
    \item  \Cref{part:decision-optimization},决策与优化:这部分探究AI如何面对环境做出决策以及优化. 我们将会给出优化的基本概念以及特别重要的一类优化问题——凸优化(\Cref{chap:convex-analysis}),以及如何处理带约束的优化问题(\Cref{chap:duality}),最后我们将会给出“多体优化”的基本工具——不动点理论(\Cref{chap:fixed-point-theory}). 
    \item  \Cref{part:logic-game},博弈与逻辑:这部分探究AI如何面对其他个体做出决策以及优化,即博弈论,它也为多体智能的研究提供了一个标准的语言(\Cref{chap:game}). 
    \item  \Cref{part:cognitive-logic},认知与逻辑:这部分探究AI如形成对其他个体的认知,特别是如何在数学上建模这种认知. 我们将给出两种风格的数学模型:基于Bayes概率论的Bayes博弈(\Cref{chap:bayesian-game})和基于形式逻辑的模态逻辑(\Cref{chap:modal-logic}). 
\end{itemize}

